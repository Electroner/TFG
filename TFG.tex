\documentclass[a4paper,11pt]{book}
%\documentclass[a4paper,twoside,11pt,titlepage]{book}
\usepackage{listings}
\usepackage{fancyhdr}
\usepackage[utf8]{inputenc}
\usepackage[spanish]{babel}

% \usepackage[style=list, number=none]{glossary} %
%\usepackage{titlesec}
%\usepackage{pailatino}

\decimalpoint
\usepackage{dcolumn}
\newcolumntype{.}{D{.}{\esperiod}{-1}}
\makeatletter
\addto\shorthandsspanish{\let\esperiod\es@period@code}
\makeatother


%\usepackage[chapter]{algorithm}
\RequirePackage{verbatim}
%\RequirePackage[Glenn]{fncychap}
\usepackage{fancyhdr}
\usepackage{graphicx}
\usepackage{rotating}
\usepackage{afterpage}
\usepackage{eurosym}
\usepackage{etoolbox}
\usepackage{textcase}
\usepackage{longtable}
\usepackage{wrapfig}
\usepackage{float}

\usepackage[pdfborder={1 1 1}]{hyperref} %referencia

% ********************************************************************
% Re-usable information
% ********************************************************************
\newcommand{\myTitle}{ModernWood\xspace}
\newcommand{\myDegree}{Grado en INGENIERIA INFORMATICA\xspace}
\newcommand{\myName}{Carlos López Martínez\xspace}
\newcommand{\myProf}{Jesús González Peñalver\xspace}
\newcommand{\myOtherProf}{Jesús González Peñalver\xspace}
%\newcommand{\mySupervisor}{Put name here\xspace}
\newcommand{\myFaculty}{Escuela Técnica Superior de Ingenierías Informática y de
Telecomunicación\xspace}
\newcommand{\myFacultyShort}{E.T.S. de Ingenierías Informática y de
Telecomunicación\xspace}
\newcommand{\myDepartment}{Departamento de ...\xspace}
\newcommand{\myUni}{\protect{Universidad de Granada}\xspace}
\newcommand{\myLocation}{Granada\xspace}
\newcommand{\myTime}{\today\xspace}
\newcommand{\myVersion}{Version 0.1\xspace}


\hypersetup{
  pdfauthor = {\texorpdfstring{\myName}{Nombre Alternativo}},
  pdftitle = {\texorpdfstring{\myTitle}{Título Alternativo}},
  pdfsubject = {},
  pdfkeywords = {palabra\_clave1, palabra\_clave2, palabra\_clave3, ...},
  pdfcreator = {LaTeX con el paquete ....},
  pdfproducer = {pdflatex}
}


%\hyphenation{}


%\usepackage{doxygen/doxygen}
%\usepackage{pdfpages}
\usepackage{url}
\usepackage{colortbl,longtable}
\usepackage[stable]{footmisc}
%\usepackage{index}

%\makeindex
%\usepackage[style=long, cols=2,border=plain,toc=true,number=none]{glossary}

% Definición de comandos que me son tiles:
%\renewcommand{\indexname}{Índice alfabético}
%\renewcommand{\glossaryname}{Glosario}

\usepackage[nonumberlist]{glossaries}
\loadglsentries{Glosario/Glosario}
\makenoidxglossaries
% Agrega más entradas según sea necesario


\pagestyle{fancy}
\fancyhf{}
\fancyhead[LO]{\leftmark}
\fancyhead[RE]{\rightmark}
\fancyhead[RO,LE]{\textbf{\thepage}}
\renewcommand{\chaptermark}[1]{\markboth{\textbf{#1}}{}}
\renewcommand{\sectionmark}[1]{\markright{\textbf{\thesection. #1}}}

\setlength{\headheight}{1.5\headheight}

\newcommand{\HRule}{\rule{\linewidth}{0.5mm}}
%Definimos los tipos teorema, ejemplo y definición podremos usar estos tipos
%simplemente poniendo \begin{teorema} \end{teorema} ...
\newtheorem{teorema}{Teorema}[chapter]
\newtheorem{ejemplo}{Ejemplo}[chapter]
\newtheorem{definicion}{Definición}[chapter]

\definecolor{gray97}{gray}{.97}
\definecolor{gray75}{gray}{.75}
\definecolor{gray45}{gray}{.45}
\definecolor{gray30}{gray}{.94}

\lstset{ frame=Ltb,
     framerule=0.5pt,
     aboveskip=0.5cm,
     framextopmargin=3pt,
     framexbottommargin=3pt,
     framexleftmargin=0.1cm,
     framesep=0pt,
     rulesep=.4pt,
     backgroundcolor=\color{gray97},
     rulesepcolor=\color{black},
     %
     stringstyle=\ttfamily,
     showstringspaces = false,
     basicstyle=\scriptsize\ttfamily,
     commentstyle=\color{gray45},
     keywordstyle=\bfseries,
     %
     numbers=left,
     numbersep=6pt,
     numberstyle=\tiny,
     numberfirstline = false,
     breaklines=true,
   }
 
% minimizar fragmentado de listados
\lstnewenvironment{listing}[1][]
   {\lstset{#1}\pagebreak[0]}{\pagebreak[0]}

\lstdefinestyle{CodigoC}
   {
	basicstyle=\scriptsize,
	frame=single,
	language=C,
	numbers=left
   }
\lstdefinestyle{CodigoC++}
   {
	basicstyle=\small,
	frame=single,
	backgroundcolor=\color{gray30},
	language=C++,
	numbers=left
   }

 
\lstdefinestyle{Consola}
   {basicstyle=\scriptsize\bf\ttfamily,
    backgroundcolor=\color{gray30},
    frame=single,
    numbers=none
   }


\newcommand{\bigrule}{\titlerule[0.5mm]}


%CUSTOM CONSOLE STYLE
\usepackage{xcolor}

% Definir colores personalizados
\definecolor{codegreen}{rgb}{0,0.6,0}
\definecolor{codegray}{rgb}{0.5,0.5,0.5}
\definecolor{codepurple}{rgb}{0.58,0,0.82}
\definecolor{backcolour}{rgb}{0.95,0.95,0.95} % Cambiado a un gris claro

% Configuración de listings para código de consola
\lstdefinestyle{console}{
    backgroundcolor=\color{backcolour},   
    commentstyle=\color{codegray},
    keywordstyle=\color{codegray},
    numberstyle=\tiny\color{codegray},
    stringstyle=\color{codegray},
    basicstyle=\footnotesize\ttfamily,
    breakatwhitespace=false,         
    breaklines=true,                 
    captionpos=b,                    
    keepspaces=true,                 
    numbers=none,                    
    numbersep=5pt,                  
    showspaces=false,                
    showstringspaces=false,
    showtabs=false,                  
    tabsize=2,
    frame=single,
    moredelim=[is][\textcolor{codepurple}]{|}{|},
}

%Para conseguir que en las páginas en blanco no ponga cabecerass
\makeatletter
\def\clearpage{%
  \ifvmode
    \ifnum \@dbltopnum =\m@ne
      \ifdim \pagetotal <\topskip
        \hbox{}
      \fi
    \fi
  \fi
  \newpage
  \thispagestyle{empty}
  \write\m@ne{}
  \vbox{}
  \penalty -\@Mi
}
\makeatother

\usepackage{pdfpages}
\fancyhf{}
\fancyfoot[C]{\thepage}
\fancyhead[LO,RE]{\leftmark} % Título del capítulo
\fancyhead[LE,RO]{\rightmark} % Título de la sección

\renewcommand{\sectionmark}[1]{\markright{\thesection. \ #1}}  % Activa la configuración de la sección
\renewcommand{\subsectionmark}[1]{\markright{\thesubsection. \ #1}}  % Activa la configuración de la subsección

%Colores para los 
\renewcommand*{\glstextformat}[1]{\textcolor{blue}{#1}}
\hypersetup
{
    colorlinks=true,
    linkcolor={black}, %para el indice
    citecolor={blue},
    urlcolor={black},
    filecolor={black},
    linktoc=all,
}

\newcommand{\glsnocase}[1]{%
  \glsdisp{#1}{\MakeTextLowercase{#1}}%
}

\begin{document}

\begin{titlepage}
\newlength{\centeroffset}
\setlength{\centeroffset}{-0.5\oddsidemargin}
\addtolength{\centeroffset}{0.5\evensidemargin}
\thispagestyle{empty}

\noindent\hspace*{\centeroffset}\begin{minipage}{\textwidth}
\centering
\includegraphics[width=0.9\textwidth]{imagenes/logo_ugr.png}\\[1.4cm]

\textsc{\Large TRABAJO FIN DE GRADO\\[0.2cm]}
\textsc{INGENIERÍA INFORMÁTICA}\\[1cm]
{\Huge\bfseries Diseño y Desarrollo de un Dispositivo de Interfaz Humana\\}
\noindent\rule[-1ex]{\textwidth}{3pt}\\[3.5ex]
{\large\bfseries ModernWood}
\end{minipage}
\vspace{1.5cm}

\noindent\hspace*{\centeroffset}\begin{minipage}{\textwidth}
\centering

\textbf{Autor}\\ {Carlos López Martínez}\\[2.5ex]
\textbf{Directores}\\
Jesús González Peñalver\\[1.5cm]

\begin{center}
\hspace{-3em}
  \begin{minipage}{0.25\textwidth}
    \centering
    \includegraphics[width=\linewidth]{imagenes/etsiit_logo.png}
  \end{minipage}
  \hspace{1em}
  \vline
  \hspace{3em}
  \begin{minipage}{0.1\textwidth}
    \centering
    \begin{minipage}{1\textwidth}
        \includegraphics[width=\linewidth]{imagenes/QRGithub.png}
    \end{minipage}%
    \begin{minipage}{1\textwidth}
        Github
    \end{minipage}
  \end{minipage}
\end{center}

\textsc{Escuela Técnica Superior de Ingenierías Informática y de Telecomunicación}\\
\textsc{---}\\
Granada, \today
\end{minipage}
\end{titlepage}
\chapter*{}
%\thispagestyle{empty}
%\cleardoublepage

%\thispagestyle{empty}

\input{portada/portada_2}



\cleardoublepage
\thispagestyle{empty}

\begin{center}
{\large\bfseries Diseño y Desarrollo de un Dispositivo de Interfaz Humana: Teclado ISO 105 Español}\\
\end{center}
\begin{center}
Carlos López Martínez\\
\end{center}

%\vspace{0.7cm}
\noindent{\textbf{Palabras clave}: ISO, PlatformIO, Microcontrolador, ESP32-S3, Bluetooth, Wireless, USB, Windows, Linux}\\

\vspace{0.7cm}
\noindent{\textbf{Resumen}}\\

Creación de un dispositivo de interfaz humana siguiendo el estándar ISO 105 en español. Enfocado en la creación de un dispositivo de interfaz humana, este proyecto se propone diseñar y desarrollar un teclado conforme al estándar ISO 105 en español, ya que no existe gran variedad de estos en el mercado. Para alcanzar este objetivo, se empleará la plataforma "PlatformIO" junto con un microcontrolador ESP32-S3, aprovechando sus capacidades de conectividad y procesamiento. El dispositivo permitirá una interacción con SO Windows y Linux por USB (Cableado) y Bluetooth (Wireless)

Para todo el desarrollo del dispositivo se tendrá en cuenta el coste del mismo, ya que deberá ser abordable, al menos como producto de alta gama, así como se tendrá en cuenta su reparabilidad, estética y en todo momento el rendimiento y funcionamiento.
El dispositivo tiene que funcionar de forma correcta tanto de forma inalámbrica como cableada independientemente, pudiéndose bajar los costes si solo se desea una versión cableada.

El dispositivo dispondrá de una pantalla LCD a color para mostrar información y configuración del mismo, así como código ampliable y su asignación a teclas especiales.
\cleardoublepage


\thispagestyle{empty}


\begin{center}
{\large\bfseries Design and Development of a Human Interface Device: ISO 105 Spanish Keyboard}\\
\end{center}
\begin{center}
Carlos López Martínez\\
\end{center}

%\vspace{0.7cm}
\noindent{\textbf{Palabras clave}: ISO, PlatformIO, microcontroller, ESP32-S3, Bluetooth, Wireless, USB, Windows, Linux}\\

\vspace{0.7cm}
\noindent{\textbf{Abstract}}\\

Creation of a human interface device following the ISO 105 standard in Spanish. Focused on the creation of a human interface device, this project aims to design and develop a keyboard in accordance with the ISO 105 standard in Spanish, as there is not a wide variety of these in the market. To achieve this goal, the "PlatformIO" platform will be used along with an ESP32-S3 microcontroller, taking advantage of its connectivity and processing capabilities. The device will allow interaction with Windows and Linux OS via USB (Wired) and Bluetooth (Wireless).

Throughout the development of the device, its cost will be taken into account, as it must be affordable, at least as a high-end product. Its repairability, aesthetics, and performance and operation will also be considered at all times. The device has to function correctly both wirelessly and wired independently. Costs can be reduced if only a wired version is desired.

The device will have a color LCD screen to display information and configuration of the same. As well as expandable code and its assignment to special keys.

\chapter*{}
\thispagestyle{empty}

\noindent\rule[-1ex]{\textwidth}{2pt}\\[4.5ex]

Yo, \textbf{Carlos López Martínez}, alumno de la titulación INGENIERÍA INFORMÁTICA de la \textbf{Escuela Técnica Superior
de Ingenierías Informática y de Telecomunicación de la Universidad de Granada}, con DNI 20888530E, autorizo la
ubicación de la siguiente copia de mi Trabajo Fin de Grado en la biblioteca del centro para que pueda ser
consultada por las personas que lo deseen.

\vspace{3cm}

\begin{flushleft}
       \hspace*{1.25cm}\includegraphics[width=0.2\linewidth]{prefacios/Firma.jpg}
\end{flushleft}
\noindent Fdo: Carlos López Martínez

\vspace{2cm}

\begin{flushright}
\today
\end{flushright}


\chapter*{}
\thispagestyle{empty}

\noindent\rule[-1ex]{\textwidth}{2pt}\\[4.5ex]

D. \textbf{Jesús González Peñalver}, Catedrático del departamento de Arquitectura y Tecnología de Computadores de la Universidad de Granada.

\vspace{0.5cm}

\textbf{Informa:}

\vspace{0.5cm}

Que el presente trabajo, titulado \textit{\textbf{Diseño y Desarrollo de un Dispositivo de Interfaz Humana}},
ha sido realizado bajo su supervisión por \textbf{Carlos López Martínez}, y autoriza la defensa de dicho trabajo ante el tribunal
que corresponda.

\vspace{0.5cm}

Y para que conste, expiden y firman el presente informe en Granada a \today

\vspace{1cm}

\textbf{El director:}

\vspace{5cm}

\noindent \textbf{Jesús González Peñalver}

\chapter*{Agradecimientos}
\thispagestyle{empty}

       \vspace{1cm}

A mi familia, por todo el apoyo que me han dado a lo largo de todos
los años. A mis amigos por las ideas y momentos de inspiración. Y a
todos los profesores que me han ayudado y han logrado hacer del Grado de
Ingeniería Informática una gran experiencia.
%\frontmatter
\mainmatter
\tableofcontents
%\listoffigures
%\listoftables
%

\setlength{\parskip}{5pt}

\chapter{Introducción}
\section{Motivación}

Cada vez es más común el uso de ordenadores y sistemas informáticos en todos los ámbitos de la vida, por lo que el uso de dispositivos para poder interactuar con estos también es igual de común. Desde hace muchísimo tiempo, el dispositivo más usado para interactuar con los ordenadores y máquinas a lo largo del mundo es y ha sido el teclado. Existen teclados de todo tipo, formas, distribuciones y tecnologías. Todos ellos con el mismo objetivo, introducir caracteres en el sistema.

Personalmente, desde que puedo recordar siempre he usado el ordenador de casa y he crecido con el teclado de mi lado. He pasado por muchos teclados a lo largo de mi vida y todos ellos han acabado más o menos de la misma forma, estropeados o guardados porque me cansaba del teclado y de como era la sensación al usarlo. Reparar estos teclados era igual de costoso que el propio problema que tenían.

Intente buscar siempre dispositivos que satisficieran mis necesidades, robusto, fácil de arreglar, duraderos, inalámbricos y elegantes. En el mercado existen varios de estos, pero era mi último requisito el que hacía que mi búsqueda no arrojase ningún resultado, que fuera en español. Al fin y al cabo escribo en español y estoy acostumbrado a una distribución española, por lo que dado las exigencias que tengo como usuario y que no hay mercado para lo que busco decidí embarcarme en este proyecto para llenar el hueco que hay en el mercado y satisfacer mis demandas.

Mi desafío no solo radica en encontrar el teclado perfecto para mí, sino también en transformar la concepción convencional de diseño y fabricación de teclados. Para abordar este problema, mi enfoque va más allá de simplemente crear otro dispositivo; mi objetivo es redefinir lo que se considera estándar mediante la implementación de nuevas técnicas y conceptos innovadores.

En lugar de adherirme a las técnicas de fabricación y ensamblaje convencionales, que a menudo resultan en dispositivos difíciles de reparar y poco robustos, planeo adoptar un enfoque disruptivo. La fabricación de la placa de circuito impreso (\gls{PCB}) será un elemento central de esta estrategia. Al diseñar una \gls{PCB} que priorice la facilidad de reparación y la durabilidad, se espera superar las limitaciones de los teclados tradicionales.

La estética del teclado también experimentará un cambio significativo en mi proyecto. En lugar de centrarme únicamente en el aspecto visual, mi enfoque se orientará hacia la funcionalidad y la simplicidad elegante. Esto implica la eliminación de partes estéticas innecesarias para priorizar la funcionalidad y la durabilidad. La forma seguirá a la función, y mi objetivo es crear un teclado que no solo sea una herramienta eficiente, sino también un testimonio de la belleza en la simplicidad.

La modularidad será una característica clave de mi diseño. La capacidad de desmontar y reemplazar fácilmente componentes permitirá a los usuarios personalizar su experiencia según sus necesidades específicas. Desde interruptores hasta placas de circuito, cada elemento será diseñado para ser modular, facilitando tanto la personalización como el mantenimiento.

En términos de conectividad, mi teclado ofrecerá versatilidad. No solo será inalámbrico, aprovechando la comodidad de la tecnología sin cables, sino que también contará con una opción por cable para situaciones donde la estabilidad de la conexión sea prioritaria. Esta dualidad busca ofrecer a los usuarios la flexibilidad necesaria para adaptarse a diferentes entornos y preferencias.

En resumen, mi proyecto busca romper las convenciones establecidas en el diseño y la fabricación de teclados. A través de la implementación de varias tecnologías, un enfoque renovado en la estética, así como la incorporación de características como la modularidad y el diseño simple, aspiro a crear un teclado que no solo satisfaga mis necesidades y pueda llegar a llenar ese vacío en el mercado existente y crear, de una manera u otra, un producto de buena calidad.
\pagebreak

\section{Estado actual de las alternativas}
%Sección
Actualmente, existen muchos teclados, formas, distribuciones, materiales y precios. \cite{cdw-keyboards} Dado que el nuestro, aunque pueda ser fabricado en grandes cantidades, va a ser fabricado como prototipo o una vez. Vamos a mirar las opciones que se conocen como \gls{DIY} o personalizados que son las correspondientes a este tipo de mercado donde se fabrican bajo un precio menos ajustado y en menos cantidades.

En el ámbito de los teclados \gls{DIY} o personalizados, se encuentran diversas opciones que permiten a los usuarios crear su propio dispositivo según sus preferencias y necesidades. Estas alternativas suelen destacar por ofrecer un mayor nivel de personalización en términos de diseño, disposición de teclas, interruptores y retro-iluminación, en comparación con los teclados convencionales o fabricados en masa. \cite{Diy-Keyboard-POPSC}

Una de las opciones más populares para este tipo de teclados son las placas base personalízales. Estas placas permiten a los usuarios seleccionar y soldar sus propios interruptores y estabilizadores, lo que brinda una libertad total en la elección de la disposición de las teclas. Además, suelen admitir la programación de macros y asignación de funciones a través de software.

También existe la posibilidad de elegir interruptores mecánicos específicos siendo una característica clave. Existen diversos tipos de interruptores, como los Cherry MX, Gateron, Kailh, entre otros, cada uno con características únicas en términos de tacto, recorrido y sonoridad. Además, acompañando a estos interruptores, siempre hay lo que se llaman como \gls{Keycaps}, que son los componentes que cubren los interruptores y que al fin y al cabo es lo que el usuario acaba tocando a la hora de escribir en el teclado.

Los teclados personalizados, aunque siempre ofrecen más opciones en cuanto a qué cosas se quieren, también suele conllevar un mayor gasto y por eso es necesario tener en cuenta que se quiere y como pueden cambiar los precios de estos.

%\newpage
\subsection{Conectividad}
En cuanto a conectividad, no hay tanta variedad, ya que esto se refiere a la forma de conectarse a un computador, y por el momento el grueso del mercado y de casi todos los teclados que se usan de forma comercial usan 4 tipos de conexión. 
\newpage
\begin{itemize}
    \item Por Cable \cite{Keyboards-connection-types-wired}
    \begin{itemize}
        \item \gls{USB} \\
            Esta forma de conectividad representa el grueso predominante en la actualidad, ya que prácticamente todos los dispositivos recurren a esta interfaz para establecer conexión con la computadora. Se trata de un estándar en constante evolución, actualizado mediante diversas versiones a lo largo del tiempo.  (\gls{USB} 1.0, \gls{USB} 2.0, \gls{USB} 3.0, \gls{USB} 3.1, \gls{USB} 3.2 ...)
        \item \gls{PS2} \\
            La conexión \gls{PS2} fue ampliamente utilizada en la década de 1990 y principios de la década de 2000 como un estándar para conectar periféricos a computadoras, especialmente dispositivos de entrada como teclados y ratones. Aunque ha sido superada en popularidad por el \gls{USB}, el \gls{PS2} todavía se encuentra en algunos dispositivos más antiguos. Esta interfaz se caracteriza por su conector redondo con pines y ha experimentado varias revisiones a lo largo del tiempo, como el \gls{PS2} estándar y el \gls{PS2} Mini-DIN de 6 pines. A medida que la tecnología ha avanzado, el \gls{PS2} ha quedado en gran medida relegado en favor de interfaces más modernas, pero su legado persiste en sistemas heredados y dispositivos retro. 
    \end{itemize}
    \item Por radiofrecuencia \cite{Keyboards-connection-types-wireless}
    \begin{itemize}
        \item \gls{Bluetooth} \\
            La conectividad por \glsnocase{Bluetooth} ha ganado amplia aceptación en el ámbito inalámbrico, facilitando la comunicación entre dispositivos a corta distancia. Este estándar ha demostrado ser especialmente útil para la conexión de periféricos como auriculares, teclados y ratones de forma inalámbrica. Este estándar también ha ido sufriendo actualizaciones que lo han mejorado en todos sus aspectos. Actualmente casi todos los teclados que son inalámbricos disponen de \glsnocase{Bluetooth}.
    
        \item \gls{Dongle} receptor \\
            El \gls{Dongle} receptor, también conocido como adaptador, desempeña un papel crucial al habilitar la conectividad por radiofrecuencia en dispositivos que no cuentan nativamente con la capacidad \glsnocase{Bluetooth}. Al conectar este pequeño dispositivo, se amplía la gama de dispositivos compatibles y se permite la comunicación inalámbrica. Este se trata de un dispositivo añadido al dispositivo que se conecta a un \gls{USB} y hace de antena receptora, normalmente para un protocolo que no es \glsnocase{Bluetooth} y más especifico a la aplicación, por lo que hace que su \glsnocase{Latencia} y alcance mejoren considerablemente.
    \end{itemize}
\end{itemize}

\subsection{Formatos}

A lo largo de la evolución de los teclados, las configuraciones y disposiciones de las teclas han experimentado cambios significativos para adaptarse a las necesidades cambiantes de los usuarios y a los avances tecnológicos.

Inicialmente, los teclados adoptaron el diseño \gls{QWERTY}, popularizado por las máquinas de escribir y posteriormente estandarizado por IBM. Este diseño se mantiene como el más común en la actualidad.

A lo largo de los años, surgieron alternativas, como el diseño Dvorak, que redistribuye las letras según su frecuencia de uso para aumentar la eficiencia de la escritura.

Para abordar preocupaciones ergonómicas, los teclados divididos surgieron con el objetivo de reducir la tensión al separar el teclado en dos secciones, ya sea físicamente o mediante un diseño ergonómico que coloca las secciones en ángulos más naturales para las manos.

En términos de tamaño y portabilidad, los teclados completos (100\%) ofrecen la disposición estándar, mientras que los teclados Tenkeyless (TKL) eliminan el teclado numérico para reducir el tamaño. Los teclados 75\% reducen aún más el tamaño al ajustar la disposición sin sacrificar funciones esenciales. Luego en el mundo de \gls{DIY} existen variaciones extremas de esto, pudiendo encontrar teclados de 50\%, 45\% y hasta un 35\% que solo disponen de las teclas alfabéticas.

La elección de un formato de teclado se basa en las preferencias y necesidades individuales del usuario, considerando aspectos como la ergonomía, la portabilidad y el uso previsto, ya sea para trabajo, juegos u otros propósitos específicos. \cite{Keyboards-types}

\subsection{Precios}
Cuando se trata de precios, la diversidad en el mercado de teclados refleja una amplia gama de opciones que se adaptan a diferentes presupuestos y necesidades. Desde opciones más asequibles hasta modelos de alta gama, los precios de los teclados varían según diversos factores.

Los teclados estándar con configuraciones tradicionales y cableado suelen ser más asequibles, brindando una opción económica para aquellos con presupuestos más ajustados. Estos teclados son ideales para usuarios que no necesitan características avanzadas o diseños especializados.

En el extremo superior del espectro, los teclados de gama alta ofrecen características avanzadas como retro-iluminación personalizable, interruptores mecánicos de alta calidad, y construcciones premium. Estos modelos suelen apuntar a entusiastas de los juegos, profesionales creativos o usuarios que buscan una experiencia de escritura excepcional.

Además, la introducción de teclados inalámbricos, ergonómicos o compactos también afecta los precios. Los teclados ergonómicos diseñados para reducir la fatiga y mejorar la comodidad pueden tener un costo ligeramente superior, mientras que los modelos inalámbricos ofrecen la ventaja de la movilidad a un precio adicional.

En resumen, la amplia variedad de teclados disponibles en el mercado garantiza que haya opciones para todos los presupuestos. La clave está en identificar las características prioritarias y el uso previsto para encontrar el equilibrio perfecto entre funcionalidad y precio. \cite{Keyboards-Prices}

\subsection{Otros teclados personalizados}
Siempre pueden haber otra razón para buscar un teclado personalizado, entre ellas podemos encontrar cosas como diseños retro y \glsnocase{Vintage}. Teclados Temáticos o de Edición Limitada. 
O también se hace si se quiere conseguir un objetivo principal, normalmente se quiere conseguir, a parte de todo lo mencionado, objetivos de peso, lograr una funcionalidad extra que un teclado normal no va a proveer o de forma extraordinaria una forma de presumir de las capacidades de fabricación y diseño del autor.
\chapter{Especificación del Sistema}

\section{Requisitos}

% Sección
\subsection{Requisitos funcionales} \label{RequisitosFuncionales}

\subsubsection{Conectividad por cable} \label{DiseñoConectividadCable}
\begin{itemize}
\item \textbf{RF-1:} El teclado deberá poder conectarse mediante un cable \gls{USB} para garantizar la compatibilidad con dispositivos sin capacidad \gls{Bluetooth}.
\item \textbf{RF-2:} Se deberá permitir la conexión y desconexión en caliente (\glsnocase{Hot-Plugging}) a través del cable \gls{USB} sin afectar el rendimiento del teclado.
\end{itemize}

\subsubsection{Conectividad Bluetooth} \label{DiseñoConectividadSinCable}
\begin{itemize}
\item \textbf{RF-3:} El teclado deberá ser capaz de establecer una conexión \gls{Bluetooth} con dispositivos compatibles.
\item \textbf{RF-4:} Deberá admitir el emparejamiento seguro con al menos un dispositivo a la vez.
\item \textbf{RF-5:} Se permitirá la conexión \gls{Bluetooth} mientras el dispositivo esté conectado por cable.
\item \textbf{RF-6:} Se permitirá la opción de apagar completamente el \gls{Bluetooth} del dispositivo.
\end{itemize}

\subsubsection{Batería/Energía y Alimentación}
\begin{itemize}
\item \textbf{RF-7:} El teclado contará con una batería recargable que proporcionará una autonomía mientras no esté conectado con cable.
\item \textbf{RF-8:} Se deberá incorporar un sistema de carga eficiente que permita recargar la batería mientras el teclado está conectado por cable.
\item \textbf{RF-9:} El teclado será capaz de funcionar mientras se carga para garantizar un uso continuo.
\item \textbf{RNF-10:} El teclado entrará en modo de suspensión automáticamente después de un período de inactividad para conservar energía.
\end{itemize}

\subsubsection{Distribución de Teclas}
\begin{itemize}
\item \textbf{RF-11:} El diseño del teclado seguirá la distribución \gls{ISO} de 105 teclas para cumplir con los estándares internacionales.
\end{itemize}

\subsubsection{Compatibilidad y Configuración}
\begin{itemize}
\item \textbf{RF-12:} El teclado será compatible con los principales sistemas operativos, como \gls{Windows} y \gls{Linux}.
\item \textbf{RF-13:} Deberá ser posible configurar las funciones y asignaciones de las teclas mediante \glsnocase{Firmware}, además la configuración del usuario se deberá guardar entre usos y apagados del teclado.
\end{itemize}

\subsection{Requisitos no funcionales} \label{RequisitosNoFuncionales}

\subsubsection{Estética y Diseño}
\begin{itemize}
\item \textbf{RNF-1:} El diseño del teclado deberá ser estéticamente agradable.
\item \textbf{RNF-2:} La calidad de los materiales utilizados en la fabricación garantizará durabilidad y resistencia al desgaste.
\item \textbf{RNF-3:} El producto será sencillo de montar y desmontar para su limpieza. El mantenimiento debe ser sencillo.
\end{itemize}

\subsubsection{Seguridad}
\begin{itemize}
\item \textbf{RNF-4:} El sistema de emparejamiento \gls{Bluetooth} deberá seguir protocolos de seguridad estándar para evitar accesos no autorizados.
\end{itemize}

\subsubsection{Rendimiento y Latencia} \label{DiseñoRendimiento}
\begin{itemize}
\item \textbf{RNF-5:} El teclado garantizará un rendimiento sin demoras perceptibles, manteniendo una baja \glsnocase{Latencia} durante la escritura y la conexión inalámbrica.
\item \textbf{RNF-6:} La respuesta de las teclas será consistente y proporcionará una experiencia de escritura fluida, independientemente de la conexión utilizada (\gls{Bluetooth} o por cable).
\end{itemize}

\subsubsection{Energía y Eficiencia} \label{DiseñoAhorroEnergia}
\begin{itemize}
\item \textbf{RNF-7:} Se implementarán medidas de ahorro de energía para optimizar la duración de la batería.
\end{itemize}

\subsubsection{Interfaz de Usuario} \label{DiseñoInterfazUsuario}
\begin{itemize}
\item \textbf{RNF-8:} La interfaz de usuario debe ser simple de entender y usar.
\end{itemize}

\section{Especificaciones}

% Sección
\subsection{Especificación Hardware}

\subsubsection{Conectividad por cable}
\begin{itemize}
\item La conexión por cable se realizará a través de un conector XS12 de aviación por su robustez y estética para asegurar una conexión estable y un menor desgaste.
\item El conector atornillado a la carcasa tendrá otro cable Micro JST de 4mm para poder separar la placa base de la carcasa de forma sencilla sin tener que desoldar nada.
\end{itemize}

\subsubsection{Conectividad \gls{Bluetooth}}
\begin{itemize}
\item El teclado estará equipado con un módulo \gls{Bluetooth} que permita garantizar una conexión estable.
\item El dispositivo \gls{Bluetooth}, debe poder establecer una conexión de al menos de 5 metros.
\end{itemize}

\subsubsection{Batería y Alimentación} \label{DiseñoBateriaAlimentacion}
\begin{itemize}
\item La batería recargable será de \glsnocase{Ion de litio} de alta capacidad. Que se ubicara en la parte inferior de la placa base.
\item La batería tendrá un conector para poder desconectarla de la placa base.
\end{itemize}

\subsubsection{Distribución de Teclas} \label{DiseñoDistribucion}
\begin{itemize}
\item Las teclas seguirán el estándar \gls{ISO} con disposición \gls{QWERTY} para adaptarse a los usuarios de habla hispana.
\item Se utilizarán interruptores mecánicos de alta calidad para garantizar una respuesta táctil precisa y duradera. La elección del tipo será del usuario por gusto personal.
\end{itemize}

\subsection{Especificación Software} \label{DiseñoSoftware}

\subsubsection{Configuración y Personalización}
\begin{itemize}
\item El \glsnocase{Firmware} del teclado admitirá perfiles personalizados que podrán almacenarse en la memoria interna del dispositivo.
\item El dispositivo permitirá una opción de crear programas completos que se podrán ejecutar en una tecla. Estos se programarán por \glsnocase{Firmware}.
\end{itemize}

\subsubsection{Compatibilidad con Sistemas Operativos}
\begin{itemize}
\item El teclado será compatible con \gls{Windows} y \gls{Linux}, garantizando una experiencia uniforme en diferentes plataformas.
\item En todo momento será \gls{Plug-and-Play} para facilitar la instalación y uso sin necesidad de \glsnocase{Controladores} adicionales.
\end{itemize}

\subsubsection{Actualizaciones de \glsnocase{Firmware}} \label{DiseñoActualizaciones}
\begin{itemize}
\item Se diseñará el \glsnocase{Firmware} del teclado con la capacidad de recibir actualizaciones, permitiendo mejorar la funcionalidad y corregir posibles vulnerabilidades de seguridad.
\item Las actualizaciones podrán aplicarse de manera sencilla mediante con conector para conectar un \gls{USB} A \gls{TTL}.
\end{itemize}

\subsubsection{Indicadores \gls{LED}}
\begin{itemize}
\item Se incluirán 10 \gls{LED}S para uso personalizado del usuario, tanto estético como funcional, para indicar estados de batería o conexión.
\item Los indicadores \gls{LED} serán configurables, permitiendo a los usuarios personalizar la apariencia y comportamiento de los mismos.
\end{itemize}

\section{Planificación}

Esta sección detalla el plan de desarrollo del teclado (\gls{HID}) con conectividad \gls{Bluetooth}, batería y conexión por cable. La planificación aborda las fases clave del proyecto, asignación de recursos y plazos de entrega. Para ello se ha realizado un gráfico de Gantt para poder organizar las diferentes fases del proyecto en hitos a cumplir.

El gráfico de Gantt realizado podemos verlo en la figura \ref{fig:DiagramaGantt} que señala de manera clara y concisa la planificación temporal del proyecto. Esta planificación se ha realizado para poder estimar mejor los tiempos de trabajo y que aspectos serían necesarios tener para poder progresar. Esto nos ayudará a saber en qué punto del proyecto nos encontramos y qué aspectos debemos mejorar para poder cumplir con los plazos establecidos. Además, nos ayudará a saber qué recursos necesitamos, tanto para estimar el coste del proyecto como para saber qué aspecto es necesario terminar antes para poder avanzar en el proyecto.

\begin{sidewaysfigure}
\centering
\includegraphics[width=\textheight]{imagenes/Capitulos/Cap02/DiagramaGantt.png}
\caption{Planificación del proyecto con diagrama Gantt.}
\label{fig:DiagramaGantt}
\end{sidewaysfigure}

\chapter{Diseño Y Prototipado} \label{CapDiseño}

\section{Teclados, entendiendo sus partes}

En esta primera sección vamos a ver todas las partes de un teclado y que funciones tienen cada una de ellas. Además, como interactúan entre sí y qué opciones hay. \cite{TearDownImageSource} La estructura general de un teclado es la siguiente.

\begin{figure}[H]
    \centering
    \includegraphics[width=1\textwidth]{imagenes/Capitulos/Cap03/KeyboardTeardown.png}
    \caption{Estructura física de un teclado \cite{TearDownImageSource}}
    \label{fig:TearDown}
\end{figure}

En esta figura podemos ver todas las diferentes partes de un teclado. Aunque es cierto que aquí se muestran muchas más de las necesarias y de las que se usan normalmente. Se van a exponer todas las partes que se muestran aquí y se explicaran de arriba a abajo y además se indicaran las que se van a usar.

\subsubsection{\gls{Keycaps}}

Los \glsnocase{Keycaps} son las tapas individuales que cubren las teclas en un teclado. Estas tapas suelen estar hechas de plástico y están diseñadas para permitir que los usuarios escriban y presionen las teclas de manera cómoda y precisa. Los \glsnocase{Keycaps} pueden variar en diseño, color, material y perfil, y algunos teclados personalizados incluso permiten a los usuarios intercambiar los \glsnocase{Keycaps} para personalizar la apariencia o mejorar la sensación táctil del teclado. Algunos entusiastas de los teclados también coleccionan \glsnocase{Keycaps} únicos y personalizados para crear conjuntos únicos y estéticamente atractivos.

\begin{figure}[H]
    \centering
    \includegraphics[width=0.75\textwidth]{imagenes/Capitulos/Cap03/Keycaps.png}
    \caption{Keycaps \cite{KeycapsImageSource}}
    \label{fig:Keycaps}
\end{figure}

\subsubsection{\gls{Switches} o Interruptores}

Los \glsnocase{Switches} son los mecanismos debajo de cada tecla de un teclado que detectan cuándo se presiona una tecla y envían la señal al dispositivo electrónico al que está conectado el teclado. Estos \glsnocase{Switches} pueden variar significativamente en términos de sensación táctil, fuerza de actuación y ruido producido al presionar la tecla.

\begin{itemize}
  \item \textbf{\gls{Switches} de membrana}: Estos son los más comunes y se utilizan en muchos teclados estándar. Consisten en una membrana de goma debajo de las teclas que se comprime cuando se presiona una tecla, cerrando un circuito eléctrico y enviando una señal al dispositivo.
  
  \item \textbf{\gls{Switches} de tijera}: Estos \glsnocase{Switches} utilizan una estructura de tijera debajo de las teclas para proporcionar una sensación de escritura más estable y una mejor respuesta táctil que los \glsnocase{Switches} de membrana estándar.
  
  \item \textbf{\gls{Switches} mecánicos}: Son \glsnocase{Switches} individuales que utilizan un mecanismo mecánico para registrar la pulsación de una tecla. Hay varios tipos de \glsnocase{Switches} mecánicos, incluyendo los populares \glsnocase{Switches} Cherry MX, que vienen en variantes como los \glsnocase{Switches} lineales, táctiles y clicky, que ofrecen diferentes sensaciones táctiles y niveles de ruido.
  
  \item \textbf{\gls{Switches} ópticos}: En lugar de utilizar contactos eléctricos, los \glsnocase{Switches} ópticos utilizan luz infrarroja para detectar cuando se presiona una tecla. Ofrecen una mayor durabilidad y una respuesta más rápida en comparación con los \glsnocase{Switches} mecánicos tradicionales.
\end{itemize}

\begin{figure}[H]
    \centering
    \includegraphics[width=1\textwidth]{imagenes/Capitulos/Cap03/Switches.png}
    \caption{Switches o Interruptores mecánicos \cite{SwitchesImageSource}}
    \label{fig:Switches}
\end{figure}

Los \glsnocase{Switches} son un componente clave en la experiencia de escritura de un teclado y pueden afectar la velocidad, la comodidad y la precisión al escribir. Los entusiastas de los teclados a menudo tienen preferencias personales sobre el tipo de \glsnocase{Switches} que prefieren, y algunos incluso personalizan sus teclados con \glsnocase{Switches} específicos para adaptarse a sus necesidades y preferencias individuales.
\newpage
\subsubsection{Carcasa superior}

La carcasa superior de un teclado es la parte que cubre la parte superior del teclado y proporciona la estructura externa. Esta carcasa puede variar en diseño y material según el tipo de teclado. En los teclados estándar, la carcasa superior suele estar hecha de plástico, mientras que en teclados de alta gama puede estar hecha de materiales más duraderos como aluminio o acero. La carcasa superior también puede incluir características adicionales, como reposamuñecas integrados o iluminación \gls{LED}, dependiendo del diseño y la funcionalidad del teclado. En nuestro caso vamos a combinar esta junto con la inferior para formar una combinada que proporciona soporte y estructura sin que tenga que ser otra pieza nueva.

\begin{figure}[H]
    \centering
    \includegraphics[width=1\textwidth]{imagenes/Capitulos/Cap03/TopCase2.png}
    \caption{Carcasa Superior \cite{TopCase2ImageSource}}
    \label{fig:TopCase2}
\end{figure}

\begin{figure}[H]
    \centering
    \includegraphics[width=1\textwidth]{imagenes/Capitulos/Cap03/TopCase1.png}
    \caption{Carcasa Superior Reverso \cite{TopCase1ImageSource}}
    \label{fig:TopCase1}
\end{figure}

\subsubsection{\gls{Plate} o Plancha} \label{DiseñoPlatePCB}

El \glsnocase{Plate}, también conocido como placa o plancha en español, es una pieza metálica o plástica que se coloca debajo de los \glsnocase{Switches} en un teclado mecánico. Su principal función es proporcionar soporte estructural a los \glsnocase{Switches} y distribuir uniformemente la fuerza ejercida al presionar las teclas sobre la superficie del teclado. Además, el \glsnocase{Plate} también influye en la sensación táctil y el sonido de las pulsaciones de las teclas.

La elección del material y diseño del \glsnocase{Plate} puede afectar la experiencia de escritura del usuario. Por ejemplo, los \glsnocase{Plate}s de acero ofrecen una mayor rigidez y pueden producir un sonido más nítido al escribir, mientras que los \glsnocase{Plate} de aluminio pueden proporcionar una sensación más suave y amortiguada. Algunos teclados personalizados permiten a los usuarios elegir entre diferentes materiales y grosores de \glsnocase{Plate} para adaptarse a sus preferencias individuales.

\begin{itemize}
    \item \textbf{Montaje en carcasa inferior}: Atornillada la \gls{PCB} a la carcasa.
    \begin{figure}[H]
        \centering
        \includegraphics[width=0.75\textwidth]{imagenes/Capitulos/Cap03/Montajes/Montaje1.png}
        \caption{Montaje en carcasa inferior \cite{Keyboards-Mounting-Styles}}
        \label{fig:Montaje1}
    \end{figure}
    
    \item \textbf{Montaje en carcasa superior}: Atornillada a la parte superior de la carcasa que luego se monta sobre la inferior.
    \begin{figure}[H]
        \centering
        \includegraphics[width=0.75\textwidth]{imagenes/Capitulos/Cap03/Montajes/Montaje2.png}
        \caption{Montaje en carcasa superior \cite{Keyboards-Mounting-Styles}}
        \label{fig:Montaje2}
    \end{figure}
    
    \item \textbf{Montaje en carcasa inferior escondida}: Atornillada a la carcasa inferior.
    \begin{figure}[H]
        \centering
        \includegraphics[width=0.75\textwidth]{imagenes/Capitulos/Cap03/Montajes/Montaje3.png}
        \caption{Montaje en carcasa \cite{Keyboards-Mounting-Styles}}
        \label{fig:Montaje3}
    \end{figure}
    
    \item \textbf{Montaje en carcasa sandwich}: Atornillada siendo atravesada por la carcasa inferior desde la parte de abajo hasta la carcasa superior.
    \begin{figure}[H]
        \centering
        \includegraphics[width=0.75\textwidth]{imagenes/Capitulos/Cap03/Montajes/Montaje4.png}
        \caption{Montaje en carcasa sandwich \cite{Keyboards-Mounting-Styles}}
        \label{fig:Montaje4}
    \end{figure}
    
    \item \textbf{\gls{Plate} como carcasa superior}: La \glsnocase{Plate} toma el rol de ser también la tapa o carcasa del teclado.
    \begin{figure}[H]
        \centering
        \includegraphics[width=0.75\textwidth]{imagenes/Capitulos/Cap03/Montajes/Montaje5.png}
        \caption{Montaje en forma de carcasa superior \cite{Keyboards-Mounting-Styles}}
        \label{fig:Montaje5}
    \end{figure}
    
    \item \textbf{Montaje en empaquetado}: Al igual que el sandwich solo que esta queda bloqueada por unos pasadores además de ser atornillada.
    \begin{figure}[H]
        \centering
        \includegraphics[width=0.75\textwidth]{imagenes/Capitulos/Cap03/Montajes/Montaje6.png}
        \caption{Montaje en empaquetado \cite{Keyboards-Mounting-Styles}}
        \label{fig:Montaje6}
    \end{figure}
    
    \item \textbf{Montaje sin \glsnocase{Plate}}: Este modo no usa \glsnocase{Plate} a la hora de construir el teclado.
    \begin{figure}[H]
        \centering
        \includegraphics[width=0.75\textwidth]{imagenes/Capitulos/Cap03/Montajes/Montaje7.png}
        \caption{Montaje sin \glsnocase{Plate} \cite{Keyboards-Mounting-Styles}}
        \label{fig:Montaje7}
    \end{figure}
    
\end{itemize}
\newpage
La opción sin \glsnocase{Plate} es la opción que se ha escogido para nuestro teclado. Ya que facilitara el diseño, abaratara el presupuesto y simplificara el montaje.

Para hacernos una idea esta seria una \glsnocase{Plate} convencional de un teclado 60\% son sus tornillos correspondientes.

\begin{figure}[H]
    \centering
    \includegraphics[width=1\textwidth]{imagenes/Capitulos/Cap03/Plate.png}
    \caption{Plate \cite{PlateImageSource}}
    \label{fig:Plate}
\end{figure}

\subsubsection{\gls{PCB}}

La \gls{PCB}, o Placa de Circuito Impreso, es una parte fundamental de un teclado mecánico. Es una placa de material aislante, generalmente fibra de vidrio o resina epoxi, sobre la cual están montados los \glsnocase{Switches} y otros componentes electrónicos del teclado. La \gls{PCB} proporciona la estructura física para el ensamblaje de los componentes del teclado y permite la conexión eléctrica entre ellos.

Los circuitos impresos en la \gls{PCB} dirigen las señales eléctricas de las teclas presionadas a través de los \glsnocase{Switches} hacia el controlador del teclado, que luego las envía al dispositivo al que está conectado el teclado, como una computadora o una consola de juegos. Además, la \gls{PCB} puede contener componentes adicionales, como diodos para la función anti-fantasma y \gls{LED}s para la retroiluminación.

La calidad y el diseño de la \gls{PCB} pueden afectar la durabilidad, la capacidad de personalización y la eficiencia energética del teclado. En teclados personalizados, la \gls{PCB} a menudo es una parte importante del diseño y puede ser programable para permitir la personalización de las funciones de las teclas.

\subsubsection{Espuma}

La espuma de amortiguación de sonido es un componente opcional que se puede agregar a un teclado mecánico para reducir el ruido producido por las pulsaciones de las teclas. Esta espuma se coloca generalmente entre la placa \gls{PCB} y la carcasa inferior del teclado, ayudando a absorber las vibraciones y el ruido generado por las pulsaciones de las teclas.

Aunque la espuma de amortiguación de sonido puede mejorar la experiencia auditiva al usar el teclado, su inclusión es opcional y depende de las preferencias personales del usuario. Algunas personas prefieren el sonido más nítido y distintivo de un teclado sin espuma de amortiguación, mientras que otras prefieren un teclado más silencioso y amortiguado.

La calidad y el tipo de espuma utilizada pueden afectar la eficacia de la amortiguación de sonido. Algunos teclados personalizados vienen con espumas específicamente diseñadas para reducir el ruido, mientras que otros pueden permitir que los usuarios añadan su propia espuma según sus necesidades y preferencias.

\subsubsection{Carcasa Inferior}

La carcasa inferior de un teclado es la parte que cubre la parte inferior del teclado y proporciona soporte estructural. Esta carcasa puede estar hecha de plástico u otros materiales resistentes y generalmente contiene la placa \gls{PCB} y otros componentes internos del teclado. Esta Proporciona una base sólida para los componentes internos y ayuda a protegerlos de daños externos.

La carcasa inferior puede tener características adicionales, como patas ajustables para elevar la inclinación del teclado, o canales de enrutamiento para cables, que mejoran la comodidad y la organización al usar el teclado.

\begin{figure}[H]
    \centering
    \includegraphics[width=1\textwidth]{imagenes/Capitulos/Cap03/BottomCase.png}
    \caption{Carcasa Inferior de Metal de teclado 60\%}
    \label{fig:BottomCase}
\end{figure}

\section{Investigación: Herramientas y Tecnologías}

La investigación y elección de herramientas y tecnologías son elementos fundamentales en el desarrollo de cualquier proyecto. En esta etapa, nos centraremos en identificar y evaluar diversas opciones disponibles en el panorama tecnológico actual, prestando especial atención a criterios clave como accesibilidad, robustez, comunidad de soporte y facilidad de integración. Abordaremos las necesidades específicas de cada fase del proyecto, desde el desarrollo de hardware hasta el software, con el objetivo de seleccionar herramientas y tecnologías que maximicen la eficiencia y la sinergia entre los componentes del sistema.

Aunque a lo largo de esta etapa se han llegado a encontrar alternativas mejores, estas supondrían un coste/beneficio demasiado alto para su desarrollo en el cuadro de tiempo dado.
Ya que estas suponen una curva de aprendizaje o coste de compra/uso demasiado alto para ser abordables. Aunque estas se mencionaran en cada sección y aparecerán descartadas por falta de tiempo o recursos. Aun así para alguien con el tiempo suficiente y la preparación necesaria serían sin duda la mejor elección.

En todas las secciones próximas se expondrán las elecciones realizadas en cada ámbito. Y también se expondrán todas y cada una de las alternativas que se hayan llegado a considerar junto a la explicación de porque han sido descartadas. Finalmente, se mostrará una tabla con los pros y contras para que cada uno pueda tomar sus decisiones a cerca de estas elecciones.

\section{Alternativas de diseño}
Aquí vamos a contemplar las diferentes opciones que tenemos a la hora de hacer un teclado, ya que como se ha mencionado antes, hay muchos formatos y tipos de teclados. Directamente y personalmente sé cuál va a ser el diseño elegido, ya que ando buscando un tipo de teclado que no he podido llegar a encontrar en decenas de páginas \glsnocase{Online} que he visitado y que cumplen las características siguientes:
\begin{itemize}
\item \gls{ISO} \\ La distribución será la \gls{ISO}, ya que mi idioma principal es el castellano.
\item 100\% o 105 \\ El teclado quiero que tenga todas las teclas que puede tener un teclado convencional, incluyendo el teclado numérico y teclas de función. En este caso, al ser \gls{ISO} la cantidad de teclas para llegar al 100\% son 105.
\item \gls{Bluetooth} \\ Conectividad inalámbrica.
\item Conectividad por cable \\ Para poder usar el teclado en un modo de baja latencia.
\item Interruptores Mecánicos. \\ Para asegurar todo tipo de sensaciones a la hora de escribir. En mi caso Interruptores lineales.
\end{itemize}

\subsection{Elección de las herramientas} \label{Herramientas}
Vamos a empezar por las herramientas de diseño, la primera de todas la que hemos usado para crear los esquemas y planos del teclado, tomar medidas y planificar todo el diseño.

\begin{itemize}    
    \item Planos \\
    Las herramientas que se han encontrado para la creación de los planos son básicamente dos, una de pago y otra opción de software libre. Estas son AutoCAD y QCad. Dado que poseo una licencia de AutoCAD y personalmente tengo experiencia en esta herramienta he decidido usar esta herramienta para la creación de los planos del teclado. Para así poder ubicar todos los elementos de forma sencilla y sus dimensiones.
    
    \item Diseño de \gls{PCB} \\
    Existen varias opciones disponibles en el mercado para el diseño de \gls{PCB}. Algunas de las más populares son Altium Designer, KiCad, Eagle. Todos ellos son bastante parecidos en cuanto a las funcionalidades que se van a usar para el proyecto. Como se ha mencionado con las herramientas para los planos, también tenemos de pago y de software libre. Realmente aquí se ha escogido la herramienta por su conocimiento previo gracias a materias del grado como Diseño de Circuitos Impresos, donde se trabajó con Eagle. Por lo que el teclado está realizado con la herramienta Eagle.
    
    \item Editor de Texto/Programación \\
    Aquí disponemos de muchas opciones, entre las más famosas tenemos VSCode, Vim, \gls{Arduino} Editor (Si se usa Arduino) o Sublime. Por conveniencia y uso a lo largo del tiempo, se va a usar VSCode, ya que se dispone del framework \gls{PlatformIO}, que es el que se va a usar a posteriori para programar nuestro controlador y diseñar el programa/firmware.
    
    \item Software 3D \\
    Aquí tenemos las opciones más sencillas de diseño 3D, FreeCAD y Fusion360. Como ambas permiten las mismas opciones, se ha elegido FreeCAD como software de edición 3D.
    
    \item \gls{CNC} Corte \\
    En la búsqueda de software para crear los archivos necesarios para hacer funcionar la gls{CNC} para así hacer la carcasa se han encontrado diversos programas. Aunque la elección es clara, FreeCAD, ya que tenemos los archivos 3D ya en la aplicación y no tenemos que hacer nada más.
    
\end{itemize}
\newpage
%\subsubsection{Pros y Contras}

\section{Elección de Hardware} \label{DiseñoHardware}
En esta sección se van a tratar las diferentes opciones para los materiales. Los componentes electrónicos, las diferentes partes del teclado y el porqué de estas.

\subsection{Layout} \label{DiseñoLayaout}
Lo primero que se necesitó es conocer las dimensiones de la distribución. Así que buscando entre varias páginas encontré una herramienta que se llama ''Keyboard-Layout-Editor'' \cite{Layout-Editor} esta directamente te dejaba personalizar la distribución. La única modificación que sufrió la distribución \gls{ISO} fue que bajé las teclas especiales de la sección Print-Screen a al bloque de Pause/Delete. Esto fue pensado a futuro para poder colocar la electrónica en un sitio accesible.
    
\begin{figure}[H] % Colocar las figuras exactamente aquí
    \centering
    \includegraphics[width=1\textwidth]{imagenes/Capitulos/Cap03/ISO105Layout.png}
    \caption{\gls{ISO} 105 sin modificar}
    \label{fig:ISO_layout}
\end{figure}

\begin{figure}[H] % Colocar las figuras exactamente aquí
    \centering
    \includegraphics[width=1\textwidth]{imagenes/Capitulos/Cap03/ModernWoodLayout.png}
    \caption{\gls{ISO} 105 Modificado para el teclado ModernWood}
    \label{fig:eISo_layout_modificado}
\end{figure}

Esta herramienta permite exportar el esquema del teclado en un .JSON que posteriormente será necesario para usarlo en la página de ''Plate \& Case Builder'' \cite{builder-swillkb}. Esta herramienta no da un archivo .DWG que es un formato de AutoCAD para planos. El plano que nos ofrece es la plancha del teclado. Aunque nuestro teclado no va a utilizar, esta nos servirá posteriormente para diseñar la \gls{PCB} y la carcasa, ya que nos dirá donde va cada interruptor de forma muy precisa.

\subsection{Componentes Electrónicos}
A lo largo de toda la investigación he ido encontrando muchas alternativas para realizar el teclado, diferentes sistemas, diferentes microcontroladores y diferentes tipos de circuitos.

\subsubsection{Microcontrolador}
Este es el cerebro del teclado, este va a ser el encargado de la comunicación, de registrar las teclas, de construir los paquetes de datos, de mostrar los datos correspondientes y de mantener toda la iluminación.

Por esto es la pieza sobre lo que gira todo el proyecto y la parte más importante, también hay que considerar la facilidad para programarlos y la cantidad de recursos que tiene, tanto de librerías como de documentación.
Cuanto más extendido sea el uso, más sencillo será lograr que funcione.

A lo largo del periodo de investigación y diseño se han encontrado varias alternativas. Estas tienen que cumplir que tenga un hardware dedicado a la conectividad \gls{USB} llamado ''\gls{USB} OTG'' ya que se ideó así en la sección \ref{DiseñoConectividadCable}. Los principales y más destacables son:

\begin{itemize}
    \item \textbf{Atmega32u4}: Es un microcontrolador de la familia AVR de Atmel. Es ampliamente utilizado en proyectos de electrónica, especialmente en entornos Arduino. Ofrece una buena cantidad de recursos y es fácil de programar.
    
    \item \textbf{STM32F103C8T6}: Es un microcontrolador de la familia STM32 de STMicroelectronics. Ofrece un buen equilibrio entre potencia de procesamiento y recursos. Es popular en proyectos de electrónica debido a su amplia disponibilidad y comunidad activa de usuarios.
    
    \item \textbf{nrf52840}: Es un microcontrolador de Nordic Semiconductor, diseñado específicamente para aplicaciones de conectividad inalámbrica de baja energía. Ofrece capacidades avanzadas de \gls{Bluetooth} y un bajo consumo de energía, lo que lo hace adecuado para dispositivos portátiles y \glsnocase{Wearable}s.
    
    \item \textbf{ESP32S3}: Es un microcontrolador de Espressif Systems, conocido por su conectividad \gls{WiFi} y \gls{Bluetooth} integrada. Ofrece una potencia de procesamiento considerable y una amplia gama de recursos.
\end{itemize}
Cada uno de estos microcontroladores ofrece pros y contras al proyecto.

\subsubsection{Atmega32u4}
\begin{table}[H]
\centering
\small
\begin{tabular}{|p{0.45\linewidth}|p{0.45\linewidth}|}
\hline
\textbf{Pros} &
\textbf{Contras} \\
\hline
\parbox[t]{\linewidth}{
\vspace{0.1cm}
- Ampliamente utilizado y documentado en proyectos de teclado y dispositivos de entrada. \bigskip \\
- Facilidad de programación con el ecosistema Arduino. \bigskip \\
- Integración de puertos \gls{USB} nativos, lo que facilita la comunicación. \bigskip \\
- Fácil de soldar \bigskip \\
- Librerías para \gls{PCB} \bigskip \\
- Muy barato \bigskip \\
- 5V \\
\vspace{0.1cm}
} &
\parbox[t]{\linewidth}{
\vspace{0.1cm}
- Requiere hardware adicional y configuración personalizada para habilitar la conectividad \gls{Bluetooth}. \bigskip \\
- Limitaciones en términos de potencia de procesamiento y recursos en comparación con otros microcontroladores más orientados a la conectividad.} \bigskip \\
\hline
\end{tabular}
\end{table}

\subsubsection{STM32F103C8T6}
\begin{table}[H]
\centering
\small
\begin{tabular}{|p{0.45\linewidth}|p{0.45\linewidth}|}
\hline
\textbf{Pros} &
\textbf{Contras} \\
\hline
\parbox[t]{\linewidth}{
\vspace{0.1cm}
- Soporte para una amplia variedad de librerías y documentación, incluidas las relacionadas con \gls{Bluetooth}. \bigskip \\
- Potente y versátil, lo que facilita la implementación de soluciones \gls{Bluetooth} personalizadas. \bigskip \\
- Ampliamente utilizado en aplicaciones de \gls{IoT} y sistemas embebidos, lo que puede ofrecer soluciones ya probadas para la conectividad \gls{Bluetooth}. \bigskip \\
- Librerias para \gls{PCB} \bigskip \\
- Facil de soldar \bigskip \\
- Barato
} &
\parbox[t]{\linewidth}{
\vspace{0.1cm}
- Requiere herramientas de desarrollo específicas y conocimientos más avanzados en programación. \bigskip \\
- Curva de aprendizaje más pronunciada debido a su complejidad en comparación con los microcontroladores basados en Arduino. \bigskip \\
- Posiblemente excesivamente difícil para aplicaciones simples de teclado. \bigskip \\
- Requiere hardware adicional y configuración personalizada para habilitar la conectividad \gls{Bluetooth}. \bigskip \\
- Requiere el uso de su framework específico  \bigskip \\
- 3.3V} \bigskip \\
\hline
\end{tabular}
\end{table}

\subsubsection{Nrf52840}
\begin{table}[H]
\centering
\small
\begin{tabular}{|p{0.45\linewidth}|p{0.45\linewidth}|}
\hline
\textbf{Pros} &
\textbf{Contras} \\
\hline
\parbox[t]{\linewidth}{
\vspace{0.1cm}
- Diseñado específicamente para aplicaciones de baja potencia y conectividad inalámbrica, lo que lo hace ideal para la implementación de \gls{Bluetooth} de baja energía. \bigskip \\
- Amplia gama de características de conectividad, incluido \gls{Bluetooth} de baja energía. \bigskip \\
- Buen soporte de documentación y comunidad de desarrollo. \bigskip \\
- Modo \glsnocase{Deep Sleep} para ahorrar energía
} &
\parbox[t]{\linewidth}{
\vspace{0.1cm}
- Sin librerias para \gls{PCB} Eagle \bigskip \\
- Muy dificil de soldar \bigskip \\
- Menos común en aplicaciones de teclado en comparación con otros microcontroladores. \bigskip \\
- Requiere más experiencia en el desarrollo de sistemas inalámbricos y \gls{Bluetooth}. \bigskip \\
- Requiere el uso de su framework específico. \bigskip \\
- Caro. \bigskip \\
- 3.3V} \bigskip \\
\hline
\end{tabular}
\end{table}

\subsubsection{ESP32S3}
\begin{table}[H]
\centering
\small
\begin{tabular}{|p{0.45\linewidth}|p{0.45\linewidth}|}
\hline
\textbf{Pros} &
\textbf{Contras} \\
\hline
\parbox[t]{\linewidth}{
\vspace{0.1cm}
- Capacidades de conectividad \gls{WiFi} y \gls{Bluetooth} integradas, lo que facilita la implementación de soluciones \gls{Bluetooth}. \bigskip \\
- Potente y versátil, con una amplia comunidad de desarrollo y soporte de documentación. \bigskip \\
- Ideal para proyectos que requieren conectividad inalámbrica y capacidades avanzadas de procesamiento. \bigskip \\
- Con todo tipo de librerías \bigskip \\
- Modo \glsnocase{Deep Sleep} para ahorrar energía \bigskip \\
- Muy barato \vspace{0.1cm}
} &
\parbox[t]{\linewidth}{
\vspace{0.1cm}
- Dificultad media para soldar. \bigskip \\
- Consumo alto de energía sin \glsnocase{Deep Sleep}. \bigskip \\
- 3.3V } \bigskip \\
\hline
\end{tabular}
\end{table}

La elección final se va a volver a hacer en cuanto a estética y funcionalidad. Como buscar nuevo hardware para la conectividad \glsnocase{Bluetooth} iba a ser otra tarea costosa en cuanto a tiempo y recursos, los microcontroladores que necesitasen un hardware específico para la conectividad han sido descartados.
Lo que nos deja solo con ESP32S3 y Nrf52840. Entre estos dos, dada la dificultad de aprendizaje, la dificultad muy alta para soldarlo a la \gls{PCB} y que hay menos documentación, nos vamos a quedar con el microcontrolador \textbf{ESP32S3}. Una vez que ya sabemos qué microcontrolador vamos a usar podemos atender a las diferentes secciones.

\subsubsection{Sistema de Batería} \label{InvestigacionSistemaBateria}
Como se ideó en la sección de diseño \ref{DiseñoBateriaAlimentacion} y dado que el hueco para albergar la batería es pequeño y bastante plano, ya que tiene que ir entre la placa \gls{PCB} y la carcasa no hay muchas opciones. Esta será una batería Li-ion (Ion Litio) y dado que el microcontrolador funciona a 3.3V tendremos que buscar una batería típica de 3.7V.

Ahora el problema subyacente es cargar la batería y al mismo tiempo poder usar el teclado sin que esta se dañe. Buscando alternativas y modos de realizar esta tarea se ha encontrado un módulo para desarrollo llamado TP4056 que contiene un sistema de protección de la batería y un sistema de carga y consumo en paralelo manteniendo la vida de la batería. Como se quiere aun así mantener la estética, no se empleara el módulo directamente sino que se buscara un esquemático del módulo y se implementará directamente sobre la \gls{PCB}. Se han encontrado varios archivos esquemáticos.

\begin{figure}[H]
    \centering
    \includegraphics[width=1\textwidth]{imagenes/Capitulos/Cap03/TP4056Schematic.png}
    \caption{Esquemático del modulo TP4056 \cite{TP4056SchematicInternet}}
    \label{fig:SchemeTP4056}
\end{figure}

\subsubsection{Sistema de Pantalla} \label{DiseñoPantalla}
La idea es que este teclado dispusiera de una pantalla para mostrar información básica del estado del teclado, configuración y un menú de usuario. La búsqueda de una pantalla tampoco ha supuesto un gran reto. Solo tenía que cumplir varios requisitos. Que cupiese en el área designada para la electrónica y que fuera compatible con el microcontrolador. Hay dos pantallas que cupiesen en el área designada y no fueran unan \gls{LCD} de poca resolución, estas son la \textbf{\gls{TFT} 0.96"} y la \textbf{\gls{OLED} 0.96"}. Entre estas dos se tenía que escoger una.

\subsubsection{\gls{TFT} 0.96"}
\begin{table}[H]
\centering
\small
\begin{tabular}{|p{0.45\linewidth}|p{0.45\linewidth}|}
\hline
\textbf{Pros} &
\textbf{Contras} \\
\hline
\parbox[t]{\linewidth}{
\vspace{0.1cm}
- Pantalla a color que permite una representación más detallada. \bigskip \\
- Fácil de programar y utilizar con librerías disponibles. \bigskip \\
- Ideal para proyectos que requieren interfaces de usuario simples. \bigskip \\
- Amplia disponibilidad y variedad de opciones en el mercado. \vspace{0.1cm}
} &
\parbox[t]{\linewidth}{
\vspace{0.1cm}
- Consumo moderado de energía, pero más alto que las pantallas \gls{OLED}. \bigskip \\
\vspace{0.1cm}
} \\
\hline
\end{tabular}
\end{table}

\subsubsection{OLED 0.96"}
\begin{table}[H]
\centering
\small
\begin{tabular}{|p{0.45\linewidth}|p{0.45\linewidth}|}
\hline
\textbf{Pros} &
\textbf{Contras} \\
\hline
\parbox[t]{\linewidth}{
\vspace{0.1cm}
- Pantalla que ofrece un contraste superior a un solo color. \bigskip \\
- Consumo de energía muy bajo, especialmente al mostrar contenido estático. \bigskip \\
- Ángulos de visión más amplios y mejor visibilidad en condiciones de luz ambiental variable. \bigskip \\
- Ideal para aplicaciones de bajo consumo. \bigskip \\
\vspace{0.1cm}
} &
\parbox[t]{\linewidth}{
\vspace{0.1cm}
- Solo un color. \vspace{0.1cm}
} \\
\hline
\end{tabular}
\end{table}

Se ha acabado escogiendo la pantalla \textbf{\gls{TFT} 0.96"} solo por la cantidad de colores que puede mostrar. Esta usa 8 pines de nuestro microcontrolador.

\subsubsection{Sistema de \glsnocase{Polling}}
Dado que los microcontroladores no tienen la cantidad de pines necesarios para conectar la matriz de pines de 16x6 en total son necesario 22 pines exclusivamente para todas las teclas. Recordando materias de electrónica, recordé la existencia de un dispositivo que permite multiplexar señales eléctricas. Dado que tenemos que ir recorriendo las secciones verticales una a una e ir leyendo las horizontales para saber qué tecla se ha pulsado, podemos multiplexar esas 16 líneas en tan solo 4 gracias al dispositivo \glsnocase{Multiplexor}.

El objetivo es encontrar un \glsnocase{Multiplexor} que quepa y que nos ofrezca la función de 4 a 16, que valga a 3.3V y que pueda ser soldado con facilidad. Tan solo con una búsqueda aparece el \textbf{CD74HC154M96} dado que directamente cumple todos los requisitos pedidos nos vamos a quedar con este.

Esto nos permitirá configurar el microcontrolador con 4 pines de salida para seleccionar la columna que queremos comprobar y leer los 6 pines configurados como entrada para saber qué tecla exacta está o no pulsada.

\begin{figure}[H]
    \centering
    \includegraphics[width=1\textwidth]{imagenes/Capitulos/Cap03/EjemploArrayTeclado.png}
    \caption{Ejemplo básico de matriz de teclado \cite{EjemploArrayTeclado}}
    \label{fig:EjemploArrayTeclado}
\end{figure}

\newpage
\subsubsection{Conexión \gls{USB}} \label{DiseñoConexiones}
Como estamos usando un microcontrolador que dispone de \gls{USB}-OTG no será necesario usar un elemento hardware externo que nos permita comunicarnos con el protocolo \gls{USB}. Este ya vendrá por defecto en el ESP32S3.

Eso si, necesitaremos mirar qué pines están conectados al módulo de \gls{USB} en el ESP32S3. Para la conexión al exterior usaremos simplemente un cable conectando la \gls{PCB} al conector de aviación XS-12 que nos asegure la robustez que se ideó en la sección \ref{RequisitosNoFuncionales}.

\begin{figure}[H]
    \centering
    \includegraphics[width=0.9\textwidth]{imagenes/Capitulos/Cap03/XS12.png}
    \caption{Imagen del conector XS-12}
    \label{fig:XS12}
\end{figure}

Este conector irá atornillado a la carcasa y luego se fabricara un cable de este conector a \gls{USB} tipo A. usando un cable reforzado de 4 hilos internos y dos conectores más. Un conector de seguridad GX16 de 4 pines para darle robustez y un acabado más estético y un cabezal \gls{USB} A chapado en oro y todo sellado con tubo \glsnocase{Termoretractil}.

\begin{figure}[H]
    \centering
    \includegraphics[width=0.9\textwidth]{imagenes/Capitulos/Cap03/GX16.png}
    \caption{Imagen del conector GX-16}
    \label{fig:GX16}
\end{figure}

\begin{figure}[H]
    \centering
    \includegraphics[width=1\textwidth]{imagenes/Capitulos/Cap03/USBA.png}
    \caption{Imagen del conector \gls{USB} tipo A}
    \label{fig:USBA}
\end{figure}


\subsubsection{Sistema de \gls{LED}S} \label{DiseñoLeds}
Se han elegido los \gls{LED}S programables que se usan en las tiras comunes, ya que permiten muchas combinaciones y son muy baratos y sencillos de programar, además de que solo usan un pin del microcontrolador. La elección más sencilla ha sido para \textbf{WS2812B-B/W}

\begin{figure}[H]
    \centering
    \includegraphics[width=0.8\textwidth]{imagenes/Capitulos/Cap03/WS2812B.png}
    \caption{Imagen del \gls{LED} WS2812B}
    \label{fig:WS2812B}
\end{figure}

\subsection{Materiales}
Desde un principio la elección clara era madera, ya que se quería conseguir una estética natural y tecnológica juntas. Las maderas serán a elección del usuario, para que pueda escoger tanto color, tipo de veta y sensación al tacto.
Por lo tanto esta subsección se tratara de gustos personales. Entre las elecciones para este proyecto se han considerado \textbf{Jatoba, Sucupira y Mansonia}, además se añadirán detalles en metacrilato o acrílico para proteger los componentes que son visibles, ya que el teclado esta diseñado para ser un teclado sin plate.

\subsection{Partes del teclado}
Durante la fase de diseño \ref{DiseñoDistribucion} se han hecho elecciones de como va a ser el teclado, y de qué cosas dispondrá este. En esta subsección se hará un breve resumen. El teclado se compondrá de una carcasa en forma de cajón donde la \gls{PCB} irá atornillada y no tendrá plate. Los interruptores se colocarán soldados sobre la \gls{PCB} con las keycaps. De forma adicional se colocarán unas placas de acrílico sobre los componentes electrónicos para protegerlos y que aun así queden vistos. La tornillería se compondrá de elementos como separadores y de tornillos M3 negros.

La electrónica se compondrá de un ESP32S3 un \glsnocase{Multiplexor} para el mapeo de las teclas, una pantalla \gls{TFT} de 0.96", una batería, un circuito específico para la batería (TP4056) y todo el cableado USB.
\newpage

\section{Elección de Software}
En esta sección se van a tratar las diferentes opciones para las elecciones que tenemos de herramientas para desarrollar el firmware para el ESP32S3 y qué librerías se van a usar para ello.

\subsection{Entorno de desarrollo}
Dado que vamos a trabajar con un ESP32S3 tenemos múltiples opciones, \textbf{Arduino IDE}, \textbf{PlatformIO} y \textbf{Espressif ESP-IDF}.

Estas alternativas están ordenadas de menor a mayor complejidad y funcionalidad.
\subsubsection{Arduino IDE}
Arduino IDE se compone un programa que permite la edición de un archivo .ino que posteriormente se compila y se sube al microcontrolador. Dado que es muy sencillo este entorno, está muy muy limitado para proyectos con varios archivos, librerías personalizadas y código C. Esta queda directamente descartada, ya que nuestro proyecto estará si o si dividido en varios archivos y será mucho más sencillo en las otras.

\subsubsection{PlatformIO}
PlatformIO es un entorno de desarrollo que reúne las herramientas necesarias para el desarrollo de proyectos de hardware, especialmente en microcontroladores como Atmega, ESP32, STM32, entre otros. Ofrece una solución más avanzada y versátil en comparación con el Arduino IDE.

Permite la creación y gestión de proyectos complejos que involucran múltiples archivos, bibliotecas personalizadas y configuraciones específicas para diferentes placas y microcontroladores.

Facilita la inclusión y gestión de bibliotecas externas a través de su sistema de gestión de librerías integrado. También ofrece herramientas avanzadas de compilación y depuración que facilitan el desarrollo y la corrección de errores en el código.

Admite una amplia gama de microcontroladores y placas, lo que permite a los desarrolladores trabajar con diferentes dispositivos sin restricciones. Entre ellas el microcontrolador ESP32S3.

Además, PlatformIO se integra con entornos de desarrollo integrado (IDE) como Visual Studio Code, Atom y Eclipse, lo que proporciona a los desarrolladores una experiencia de desarrollo más completa y personalizable.

\subsubsection{Espressif ESP-IDF}
El Espressif IoT Development Framework (ESP-IDF) es un conjunto de herramientas de desarrollo de software diseñado específicamente para trabajar con microcontroladores de la serie ESP32 de Espressif Systems. Este framework proporciona un entorno de desarrollo completo y robusto para la creación de aplicaciones y firmware personalizado para dispositivos basados en ESP32.

El ESP-IDF está diseñado para aprovechar al máximo las capacidades y características del ESP32, incluyendo su procesador de doble núcleo, conectividad \gls{WiFi} y \glsnocase{Bluetooth}, y una amplia gama de periféricos integrados.

Viene con una amplia gama de bibliotecas y ejemplos de código que facilitan el desarrollo de aplicaciones para una variedad de propósitos, como la comunicación inalámbrica, la gestión de energía y la interfaz con sensores y actuadores.

Espressif proporciona una documentación exhaustiva y actualizada que cubre todos los aspectos del desarrollo con ESP-IDF, incluyendo guías de inicio rápido, tutoriales y referencias de \gls{API}.

En resumen, el Espressif ESP-IDF es un framework de desarrollo de software potente y versátil que ofrece a los desarrolladores todas las herramientas necesarias para crear aplicaciones personalizadas para dispositivos basados en ESP32, aprovechando al máximo las capacidades de estos microcontroladores. Pero dado que este es mucho más amplio, también es más complejo de usar.

\subsubsection{Elección final}

Dado todo lo anterior y que nuestro proyecto no va a requerir un código muy complejo y un control de muy bajo nivel sobre el microcontrolador, vamos a elegir \textbf{PlatformIO} que nos permitirá tener un proyecto grande compuesto de múltiples archivos y nos dará la suficiente libertad sin tener que invertir muchos recursos aprendiendo a usar ESP-IDF. Aun asi instalaremos ESP-IDF para compilar el código, ya que el compilador específico genera un mejor código.
\newpage
\subsection{Librerías}
En el apartado Hardware se han listado varios componentes que van a necesitar comunicarse con el Microcontrolador ESP32S3, entre ellos tenemos los \gls{LED}S, la pantalla \gls{TFT} 0.96". Pero también sera necesario activar y poder usar el \glsnocase{Bluetooth} como elemento \gls{HID}. Por lo que haciendo el recuento tenemos que buscar estas 3 librerías que nos permitan hacer uso de estos dispositivos.
\begin{itemize}
    \item \gls{LED}S\\
    Para los \glsnocase{LED}s con una simple búsqueda podemos encontrar que Adafruit (El fabricante de la pantalla) ya pone a disposición una librería compatible con nuestro Framework. Esta librería es la \textbf{Adafruit NeoPixel}.
    \item Pantalla \gls{TFT} 0.96"\\
    Al igual que para los \glsnocase{LED}s disponemos de una librería que nos permite hacer todo tipo de cosas con la pantalla, mostrar gráficos, imágenes y texto de forma sencilla. Esta está realizada por un particular en este caso. La librería se llama \textbf{TFT\_eSPI} \cite{TFTLib}.
    \item Bluetooth \gls{HID}\\
    Aquí volvemos a encontrar que disponemos de una librería particular que nos resuelve el problema de mostrar nuestro teclado como un dispositivo \gls{HID} a otros dispositivos a través de \glsnocase{Bluetooth}, este también nos permite enviar caracteres al dispositivo conectado. Por lo que nos vendrá genial. La librería en este caso es la \textbf{NimBLE-Arduino} \cite{NimbleLib}.
\end{itemize}

\newpage
\section{Presupuesto}
Esta sección presenta un listado detallado de los costos asociados al desarrollo y fabricación del producto. El presupuesto abarca diferentes aspectos del proyecto, desde la adquisición de materiales hasta los gastos relacionados con el personal y la investigación y desarrollo. También se proporcionará una lista de todos los componentes con su coste asociado y el precio de adquisición así como el lugar de donde se ha adquirido y el coste de envío. 

\subsubsection{Componentes Electrónicos}

En la tabla \ref{Table:ComponetesElectronicos} se puede ver el coste de todos estos componentes, estos han sido elegidos en base al precio, funcionalidad y tipo. 

El total de la suma de los componentes electrónicos supone: \textbf{67,76 \euro}

\begin{table}[!htb]
\small
\begin{tabular}{|l|l|l|l|l|}
\hline
Tipo                                                              & Elemento                                                                 & Cantidad & Precio & Proveedores \\ \hline
\gls{Multiplexor}                                                       & \begin{tabular}[c]{@{}l@{}}CD74HC154M96\\  (SOIC-24-300mil)\end{tabular} & 1 (5)    & 3.4    & lcsc.com    \\ \hline
Condensador                                                       & \begin{tabular}[c]{@{}l@{}}293D105X9035A2TE3 1uF \\ (1206)\end{tabular}  & 5 (20)   & 2.06   & lcsc.com    \\ \hline
Resistencia                                                       & \begin{tabular}[c]{@{}l@{}}RT0805BRD071KL 1k \\ (0805)\end{tabular}      & 5 (20)   & 0.68   & lcsc.com    \\ \hline
Resistencia                                                       & \begin{tabular}[c]{@{}l@{}}RC0805FR-071ML 1M \\ (0805)\end{tabular}      & 3 (100)  & 0.21   & lcsc.com    \\ \hline
Resistencia                                                       & \begin{tabular}[c]{@{}l@{}}0805W8F2004T5E 2M \\ (0805)\end{tabular}      & 2 (100)  & 0.16   & lcsc.com    \\ \hline
Resistencia                                                       & \begin{tabular}[c]{@{}l@{}}RC0805FR-073M74L 3.7M \\ (0805)\end{tabular}  & 1 (50)   & 0.18   & lcsc.com    \\ \hline
\begin{tabular}[c]{@{}l@{}}Circuito \\ Carga Batería\end{tabular} & \begin{tabular}[c]{@{}l@{}}TP4056 \\ (C725790)\end{tabular}              & 1 (5)    & 0.52   & lcsc.com    \\ \hline
\begin{tabular}[c]{@{}l@{}}C. Protección\\  Batería\end{tabular}   & \begin{tabular}[c]{@{}l@{}}DW01A \\ (SOT23-6)\end{tabular}               & 1 (20)   & 0.41   & lcsc.com    \\ \hline
\begin{tabular}[c]{@{}l@{}}C. Protección \\ Batería\end{tabular}   & \begin{tabular}[c]{@{}l@{}}ME6211C33M5G \\ (SOT-23-5)\end{tabular}       & 1 (10)   & 0.48   & lcsc.com    \\ \hline
\begin{tabular}[c]{@{}l@{}}C. Protección \\ Batería\end{tabular}   & \begin{tabular}[c]{@{}l@{}}FS8205A \\ (TSSOP-8)\end{tabular}             & 1 (10)   & 0.62   & lcsc.com    \\ \hline
Chip Principal                                                    & \begin{tabular}[c]{@{}l@{}}ESP32-S3 \\ (SMD,18x25.5mm)\end{tabular}      & 1 (1)    & 4.46   & lcsc.com    \\ \hline
Leds                                                              & \begin{tabular}[c]{@{}l@{}}WS2812B \\ (C114586)\end{tabular}             & 10 (25)  & 2.18   & lcsc.com    \\ \hline
Diodos                                                            & \begin{tabular}[c]{@{}l@{}}1N5819W \\ (SOD123FL)\end{tabular}            & 96 (250) & 2.08   & lcsc.com    \\ \hline
\gls{PCB}                                                               & \begin{tabular}[c]{@{}l@{}}Placa Base \gls{DIY} \end{tabular}            & 1 (5) & 42.76     & jlcpcb.com  \\ \hline
Envíos                                                            & \begin{tabular}[c]{@{}l@{}}Envio Piezas \end{tabular}                    & 1 (1) & 7.56      & lcsc.com    \\ \hline
\end{tabular}
\caption{Tabla de componentes electrónicos}
\label{Table:ComponetesElectronicos}
\end{table}

\subsubsection{Componentes de Montaje}

En la tabla \ref{Table:ComponentesMontaje} se puede ver el coste de todos estas piezas y partes, estos han sido elegidos en base al precio, estética y funcionalidad. 

El total de la suma de los componentes para montaje supone:  \textbf{185,83 \euro}

\begin{table}[H]
\small
\begin{tabular}{|l|l|l|l|l|}
\hline
Tipo                                                              & Elemento                                                                                       & Cantidad                                                                           & Precio & Proveedores                                                     \\ \hline
\begin{tabular}[c]{@{}l@{}}Partes\\ Cable\end{tabular}            & \begin{tabular}[c]{@{}l@{}}Paracord negro\\ de 9 núcleos\end{tabular}                          & 1 (1)                                                                              & 3.16   & \begin{tabular}[c]{@{}l@{}}Aliexpress\\ Qiuhike\end{tabular}    \\ \hline
\begin{tabular}[c]{@{}l@{}}Partes\\ Cable\end{tabular}            & \begin{tabular}[c]{@{}l@{}}Cabeza USB A\\ Chapada en Oro\end{tabular}                          & 1 (5)                                                                              & 4.54   & \begin{tabular}[c]{@{}l@{}}Aliexpress\\ Lingxun\end{tabular}    \\ \hline
Tornillería                                                       & \begin{tabular}[c]{@{}l@{}}Espaciadores \\ hexagonales de latón\\ M3x3,M3x4, M3x5\end{tabular} & \begin{tabular}[c]{@{}l@{}}M3x3 25 (30)\\ M3x4 25 (30)\\ M3x5 25 (30)\end{tabular} & 10.77  & \begin{tabular}[c]{@{}l@{}}Aliexpress\\ HUJI\end{tabular}       \\ \hline
Tornillería                                                       & \begin{tabular}[c]{@{}l@{}}Tuerca hexagonal\\ para inserción muebles\end{tabular}              & 25 (30)                                                                            & 3.05   & \begin{tabular}[c]{@{}l@{}}Aliexpress\\ Electrician\end{tabular}     \\ \hline
Estética                                                          & \begin{tabular}[c]{@{}l@{}}Tubo de gel\\ de difusor led negro\end{tabular}                     & 1 (1)                                                                              & 6.92   & \begin{tabular}[c]{@{}l@{}}Aliexpress\\ ANTVLED\end{tabular}    \\ \hline
Electronica                                                       & \begin{tabular}[c]{@{}l@{}}Pantalla \gls{TFT} \gls{LCD}\\ 0.96" PCB Negra\end{tabular}                     & 1 (1)                                                                              & 5.06   & \begin{tabular}[c]{@{}l@{}}Aliexpress\\ All-goods\end{tabular}  \\ \hline
\begin{tabular}[c]{@{}l@{}}Conectores\\ Electronica\end{tabular}  & \begin{tabular}[c]{@{}l@{}}Micro conector JST \\ de 4 pines\end{tabular}                       & 2 (10)                                                                             & 5.48   & \begin{tabular}[c]{@{}l@{}}Aliexpress\\ JSAAHZ\end{tabular}     \\ \hline
\begin{tabular}[c]{@{}l@{}}Conectores\\ Electrónica\end{tabular}  & \begin{tabular}[c]{@{}l@{}}Conector XS12\\ Aviación de 12mm\end{tabular}                       & 1 (1)                                                                              & 2.24   & \begin{tabular}[c]{@{}l@{}}Aliexpress\\ MannHwa\end{tabular}    \\ \hline
Batería                                                           & \begin{tabular}[c]{@{}l@{}}Batería LiPo 3.7V\\ 706090 5000mha\end{tabular}                     & 1 (1)                                                                              & 12.13  & \begin{tabular}[c]{@{}l@{}}Aliexpress\\ EasyLander\end{tabular} \\ \hline
\begin{tabular}[c]{@{}l@{}}Partes\\ Cable\\ Conector\end{tabular} & Conector Gx16-4 Pin                                                                            & 1 (1)                                                                              & 1.35   & \begin{tabular}[c]{@{}l@{}}Aliexpress\\ Wire-conn\end{tabular}  \\ \hline
Estabilizadores                                                   & \begin{tabular}[c]{@{}l@{}}Everglide-tornillo PCB\\ Estabilizadores teclado\end{tabular}       & 1 Pack (1)                                                                         & 18.9   & \begin{tabular}[c]{@{}l@{}}Aliexpress\\ KRepublic\end{tabular}  \\ \hline
Interruptores                                                     & Gateron- 5 Pines                                                                               & 105 (110)                                                                          & 30.41  & \begin{tabular}[c]{@{}l@{}}Aliexpress\\ Lesozoh\end{tabular}    \\ \hline
\gls{Keycaps}                                                          & \begin{tabular}[c]{@{}l@{}}JakeTsai, doble capa\\ MX ISO ANSI, SA ABS\end{tabular}             & 1 Pack (1)                                                                         & 50.0   & \begin{tabular}[c]{@{}l@{}}Amazon\\ JakeTsai\end{tabular}       \\ \hline
Carcasa                                                           & \begin{tabular}[c]{@{}l@{}}Tabla de Madera elección\\ Sucupira/Jatoba/Mansonia\end{tabular}    & 1 (1)                                                                              & 31.82  & Contraveta                                                      \\ \hline
\end{tabular}
\caption{Tabla de componentes para montaje}
\label{Table:ComponentesMontaje}
\end{table}

\newpage

\subsubsection{Servicios y Herramientas}
Aparte de todo esto, que son las piezas necesarias para montarlo y que compondrán un teclado, serán necesario muchas más cosas. Entre ellas las herramientas necesarias para poder montarlo. También se tendrá en cuenta los servicios contratados, entre ellos el de \glsnocase{Fresado} para la carcasa y a su vez las horas dedicadas al proyecto. Estos servicios y herramientas se pueden ver en la tabla \ref{Table:ServiciosHerramientas}.

El total de la suma de los servicios y herramientas supone: \textbf{23.049,3 \euro}

\begin{table}[h]
\small
\begin{tabular}{|l|l|l|l|l|}
\hline
Tipo         & Elemento                                                             & Cantidad & Precio & Proveedores                                                          \\ \hline
Servicios    & \gls{Fresado} Carcasa                                                      & 1 (1)    & 51.43  & CNC GRANADA                                                                  \\ \hline
Servicios    & Costes Ingeniero                                                     & 1 (1)    & 22900   & CLM                                                                  \\ \hline
Herramientas & Soldador Estaño                                                      & 1 (1)    & 9.95   & \begin{tabular}[c]{@{}l@{}}Aliexpress\\ Preferred Store\end{tabular} \\ \hline
Herramientas & \begin{tabular}[c]{@{}l@{}}Soldador SMD\\ Aire caliente\end{tabular} & 1 (1)    & 48.95  & \begin{tabular}[c]{@{}l@{}}Amazon\\ Faokze\end{tabular}              \\ \hline
Herramientas & Bobina Estaño                                                        & 1 (1)    & 7.0    & \begin{tabular}[c]{@{}l@{}}Amazon\\ lumcov\end{tabular}              \\ \hline
Herramientas & Flux Soldadura                                                       & 1 (1)    & 7.99   & \begin{tabular}[c]{@{}l@{}}Amazon\\ Cerioll\end{tabular}             \\ \hline
Herramientas & \begin{tabular}[c]{@{}l@{}}Estaño Pasta\\ Soldadura SMD\end{tabular} & 1 (1)    & 11.99  & \begin{tabular}[c]{@{}l@{}}Amazon\\ Happy Finding\end{tabular}       \\ \hline
Herramientas & Kit Destornilladores                                                 & 1 (1)    & 11.99  & \begin{tabular}[c]{@{}l@{}}Aliexpress\\ WoodPow\end{tabular}         \\ \hline
\end{tabular}
\caption{Tabla de servicios Y herramientas}
\label{Table:ServiciosHerramientas}
\end{table}

\subsubsection{Total Presupuesto Unidad}

También es interesante saber el coste asociado a una unidad, sin precio de las horas de trabajo de diseño ni de montaje solo de materiales. Para poder hacer un estimado a la hora de venderlo como producto. El valor de un teclado en materiales con los envíos y todo ajustado a la cantidad de material que consume un teclado asciende a (25,36 \euro en la parte la electrónica), (80,21 \euro para la parte obligatoria del montaje + 12,13 \euro con batería), (80,41 \euro de la elección de \gls{Keycaps} e interruptores). Se debe tener en cuenta el servicio del fresado, ya que es obligatorio para el producto, por lo que son 51,43 \euro.

El total del producto en el caso de las elecciones hechas es de 198,11 \euro y de \textbf{249,54 \euro} con el servicio.

Como producto para la venta en internet de producto de lujo y \gls{DIY} tendríamos que añadir los costes asociados a las herramientas y al tiempo que se tarda en montarlo. Este total es de \textbf{409,4 \euro}.
\chapter{Circuitos}

Como ya se decidió en la sección \ref{Herramientas} se va a usar la herramienta de diseño Eagle.
Usaremos la versión gratuita, ya que las restricciones de esta son en tamaño máximo de la placa. Una vez que tenemos listados los elementos que necesitamos, ya que hemos hecho el diseño previo y sabemos que componentes van a estar presentes, podemos buscar librerías que nos traigan esos componentes para la herramienta Eagle.

\section{Búsqueda de componentes}

Vamos a buscar todos los componentes que van sobre la \gls{PCB}, estos son todos los de la tabla de componentes electrónicos \ref{Table:ComponetesElectronicos} y los interruptores de la tabla de Montaje \ref{Table:ComponentesMontaje}.

\subsubsection{\gls{Multiplexor}}

Para este componente se ha hecho una búsqueda en internet y la página donde se ha encontrado es Snapeda \cite{Snapeda} en la sección de partes podemos hacer una búsqueda y encontramos fácilmente \textbf{74HC154D,653} \cite{SnapedaMux}.

\subsubsection{Condensador}

Para este componente se hizo lo mismo, pero al ser un componente más genérico se puede buscar por el tipo de formato que tiene, ya que hay muchos tipos de formatos para componentes como resistencias y condensadores. En el caso del componente elegido es el formato o paquete 1206, por lo que podemos buscar realmente cualquier condensador con ese tipo de paquete y el valor se lo cambiamos en el programa. En la misma página mencionada anteriormente encontramos el condensador del formato que buscamos \cite{SnapedaCap}.

\subsubsection{Resistencia}

De la misma forma, se ha hecho con la resistencia, en este caso hay varias, pero se han escogido todas del mismo formato para simplificar el trabajo, el formato de la resistencia o paquete es el 0805. En la misma página que las dos anteriores podemos encontrar una resistencia genérica \cite{SnapedaRes}.

\subsubsection{TP4056}

Este si es un chip especifico, por lo que la búsqueda va a ser de este chip en concreto. En la misma página SnapEda \cite{Snapeda} encontramos el chip \textbf{TP4056} sin problemas \cite{SnapedaTP4056}.

\subsubsection{DW01A}

En la misma página SnapEda \cite{Snapeda} encontramos el chip \textbf{DW01A} sin problemas \cite{SnapedaDW01A}.

\subsubsection{ME6211C33}

Aunque en sí es un chip específico, tiene un formato muy común, lo que facilita la búsqueda, ya que se corresponde con cualquier regulador de tensión \gls{SMD}. En la misma página SnapEda \cite{Snapeda} encontramos el chip \textbf{LM3940IMP} con el mismo formato que nuestro \textbf{ME6211C33} \cite{SnapedaME6211C33}.

\subsubsection{FS8205A}

Este chip es otro específico por lo que la búsqueda va a ser de este chip en concreto. En la misma página SnapEda \cite{Snapeda} encontramos el chip \textbf{FS8205A} sin problemas \cite{SnapedaFS8205A}.

\subsubsection{ESp32S3} \label{ESp32S3BusquedaComponente}

Este chip es otro específico por lo que la búsqueda va a ser de este chip en concreto. En la misma página SnapEda \cite{Snapeda} encontramos el chip \textbf{ESp32S3} sin problemas, aunque este posteriormente lo vamos a modificar. \cite{SnapedaESp32S3}.

\subsubsection{\gls{LED}S}

En cuanto a los \glsnocase{LED}s podremos encontrar otros de forma genérica, igual que hicimos con las resistencias o condensadores. En la misma página que todo lo anterior podemos encontrarlo de nuevo \cite{SnapedaWS2812B}.

\subsubsection{Diodos}

Como vuelve a ser algo genérico un diodo de paquete SOD-123 podemos buscar cualquier diodo de este formato, aunque el nuestro sea el \textbf{1N5819W}. En la página que hemos usado anteriormente podemos encontrar fácilmente muchos diodos con ese formato \cite{Snapeda1N5819W}.

\subsubsection{Interruptores}

Para los interruptores vamos a prescindir de la página anterior, si no que vamos a buscar de nuevo en internet porque hay una buena comunidad detrás de todo este tipo de dispositivos y será mucho más fácil encontrar lo que necesitamos. En seguida encontramos un repositorio con la librería específica que necesitamos para nuestros interruptores \cite{GitInterruptores}.

\subsection{Creación de componentes}

Como se ha mencionado antes en la subsección de componentes \ref{ESp32S3BusquedaComponente}, vamos a modificar el componente ESP32S3.

La modificación hará que soldar este componente sea más sencillo, ya que facilitara la soldadura \gls{SMD} haciendo que la parte inferior con los pines de soldadura que van pegados a la placa sean visibles desde el otro lado de esta.

\begin{tcolorbox}[colback=blue!5!white, colframe=blue!55!white, title=Nota]
    Ver el apéndice \ref{ApendiceEsp32Hole} para más información sobre el diseño de este componente.
\end{tcolorbox}

\newpage
\section{Diseño Esquemático}
Para el diseño esquemático se han importado todos los componentes a Eagle y se han creado distintas partes del sistema para posteriormente interconectarlas todas en el diseño final.

\begin{tcolorbox}[colback=blue!5!white, colframe=blue!55!white, title=Nota]
    Ver el apéndice \ref{ApendiceDiseñoEsquematico} para más información sobre el diseño esquemático y las consideraciones tomadas. 
\end{tcolorbox}

\subsubsection{Matriz del Teclado}
En esta parte se ha seguido el ejemplo básico que podemos ver en la figura \ref{fig:EjemploArrayTeclado} en el apartado de diseño, donde se opta por una configuración de filas y columnas con su respectivo diodo para evitar \gls{Ghosting}. En la fase de diseño se decidió que iba a ser una matriz de 16*6 teclas únicas, una tecla especial individual y 13 teclas repetidas. La matriz finalmente queda como se muestra en la figura \ref{fig:MatrizTeclas}.

\begin{figure}[H]
    \centering
    \includegraphics[width=1.0\textwidth]{imagenes/Capitulos/Cap04/MatrizTeclas.png}
    \caption{Matriz final de teclas \cite{Repo:ImagenCircuito}}
    \label{fig:MatrizTeclas}
\end{figure}

\newpage
\subsubsection{\gls{Multiplexor}}
Esta parte se trata de las entradas y salidas del \glsnocase{Multiplexor} hacia la matriz y al microcontrolador. Esta parte es bastante sencilla como se puede apreciar en la figura \ref{fig:MuxImagenCircuito}.

\begin{figure}[H]
    \centering
    \includegraphics[width=0.6\textwidth]{imagenes/Capitulos/Cap04/Mux.png}
    \caption{\gls{Multiplexor} con todos los pines conectados a sus salidas/entradas \cite{Repo:ImagenCircuito}}
    \label{fig:MuxImagenCircuito}
\end{figure}

\newpage
\subsubsection{Controlador Principal}
Esta parte del circuito es el chip principal (ESP32S3) junto con los botones necesarios para programarla en el momento de subir el firmware, las resistencias que mantienen la placa encendida y la única tecla especial. Todo se puede apreciar en la figura \ref{fig:ESP32S3Circuito}

\begin{figure}[H]
    \centering
    \includegraphics[width=1.0\textwidth]{imagenes/Capitulos/Cap04/CHIP.png}
    \caption{Microcontrolador principal ESP32S3 y tecla principal \cite{Repo:ImagenCircuito}}
    \label{fig:ESP32S3Circuito}
\end{figure}

\newpage
\subsubsection{Bateria y alimentación}
Como este teclado tiene que soportar alimentación por batería a 4.2 V, por \gls{USB} a 5.0 V y a 3.3 V para el ESP32S3 es necesario un circuito que se encargue de estos voltajes, de proteger la batería y proteger todos los circuitos. Se puede ver el circuito encargado de esto en la figura \ref{fig:CircuitoBateriaAlimentacion}. Todo el diseño se ha hecho siguiendo el apartado \ref{InvestigacionSistemaBateria} y la figura \ref{fig:SchemeTP4056}.

\begin{figure}[H]
    \centering
    \includegraphics[width=1.0\textwidth]{imagenes/Capitulos/Cap04/Battery.png}
    \caption{Sistema de alimentación y protección del teclado \cite{Repo:ImagenCircuito}}
    \label{fig:CircuitoBateriaAlimentacion}
\end{figure}

\begin{tcolorbox}[colback=red!11!white, colframe=red!50!white, title=Errores]
    Ver apartado de errores \ref{CorrienteInversa} y \ref{VoltajeRegulador} donde se explica por qué se ha añadido un diodo en la puerta del regulador de tensión.
\end{tcolorbox}

\newpage
\subsubsection{\gls{LED}S}
En la fase de diseño, en el punto \ref{DiseñoLeds} se decidió que tuviera \glsnocase{LED}s, ya que eran fáciles de programar y daban mucho juego. El apartado eléctrico de los \glsnocase{LED}s se puede observar en la figura \ref{fig:CircuitoLeds}.

\begin{figure}[H]
    \centering
    \includegraphics[width=1.0\textwidth]{imagenes/Capitulos/Cap04/LEDs.png}
    \caption{Sistema de \glsnocase{LED}s del teclado \cite{Repo:ImagenCircuito}}
    \label{fig:CircuitoLeds}
\end{figure}

\subsubsection{Conectores para el teclado}
En el apartado de diseño \ref{DiseñoPantalla} aparecen los conectores que necesita la pantalla. Además, se ha añadido otro conector más para poder programar el chip cuando este sea soldado en la placa, por esta razón, como se observa en la figura \ref{fig:CircuitoConectores}, nuestro teclado se compone de 2 conectores básicos.

\begin{figure}[H]
    \centering
    \includegraphics[width=1.0\textwidth]{imagenes/Capitulos/Cap04/Conectores.png}
    \caption{Conector de programación y conector de la pantalla \gls{TFT} \cite{Repo:ImagenCircuito}}
    \label{fig:CircuitoConectores}
\end{figure}

\begin{sidewaysfigure}
\centering
\includegraphics[width=\textheight]{imagenes/Capitulos/Cap04/Esquematico.png}
\caption{Esquema completo del circuito del teclado. \cite{Repo:ImagenCircuito}}
\label{fig:CircuitoCompleto}
\end{sidewaysfigure}
\chapter{PCB}

\section{Diseño físico} \label{DiseñoFisico}

Como ya se decidió en la sección \ref{Herramientas} y como ya se ha usado para el diseño esquemático, esta nos permitirá trasladar los componentes a un modelo físico de la placa base. Pero antes de poder mover los componentes a la placa, deberíamos saber a qué lugar van y de que forma. Por lo que antes de ponernos con el diseño de la placa deberemos diseñar el esquema físico del mismo. Para este apartado usaremos la herramienta AutoCad que se especifica en la sección \ref{Herramientas}.

Será necesario el diseño de las siguientes partes, \gls{PCB}, Carcasa, Disposición de teclas, ubicación de los componentes, ubicación de los conectores.

Estos diseños puede ser creados un mismo archivo de AutoCAD con las 3 vistas. Alzado, Perfil izquierdo y Planta. Esta última nos servirá para el diseño de la \gls{PCB} y de la disposición de teclas, así como poder crear el archivo 3D para la carcasa. Con el resto de vista nos serviremos para saber si el tamaño de los cortes y disposiciones de los elementos del teclado encajan entre sí y hay suficiente espacio para albergarlos en el interior de la carcasa.

El orden del diseño va a ser el siguiente. Disposición de teclas, diseño de la \gls{PCB}, diseño de la carcasa. Una vez tenemos esto podemos crear la \gls{PCB} en el programa EAGLE. El resto de los planos serán de confirmación y para asegurarnos de que lo que estamos haciendo saldrá correctamente.

\begin{tcolorbox}[colback=blue!5!white, colframe=blue!55!white, title=Nota]
    Ver el apéndice \ref{ApendicePCB} para más información sobre el diseño los planos y las consideraciones tomadas. 
\end{tcolorbox}

\newpage
\subsection{Diseño de la distribución} \label{CreacionPlanoDistribucion}

En el capítulo de diseño \ref{CapDiseño} en la sección \ref{DiseñoLayaout} se encontró una página web ``Plate \& Case Builder`` \cite{builder-swillkb} que nos generaba un plano para AutoCad basándonos en la distribución de teclas creada en la página web ``Keyboard Layout Editor`` \cite{Layout-Editor}.

Por lo que vamos a proceder a crear el plano primero para saber la distancia relativa entre las teclas y sus posiciones entre sí. Una vez introducido en ``\glsnocase{Plate} Layout`` el texto que nos genera la página del editor de distribución le damos a generar archivo CAD. Este nos genera el plano que podemos ver en la figura \ref{fig:PlanoDistribucionLayout}.

\begin{figure}[H]
    \centering
    \includegraphics[width=1\textwidth]{imagenes/Capitulos/Cap05/PlanoDiseñoTeclas.png}
    \caption{Imagen del plano generado por la página web ``Plate \& Case Builder`` \cite{builder-swillkb}}
    \label{fig:PlanoDistribucionLayout}
\end{figure}

\subsection{Medidas Físicas} \label{MedidasFisicas}

A partir del plano generado en la sección \ref{CreacionPlanoDistribucion} empezaremos a crear el resto de planos en AutoCad.

El primer paso es crear las dimensiones de la \gls{PCB}. Que en este caso van a ser el mínimo necesario para poder encajar todas las teclas y un borde de seguridad para que las \glsnocase{Keycaps} no toquen el borde de la madera.

También para obtener el plano de la \gls{PCB} es necesario añadir los tornillos, ya que como se diseñó en la sección \ref{DiseñoPlatePCB} y se decidió que el montaje del teclado iba a ser sin \glsnocase{Plate} como en la figura \ref{fig:Montaje7}, la fuerza de las manos presionando las teclas va a ser transmitida directamente a la \gls{PCB}. Esta fuerza hará que se doble fácilmente por el material del que está fabricada, que es menos rígido que las \glsnocase{Plate} convencionales. Se van a necesitar varios tornillos distribuidos a lo largo de toda la \gls{PCB} para poder hacer un teclado robusto.

Una vez que se ha importado el plano generado de las posiciones de los interruptores se va a proceder a quitar algunos elementos innecesario de marcado y simplificar algunos elementos del plano generado.

Los tornillos van a ser de 3 mm o los llamados M3. Estos son de un tamaño adecuado para que puedan ser ubicados entre las juntas de las teclas y en los bordes. Por lo que se dispondrán circunferencias a lo largo del plano en purpura indicando las posiciones de los agujeros que además serán los tornillos ciegos de las carcasa en un futuro. Una vez hecho esto nos quedamos con 25 tornillos y en la disposición final que podemos ver en el plano de la \gls{PCB} en la figura \ref{fig:PlanoPCB}.

El teclado va a necesitar otro tipo de agujeros para sujetar un difusor de luz para los \glsnocase{LED}s. Por lo que Vamos a tener que idear la posición de los \glsnocase{LED}s y además añadir los agujeros específicos para sujetar el difusor. Ya que vamos a colocar los \glsnocase{LED}s vamos a aprovechar y también colocar los planos de los diferentes componentes. Vamos a añadir el \glsnocase{Multiplexor}, el ESP32-S3, y el conector XS-12 y la pantalla que se habían decidido colocar en la sección \ref{DiseñoHardware}. Los agujeros para los \glsnocase{LED}s estarán marcados en azul y los componentes en naranja.

Una vez añadidos al plano todos los componentes importantes, la pantalla y haber marcado todos los agujeros que necesita la \gls{PCB} nos queda, por fin, el plano completo de la \gls{PCB} con todas las medidas correctas y en su lugar correspondiente. El plano se puede ver en la figura \ref{fig:PlanoPCBConTodo}.

\begin{figure}[H]
    \centering
    \includegraphics[width=1\textwidth]{imagenes/Capitulos/Cap05/PlanoPCB.png}
    \caption{Imagen del plano de la \gls{PCB}}
    \label{fig:PlanoPCB}
\end{figure}

\begin{figure}[H]
    \centering
    \includegraphics[width=1\textwidth]{imagenes/Capitulos/Cap05/PlanoConPartes.png}
    \caption{Imagen del plano completo de la \gls{PCB}}
    \label{fig:PlanoPCBConTodo}
\end{figure}

\subsection{Creación de la placa en Eagle}

Para la creación de la placa o del archivo necesario para poder encargar la placa por internet, vamos a volver a la herramienta Eagle, dado que ya tenemos el diseño esquemático completo hecho. Gracias a los componentes que hemos buscado en internet podremos hacer la parte física de una forma sencilla. Estos componentes tienen la información de como es la pieza física, por lo que si nos vamos al plano podemos encontrar todas las piezas dispersadas como se ve en la figura \ref{fig:EaglePCBNueva}.

Para poder colocar estos componentes en su lugar adecuado, tendremos que importar el plano que hicimos en AutoCAD. Lo guardaremos como .DXF y en Eagle le daremos a importar dxf. Una vez importado el plano nos aparecerá una capa nueva llamada documentación como se puede ver en la figura \ref{fig:EaglePCBPlano}. En esta capa estará albergado el plano de la \gls{PCB}. Ahora la tarea pendiente es colocar los elementos y componentes a lo largo de todo el plano siguiendo el orden correspondiente.

Una vez colocados todos los elementos nos quedará listo para empezar a conectar todo con los llamados líneas aéreas que son, simplemente, líneas amarillas que nos indican a donde tiene que ir la vía de conexión. Esto lo podemos ver en la figura \ref{fig:EaglePCBPartesColocadas}. 

Además, en la misma figura \ref{fig:EaglePCBPartesColocadas} ya hemos añadido los agujeros a la placa en sus correspondientes lugares. Además de dimensionar la capa de \gls{PCB} que nos dará la dimensión de la misma. Los agujeros se han realizado con la herramienta de ``Drill`` que nos permite localizar un taladro en el lugar que lo pongamos del tamaño que le indiquemos. También hemos creado las líneas de la capa ``dimensión`` que nos indica como debe ser la \gls{PCB} estas han sido colocadas siguiendo el borde rojo de la figura \ref{fig:PlanoPCBConTodo} en el plano de Eagle.
\newpage

\begin{figure}[H]
    \centering
    \includegraphics[width=1\textwidth]{imagenes/Capitulos/Cap05/EaglePCBNueva.png}
    \caption{Imagen de la vista \gls{PCB} en Eagle.}
    \label{fig:EaglePCBNueva}
\end{figure}

\begin{figure}[H]
    \centering
    \includegraphics[width=1\textwidth]{imagenes/Capitulos/Cap05/EaglePCBPlano.png}
    \caption{Imagen del plano importado en Eagle.}
    \label{fig:EaglePCBPlano}
\end{figure}

\begin{figure}[H]
    \centering
    \includegraphics[width=1\textwidth]{imagenes/Capitulos/Cap05/EaglePCBPartesColocadas.png}
    \caption{Imagen de la \gls{PCB} con los componentes ubicados y los agujeros.}
    \label{fig:EaglePCBPartesColocadas}
\end{figure}
\newpage
\subsubsection{Consideraciones de potencia y diseño}
Durante el diseño ha sido fundamental prestar atención a los voltajes de funcionamiento de las diferentes partes del teclado, ya que hay varios niveles de voltaje en todo el proyecto. Pero estas consideraciones también tiene que estar presentes ahora. Ya que durante el enrutamiento que vamos a realizar a continuación. Las pistas/carriles o vías que creemos tienen que cumplir unas características para el correcto funcionamiento del dispositivo.

En primer lugar, se debe prestar especial atención al dimensionamiento de los carriles de alimentación y conexión entre los distintos componentes del teclado. Los carriles deben tener el tamaño adecuado para manejar la corriente requerida por los componentes sin causar caídas de tensión significativas que puedan afectar al rendimiento del teclado.

Además, la posición de los elementos de potencia y otros componentes críticos debe ser cuidadosamente considerada. Una disposición incorrecta puede resultar en interferencias electromagnéticas (EMI) o en problemas de disipación de calor, lo que podría afectar negativamente al funcionamiento del teclado y su durabilidad.

Para ello vamos a usar herramientas de cálculo de pistas o vías para saber qué características tienen que tener. Vamos a calcular que distancia mínima y que tamaño tienen que tener las pistas. Para ello vamos a usar dos páginas que tienen calculadoras específicas para realizar esta tarea.

El primer valor va a ser el tamaño de la vía, para ello vamos a usar la página ``4pcb`` \cite{4pcbCalculator} y para la distancia entre pistas ``protoexpress`` \cite{protoexpressCalculator}. 

Empecemos con el tamaño de la vía, los parámetros que nos pide son los siguientes.
\begin{itemize}
    \item \textbf{Intensidad}. Este valor para nosotros va a valer como máximo 0,5 Amperios, ya que es el máximo que el \gls{USB} nos da.
    \item \textbf{Grosor}. Este valor es un estándar, así que para nosotros es de 2 $\frac{oz}{ft^2}$.
    \item \textbf{Temperatura}, Incremento y ambiente. Nuestra temperatura ambiente será de 25ºC y el incremento le pondremos de otros 25.
    \item \textbf{Longitud de la pista}. La distancia. Aquí vamos a elegir la pista más larga en nuestro teclado que será de lado a lado de unos 50 cm.
\end{itemize}

El resultado que nos arroja esta calculadora es de un grosor de 0.0861 mm. Como podemos ver, el tamaño de la pista es sumamente pequeño, por lo que los valores por defecto de Eagle nos van a valer perfectamente y además mejorara la resistencia y la caída de voltaje.

Para la distancia de la pista los parámetros que nos piden son:
\begin{itemize}
    \item \textbf{Máximo voltaje de la pista}. Para nosotros el voltaje más alto que tenemos en todo el circuito es de 5V (Entrada del \gls{USB}).
\end{itemize}

Tras darle a calcular, el valor que obtenemos es de 0,0508 mm. Otra vez los valores por defecto del programa superan con creces este valor. Por lo que por ahora cumplimos con todos los requisitos.

Cabe mencionar, que dado que el microcontrolador ESP32S3 es el que posee la antena \gls{Bluetooth} integrada, no es necesario calcular nada relacionado con la antena. Ya que el fabricante del microcontrolador ya ha hecho este trabajo por nosotros.

\subsubsection{Enrutamiento}
%explicar que se ha hecho de forma automatica la seccion de las teclas y se ha revisado a mano. Y mencionar que la seccion de micro ha sido hecha a mano
Para la sección de las teclas se ha usado la herramienta de enrutamiento automático de Eagle. Esta herramienta nos permite conectar todas las pistas de forma rápida y eficiente. Una vez que se ha realizado el enrutamiento automático, se ha revisado manualmente para asegurarse de que todas las conexiones son correctas y que no hay errores en el diseño.

Para la sección del microcontrolador y los componentes críticos, se ha realizado el enrutamiento manualmente. Esto se debe a que estas secciones son más críticas y requieren una mayor atención para asegurarse de que todas las conexiones son correctas y que no hay curvas extrañas que puedan afectar al rendimiento del teclado.

Una vez conectado todo nos quedaría la \gls{PCB} terminada. La podemos ver en la figura \ref{fig:EaglePCBConectada}. En esta figura podemos ver que todas las pistas están conectadas y que todos los componentes están en su lugar correspondiente. Además, se han añadido las pistas de alimentación y las pistas de conexión entre los diferentes componentes.

%Mencionar el plano de tierra que hemos creado para mejorar el ruino de la placa. Y que siempre es recomendable tener un plano de tierra en la placa.
Además de conectar todos los componentes, se ha añadido un plano de tierra en la \gls{PCB}. Este plano de tierra ayuda a mejorar la disipación del calor y a reducir las interferencias electromagnéticas (EMI) en la placa. También ayuda a mejorar la integridad de la señal y a reducir el ruido en la placa. Por lo que siempre es recomendable tener un plano de tierra en la placa. Se ha creado un plano para cada una de las capas de la \gls{PCB}. Como nosotros tenemos 2 capas, hemos creado un plano de tierra en la capa superior y otro en la capa inferior.
Estos planos se pueden ver en la figura \ref{fig:EaglePCBPlanoTierra1} y en la figura \ref{fig:EaglePCBPlanoTierra2}, respectivamente.

\begin{figure}[H]
    \centering
    \includegraphics[width=1\textwidth]{imagenes/Capitulos/Cap05/EaglePCBConectada.png}
    \caption{Imagen de la \gls{PCB} con las pistas conectadas.}
    \label{fig:EaglePCBConectada}
\end{figure}

\begin{figure}[H]
    \centering
    \includegraphics[width=1\textwidth]{imagenes/Capitulos/Cap05/EaglePCBPlanoTierra1.png}
    \caption{Imagen del plano de tierra superior de la \gls{PCB}.}
    \label{fig:EaglePCBPlanoTierra1}
\end{figure}

\begin{figure}[H]
    \centering
    \includegraphics[width=1\textwidth]{imagenes/Capitulos/Cap05/EaglePCBPlanoTierra2.png}
    \caption{Imagen del plano de tierra inferior de la \gls{PCB}.}
    \label{fig:EaglePCBPlanoTierra2}
\end{figure}

\subsubsection{Estética}
%Quitar las marcas de las Vias entre otras, poner el texto a mano, indicadores, POsicionar todo correctamente. Mover las vias para que queden más estetias etc.
Una vez que se ha realizado el enrutamiento y se han añadido los planos de tierra, se ha procedido a mejorar la estética de la \gls{PCB}. Para ello, se han eliminado las marcas de las vías y se han movido las vías para que queden más estéticas. También se ha añadido el texto a mano y se han posicionado todos los elementos de forma correcta. Además, se han añadido indicadores para facilitar la lectura de la \gls{PCB} y se han añadido los números de referencia de los componentes para facilitar la identificación de los mismos.

A su vez, también se han creado unos iconos para poder identificar los tipos de agujeros a los que están asociados. En total hay 4 iconos. Estos son para identificar el difusor de luz, cuáles tienen un panel de metacrilato para poder proteger la sección de componentes electrónicos, otro para saber cuál es necesario emplear tuerca y el último para saber cuál es necesario atornillar a la carcasa. Respectivamente, estos iconos podemos previsualizarlos en las figuras \ref{fig:difusionIcono}, \ref{fig:MetacrilatoIcono}, \ref{fig:TuercaIcono} y \ref{fig:TornilloIcono} respectivamente.

Más adelante se usarán para saber de forma rápida que tipo de agujero es necesario en la carcasa. Y para facilitar el montaje del teclado.

\begin{itemize}
    \item \textbf{Difusor de luz}. 
    \begin{figure}[H]
        \centering
        \includegraphics[width=0.15\textwidth]{imagenes/Capitulos/Cap05/Difusion.png}
        \caption{Imagen del icono de difusor de luz.}
        \label{fig:difusionIcono}
    \end{figure}
    \newpage
    \item \textbf{Panel de metacrilato}. 
    \begin{figure}[H]
        \centering
        \includegraphics[width=0.15\textwidth]{imagenes/Capitulos/Cap05/Glass.png}
        \caption{Imagen del icono del indicador de panel de metacrilato.}
        \label{fig:MetacrilatoIcono}
    \end{figure}
    \item \textbf{Tuerca}. 
    \begin{figure}[H]
        \centering
        \includegraphics[width=0.15\textwidth]{imagenes/Capitulos/Cap05/Nut.png}
        \caption{Imagen del icono del indicador de tuerca}
        \label{fig:TuercaIcono}
    \end{figure}
    \item \textbf{Atornillar a la carcasa}. 
    \begin{figure}[H]
        \centering
        \includegraphics[width=0.1\textwidth]{imagenes/Capitulos/Cap05/Screw.png}
        \caption{Imagen del icono del indicador tornillo a la carcasa.}
        \label{fig:TornilloIcono}
    \end{figure}
\end{itemize}
\chapter{Carcasa}

\section{Diseño físico}
\subsection{Medidas Físicas}
\subsection{Ergonomía}
\section{Fabricación}
\chapter{Programación}

El software de la placa ha sido desarrollado durante el transcurso de este proyecto. Se ha usado el lenguaje de programación C y el entorno de desarrollo \gls{PlatformIO} en Visual Studio Code. A continuación se detallarán los aspectos más importantes de la programación de la placa, diseños de las funciones y estructura del código. Estructura general de la máquina de estados y las funciones de cada uno de los estados.

También se detallará la estructura de la memoria flash de la placa, donde se guardan los datos de configuración y los datos de la batería.

Se explicará porque se han tomado ciertas decisiones de diseño del microcontrolador y como se han implementado.

\begin{tcolorbox}[colback=blue!5!white, colframe=blue!55!white, title=Nota]
    Durante la mayor parte del desarrollo se ha utilizado una placa comercial para ir programando, ya que no se tenía acceso al producto final. Ver el apéndice \ref{ApendicePruebas} para saber cómo se han realizado las pruebas del software.  
\end{tcolorbox}

\section{Plataforma}
Como se decidió en el capítulo de diseño \ref{CapDiseño} en la sección \ref{DiseñoPlatformIO}. Se ha usado \gls{PlatformIO} como entorno de desarrollo. \gls{PlatformIO} es un entorno de desarrollo que permite programar en varios lenguajes de programación y para varios microcontroladores. En este caso se ha usado para programar en C para el microcontrolador ESP32.

Para poder usar \gls{PlatformIO} vamos a necesitar instalar Visual Studio Code. Para instalar Visual Studio Code basta con ir a la página web de Visual Studio Code y descargar el instalador. Para instalar \gls{PlatformIO} basta con instalar la extensión de \gls{PlatformIO} en Visual Studio Code. Para ello basta con ir a la pestaña de extensiones en Visual Studio Code y buscar \gls{PlatformIO}. Una vez instalada la extensión de \gls{PlatformIO} ya se puede usar \gls{PlatformIO} en Visual Studio Code.

En \gls{PlatformIO} vamos a comenzar configurando el proyecto. Para que automáticamente se configure el proyecto con las librerías necesarias y el microcontrolador correcto. Para ello vamos a crear un nuevo proyecto en \gls{PlatformIO} y seleccionamos el microcontrolador ESP32.

Después vamos a crear el archivo platformio.ini en la raíz del proyecto. En este archivo vamos a configurar el microcontrolador y las librerías necesarias para el proyecto. En este caso vamos a usar las librerías de Adafruit NeoPixel, NimBLE-Arduino y TFT\_eSPI. Estas librerías se pueden instalar desde el gestor de librerías de \gls{PlatformIO}. Para ello basta con añadir las librerías al archivo platformio.ini. En este caso se han añadido las librerías en el archivo platformio.ini como se muestra en el código \ref{code:ConfiguracionPlaftformIO}.

\begin{itemize}
\item Adafruit NeoPixel
\item NimBLE-Arduino
\item TFT\_eSPI
\end{itemize}
\label{LibreriasPlatformIO}

Hay algunas recomendaciones en el uso de \gls{PlatformIO}. Para la ESP32S3, en concreto para la placa de desarrollo ESP32-S3-DevKitC-1. Se recomienda usar un filtro de monitor serié para poder ver los mensajes que envía la placa sin información adicional y que no sea relevante. Para ello se puede usar el siguiente filtro:

\begin{lstlisting}[style=console, language=bash, caption={Filtro recomendado de \gls{PlatformIO}}, label={code:FiltroEspecial}]
monitor_filters = esp32_exception_decoder
\end{lstlisting}
\newpage
El archivo de configuración platformio.ini será el descrito en el código \ref{code:ConfiguracionPlaftformIO}, donde se configura el microcontrolador, las librerías y algunas opciones de compilación y entorno. Con este fragmento el proyecto se configurará automáticamente con las librerías necesarias y el microcontrolador correcto. Dejando el proyecto listo para programar. Las librerías necesarias son las descritas en la sección \ref{DiseñoLibrerias}.

\begin{lstlisting}[style=console, language=bash, caption={Configuracion \gls{PlatformIO}}, label={code:ConfiguracionPlaftformIO}]
    [env:esp32-s3-devkitc-1]
    platform = espressif32
    board = esp32-s3-devkitc-1
    framework = arduino
    board_build.f_flash = 80000000L
    board_build.flash_mode = qio
    board_build.flash_size = 16MB
    board_build.usb_cdc = 1
    upload_speed = 921600
    monitor_speed = 115200
    board_name = ModernWood
    board_upload.vid = 0x2001
    board_upload.pid = 0x1111
    lib_deps = 
        adafruit/Adafruit NeoPixel@^1.11.0
        h2zero/NimBLE-Arduino@^1.4.1
        bodmer/TFT_eSPI@^2.5.30
    build_flags =
        -I modules/include
        -I include
    build_src_filter = +<*> +<../modules/src/*>
    monitor_filters = esp32_exception_decoder
    check_skip_packages = yes
\end{lstlisting}

\begin{tcolorbox}[colback=blue!5!white, colframe=blue!55!white, title=Nota]
    Dado que para nuestro dispositivo \gls{HID} se quiere que sea algo que aparezca en el sistema como un teclado, se ha añadido a la configuración del proyecto las opciones de ``board\_name`` y los correspondientes nuevos \gls{PID} y \gls{VID}. Estos valores son necesarios para que el sistema operativo reconozca el dispositivo como un dispositivo nuevo y no como una placa de desarrollo. Más información en el apendice \ref{ApendicePIDVID}.
\end{tcolorbox}

\subsection{Compilación y entorno}

Para compilar el proyecto se puede usar el comando \textit{pio run} en la terminal de \gls{PlatformIO}. Este comando compilará el proyecto y generará el archivo binario que se puede subir a la placa. Para subir el archivo binario a la placa se puede usar el comando \textit{pio run -t upload}. Este comando subirá el archivo binario a la placa y lo ejecutará. Para ver la salida de la placa se puede usar el comando \textit{pio device monitor}. Este comando abrirá una terminal serié donde se podrá ver la salida de la placa.

También se pueden usar los botones que nos aparecen en Visual Studio Code para compilar, subir y ver la salida de la placa. Estos botones se encuentran en la parte inferior izquierda de la pantalla de Visual Studio Code. \gls{PlatformIO} por defecto detecta automáticamente el puerto serie donde está conectada la placa y la velocidad de comunicación. Si se quiere cambiar la velocidad de comunicación se puede hacer en el archivo platformio.ini en la sección de configuración del proyecto que se muestra en el código \ref{code:ConfiguracionPlaftformIO}.


\section{Interfaz}

Para la interfaz de usuario se ha usado la librería TFT\_eSPI. Esta librería es una librería de Adafruit que permite controlar pantallas \gls{TFT}. Esta librería es muy completa y permite controlar pantallas \gls{TFT} de varios tamaños y resoluciones. En este caso se ha usado una pantalla \gls{TFT} de 0.96 pulgadas y 80x160 píxeles.

Con este software se ha creado una interfaz de usuario muy sencilla. La interfaz de usuario se compone de un menú principal con 6 opciones y dentro de cada submenú hay varias opciones. La interfaz de usuario se controla con los botones del teclado. Los botones elegidos han sido Fn, Escape, las flechas de dirección y la tecla Enter. Con estos botones se puede navegar por el menú y seleccionar las opciones deseadas. Se han creado unas imágenes para cada tipo de menú y su correspondiente categoría. Estas imágenes se han guardado en la memoria flash de la placa para poder ser leídas por la pantalla \gls{TFT}. Podemos ver el menú principal en la figura \ref{fig:MenuPrincipal}.

\begin{figure}[H]
\centering
\includegraphics[width=1\textwidth]{imagenes/Capitulos/Cap07/PantallaMenuPrincipal.png}
\caption{Menú principal en la pantalla \gls{TFT} LCD 0.96``}
\label{fig:MenuPrincipal}
\end{figure}

\subsection{Menú}

El menú principal se compone de 6 opciones. Cada opción tiene un icono que corresponde a la categoría de la opción. Además, como se puede ver en la figura \ref{fig:MenuPrincipal} cada menú va a tener asociado un número. Este número es el número de la opción en el menú. Este número se va a usar para seleccionar la opción deseada. Para seleccionar una opción se va a usar la tecla Enter. Para moverse por el menú se van a usar las flechas de dirección. Para salir de una opción se va a usar la tecla Escape. Para volver al menú principal se va a usar la tecla ESC. Si se quiere entrar y salir del modo de configuración se va a usar la tecla Fn.

Los diferentes menús vienen señalados en la imagen por un número. Con este vamos a identificar al menú para poder explicar su funcionamiento. Los menús son los siguientes:
\begin{itemize}
\item 1. Ajustes
\item 2. Brillo
\item 3. Leds
\item 4. Batería y Energía
\item 5. Conexión
\item 6. Extra y Ayuda
\end{itemize}

\subsubsection{Ajustes}
Este menú se compone de varias opciones. En este menú se pueden ajustar los parámetros del funcionamiento del teclado. En este menú se pueden ajustar los parámetros de, activar y desactivar la pantalla, activar y desactivar el teclado, ajustar el tiempo de debounce del teclado y ajustar el idioma de la interfaz.

\subsubsection{Brillo}
Este menú se compone de los parámetros de brillo de la pantalla y de los \glsnocase{LED}s. En este submenú se pueden encontrar dos ajustes numéricos del 0 al 100. Estos ajustes son el brillo de la pantalla y el brillo de los \glsnocase{LED}s. También se pueden llegar a apagar los \glsnocase{LED}s y la pantalla mediante este submenú, ya que podemos poner el brillo a 0.

\subsubsection{\gls{LED}s}
Este menú se compone de los parámetros de los \glsnocase{LED}s. En este submenú se pueden encontrar los ajustes de activar o desactivar los \glsnocase{LED}s, el color de los \glsnocase{LED}s y el modo de los \glsnocase{LED}s y la velocidad de los \glsnocase{LED}s.

\subsubsection{Batería y Energía}
Este menú se compone de los parámetros de la batería y la energía. En este submenú se pueden encontrar los ajustes de la batería. Para este submenú encontramos activar o desactivar la batería y activar o desactivar el modo de ahorro de energía.

\subsubsection{Conexión}
Este menú se compone de los parámetros de la conexión. En este submenú se pueden encontrar los ajustes de la conexión. Podemos seleccionar activar el modo de \gls{Bluetooth} por preferencia, activar o desactivar el \gls{Bluetooth} y seleccionar como preferente el modo \gls{USB} sobre el modo \gls{Bluetooth}.

\subsubsection{Extra y Ayuda}
Este menú se compone de los parámetros de la ayuda y extras. En este submenú encontramos el restaurar la configuración de fábrica, el modo especial para las funciones programas por el usuario y dos opciones de información y ayuda.

\subsection{Barra de estado}

La barra de estado se compone de varios elementos. En la parte superior de la pantalla se puede ver el icono de la conexión (A), el icono del modo de funcionamiento del teclado (B), el porcentaje de batería restante (C) y el icono de la batería (D). Estos elementos se pueden ver en la figura \ref{fig:MenuPrincipal}.

\begin{itemize}
    \item \textbf{A}. Icono de la conexión: Este icono indica el estado de la conexión. Este puede alternar entre el icono de \gls{Bluetooth} y el icono de \gls{USB}.
    \item \textbf{B}. Icono del modo de funcionamiento del teclado: Este icono indica el estado del teclado. Este puede alternar entre el icono de teclado activado y el icono de configuración.
    \item \textbf{C}. Porcentaje de batería restante: Este porcentaje indica el porcentaje de batería restante que se actualiza cada 1 minuto.
    \item \textbf{D}. Icono de la batería: Este icono indica el estado de la batería. Este indica si la batería está cargando o si la batería está descargada. También indica la carga y posee colores para indicar el estado de la batería.
\end{itemize}

\section{Funcionalidad}

La funcionalidad de la placa se ha dividido en varios apartados. Cada apartado se ha dividido en varios estados. Cada estado se compone de varias funciones que se ejecutan en función del estado en el que se encuentre la placa. Para cambiar de un estado a otro se han usado variables de estado a lo largo del código. Estas variables nos indican en que modo estamos, en que menú estamos, en que opción estamos, si hemos cambiado alguna opción, si hemos pulsado algún botón, etc.

Lo que nos permite desde el bucle principal saber en qué estado nos encontramos de la ejecución del programa y que funciones debemos ejecutar en cada momento. A continuación se detallarán los diferentes estados de la placa y las funciones que se ejecutan en cada uno de los estados.

\subsection{Estados}
\subsubsection{Ahorro de energía}
El primer apartado es el ahorro de energía. En esta sección de código se comprueba si el flag de tiempo de ahorro de energía ha sido activado por interrupción hardware de un reloj. Si el flag está activado, se activa el modo de ahorro de energía y se desactiva la pantalla y los \glsnocase{LED}s. Si el flag está desactivado y el flag de que estamos durmiendo está activado, se despierta la pantalla y los \glsnocase{LED}s.

\subsubsection{Batería}
Aquí comprobamos si desde el gestor de interrupciones de la batería ha actualizado el valor de la misma. Si es valor ha variado, se actualiza el valor de la batería en la pantalla y se actualiza el icono de la batería.

\subsubsection{\gls{LED}s}
En esta sección se comprueba si los \glsnocase{LED}s están activados en la configuración y si el modo de los \glsnocase{LED}s es el modo de \glsnocase{LED}s programados por el usuario. Si es así se ejecuta el modo de \glsnocase{LED}s programados por el usuario. Si no se ejecuta el modo de \glsnocase{LED}s por defecto.

\subsubsection{Actualización de conexión}
En esta sección se comprueba si la conexión ha cambiado. Si ha cambiado, se actualiza el icono de la conexión y se actualiza el modo de conexión.

\subsubsection{Pantalla}
En esta sección se comprueba si la pantalla está activada. En el caso de estar activada, se comprueba que se ha cambiado el estado de la pantalla. En el caso afirmativo se actualiza la pantalla con la información necesaria. Además, de mostrar siempre la pantalla con el brillo seleccionado.

\subsubsection{Teclado}
Se comprueba primero que no estamos en el modo de teclas especiales o modo especial, ya que tendríamos que ejecutar la función especificada especial.
Si no estamos en el modo especial se procede a funcionar en modo normal del teclado, mientras se comprueba que el flag de la interrupción de la tecla Fn no este activado. Si está activado, se cambia el modo de teclado a configuración y se activa el flag de configuración.
En el modo configuración nos encontramos con los mismos, se comprueba que el flag de la interrupción de la tecla Fn no este activado. Si está activado se cambia el modo de teclado a normal y se desactiva el flag de configuración.

En el modo de configuración, cada vez que se pulsa una tecla se actualiza el valor de la tecla pulsada y se activa el flag de tecla pulsada, activando la función de actualizar la pantalla.

Estos estados se pueden ver en la figura \ref{fig:EstadosKeyboard} junto con la figura del diagrama de flujo del bucle principal \ref{fig:MainLoop}.

\begin{figure}[H]
    \centering
    \includegraphics[width=1\textwidth]{imagenes/Capitulos/Cap07/MainLoop.png}
    \caption{Diagrama de Flujo del bucle principal}
    \label{fig:MainLoop}
\end{figure}

\begin{figure}[H]
    \centering
    \includegraphics[width=1\textwidth]{imagenes/Capitulos/Cap07/EstadosKeyboard.png}
    \caption{Diagrama de Flujo de la secuencia de cambio de modo de teclado}
    \label{fig:EstadosKeyboard}
\end{figure}

\subsection{Conectividad}

Para la conectividad se ha decidido que sea independiente el modo \gls{Bluetooth} y el modo \gls{USB}. Para ello se ha creado un menú de configuración donde se puede seleccionar el modo de conexión preferente. Si se selecciona el modo \gls{USB} preferente, la placa se conectará por \gls{USB} siempre que esté conectado. Si se selecciona el modo \gls{Bluetooth} preferente, la placa se conectará por \gls{Bluetooth} siempre que esté conectado. Si no se selecciona ninguno de los dos modos, el teclado permanecerá desconectado. Se puede conectar por \gls{USB} y \gls{Bluetooth} a la vez, pero en tal caso tendrá preferencia el modo \gls{USB}.

\subsection{Leds}

Para los \glsnocase{LED}s se han creado varios modos de funcionamiento. Se ha creado el modo estático, donde el color seleccionado se mantiene fijo. El modo de color blanco y el modo arcoíris. Se podrán añadir más modos creando la función correspondiente y añadiéndola al menú de \glsnocase{LED}s. Revisar el apéndice \ref{ApendiceLeds} para más información.

\subsection{Macros}

Para los macros o funciones especiales, se tendrá que activar el modo desde el menú "Extra y ayuda". Ya que estas funciones serán creadas por el usuario y se tendran que añadir al código. Se podrán añadir más funciones creando la función correspondiente y añadiéndola al menú de macros. Revisar el apéndice \ref{ApendiceMacros} para más información.

\section{Boot}

Para poder cargar el código al microcontrolador se ha usado el conversor \gls{TTL} que se indicó en la sección \ref{DiseñoActualizaciones} se ha usado un conversor de nivel lógico para poder programar el chip mientras este se alimenta con el \gls{USB} 5.0 V y no con la batería.
El código una vez cargado en el microcontrolador se ejecutará en cuanto el chip revisa el voltaje necesario para funcionar. En este caso el chip se enciende en cuanto se conecta el \gls{USB} y se enciende la pantalla mostrando el logotipo del proyecto. Acto seguido se encienden los leds y se muestra el menú principal en el caso de estar activa la pantalla.

\begin{tcolorbox}[colback=red!11!white, colframe=red!50!white, title=Errores]
    Ver apartado de errores \ref{CorrienteInversa} y \ref{VoltajeRegulador} donde se explica por qué es necesario el conversor de nivel lógico.
\end{tcolorbox}
\chapter{Prototipos}

\section{Montaje}
\section{Versiones}
\subsection{V1: Correcciones}
\subsection{V2: Correcciones}
\subsection{V3: Modificaciones}
\chapter{Validación}

En todos los proyectos de desarrollo de software y hardware es necesario realizar pruebas para comprobar que el sistema funciona correctamente. Aquí se destacarán las pruebas realizadas en el sistema para comprobar su correcto funcionamiento.

\section{Pruebas Eléctricas}
Para realizar todos los test y pruebas se ha hecho una plantilla tipica para rellenar, donde se le atribuye un nombre a la prueba, una descripción de la misma, como se realiza y el resultado obtenido. Esta se puede ver en la tabla \ref{Table:PruebasElectricas}.

\subsubsection{Prueba de Continuidad}

La prueba de continuidad se llevó a cabo utilizando un multímetro digital. Se verificó la continuidad en los circuitos de las teclas, los \gls{LED} indicadores y los cables de conexión. No se detectaron interrupciones en la continuidad, lo que indica una correcta conexión de los componentes.

\subsubsection{Prueba de Resistencia}

Se utilizaron instrumentos de medición adecuados para medir la resistencia eléctrica en diferentes puntos del teclado. Los valores de resistencia obtenidos se compararon con los rangos especificados en el diseño. Todos los componentes mostraron valores de resistencia dentro de los límites aceptables.

\subsubsection{Prueba de Corriente y Voltaje}

Se midió la corriente y el voltaje en varios puntos del circuito utilizando un amperímetro y un voltímetro. Los valores de corriente y voltaje se compararon con las especificaciones del diseño. Se observó un comportamiento adecuado de los circuitos, con corrientes y voltajes dentro de los rangos esperados.

\subsubsection{Prueba de Funcionamiento de los \gls{LED}}

Se realizó una prueba específica para verificar el funcionamiento de los \gls{LED} indicadores del teclado. Se encendieron y apagaron los \gls{LED} para confirmar que emitían luz de manera adecuada y que no presentaban fallos de conexión o funcionamiento.

\subsubsection{Prueba de Comunicación USB}

Se verificó la comunicación entre el teclado y la computadora a través del puerto \gls{USB}. Se enviaron datos desde el teclado a la computadora para confirmar que la comunicación era estable y que no había pérdida de información.

\subsubsection{Prueba de Interferencias Electromagnéticas}

El teclado fue expuesto a fuentes de interferencias electromagnéticas para verificar su inmunidad a este tipo de interferencias. Se comprobó que el teclado seguía funcionando correctamente incluso en presencia de campos electromagnéticos externos no muy fuertes.

En resumen, las pruebas eléctricas confirmaron la integridad y el correcto funcionamiento del teclado diseñado, asegurando su fiabilidad y rendimiento en diferentes condiciones de operación.

\section{Pruebas en \gls{Windows}}

Las pruebas en el sistema operativo \gls{Windows} se llevaron a cabo para verificar la compatibilidad y funcionalidad del teclado diseñado en este entorno. A continuación, se detallan las pruebas realizadas. Se ha seguido una plantilla típica para rellenar, se puede ver en la tabla \ref{Table:PruebaSistemaWindows}.

\subsubsection{Prueba de Funcionamiento de las Teclas}

Se probó cada tecla del teclado para asegurar que todas fueran reconocidas correctamente por el sistema operativo \gls{Windows}. Se verificó que la pulsación de cada tecla generara la salida esperada en la pantalla y que no se produjeran errores de reconocimiento.

\subsubsection{Prueba de Comunicación USB}

Se verificó la comunicación entre el teclado y la computadora a través del puerto \gls{USB} en el sistema operativo \gls{Windows}. Se confirmó que el teclado fuera detectado correctamente por el sistema y que la comunicación fuera estable y sin errores.

\subsubsection{Prueba de Funcionamiento de los \gls{LED}}

Se probó el funcionamiento de los \gls{LED} indicadores del teclado en el sistema operativo \gls{Windows}. Se verificó que los \gls{LED} se encendieran y apagaran correctamente según el estado de las funciones correspondientes.

\section{Pruebas en \gls{Linux}}

Las pruebas en el sistema operativo \gls{Linux} se realizaron para garantizar la compatibilidad y funcionalidad del teclado diseñado en este entorno. A continuación, se describen las pruebas realizadas. Se ha seguido una plantilla típica para rellenar, se puede ver en la tabla \ref{Table:PruebaSistemaLinux}.

\subsubsection{Prueba de Reconocimiento del Teclado}

Se verificó que el teclado fuera reconocido correctamente por el sistema operativo \gls{Linux} al conectarlo a la computadora. Se comprobó que el sistema asignara los controladores adecuados y que el teclado estuviera listo para su uso sin necesidad de configuraciones adicionales.

\subsubsection{Prueba de Funcionamiento de las Teclas}

Se probó el funcionamiento de cada tecla del teclado en el sistema operativo \gls{Linux}. Se verificó que todas las teclas generaran la salida esperada en la pantalla y que no se produjeran errores de reconocimiento o asignación de caracteres.

\subsubsection{Prueba de Funcionamiento de los \gls{LED}}

Se verificó el funcionamiento de los \gls{LED} indicadores del teclado en el sistema operativo \gls{Linux}. Se confirmó que los \gls{LED} se encendieran y apagaran correctamente según el estado de las funciones correspondientes.

En resumen, las pruebas realizadas en ambos sistemas operativos confirmaron la compatibilidad y funcionalidad del teclado diseñado en diferentes entornos de software.

\begin{table}[h]
\small
\begin{tabular}{|l|p{2cm}|p{2.5cm}|p{3cm}|}
\hline
Nombre & Descripción & Como se realiza & Resultado \\
\hline
Continuidad & Verifica la continuidad & Utilizando un multímetro digital & Sin interrupciones \\
\hline
Resistencia & Mide la resistencia eléctrica & Con instrumentos de medición adecuados & Valores dentro de los límites aceptables \\
\hline
Corriente y Voltaje & Mide la corriente y el voltaje & Utilizando un amperímetro y un voltímetro & Corriente y voltaje dentro de los rangos esperados \\
\hline
\gls{LED}S & Verifica el funcionamiento de los \gls{LED} & Encendiendo y apagando los \gls{LED} & Emisión de luz adecuada \\
\hline
\gls{USB} & Verifica la comunicación \gls{USB} & Enviando datos desde el teclado a la computadora & Comunicación estable sin pérdida de información \\
\hline
\gls{EMI} & Prueba la inmunidad a interferencias electromagnéticas & Exponiendo el teclado a fuentes de \gls{EMI} & Funcionamiento correcto incluso en presencia de campos electromagnéticos externos \\
\hline
\end{tabular}
\caption{Pruebas eléctricas realizadas}
\label{Table:PruebasElectricas}
\end{table}

\begin{table}[!htb]
\small
\begin{tabular}{|l|p{3cm}|p{3.5cm}|l|}
\hline
Nombre           & Descripción                                               & Como se realiza                                    & Resultado \\ \hline
Reconocimiento   & Verifica el reconocimiento de las teclas en \gls{Windows}        & Se prueba cada tecla del teclado en \gls{Windows}        & OK         \\ \hline
Envío de Teclas  & Verifica el envío de teclas desde el teclado en \gls{Windows}    & Se verifica la comunicación USB en \gls{Windows}         & OK         \\ \hline
\gls{LED}S             & Verifica el funcionamiento de los \gls{LED} en \gls{Windows}           & Se prueba el encendido y apagado de los \gls{LED}        & OK         \\ \hline
\end{tabular}
\caption{Pruebas en \gls{Windows}}
\label{Table:PruebaSistemaWindows}
\end{table}

\phantom{Espacio}
\begin{table}[!htb]
\small
\begin{tabular}{|l|p{3cm}|p{3.5cm}|l|}
\hline
Nombre           & Descripción                                               & Como se realiza                                    & Resultado \\ \hline
Reconocimiento   & Verifica el reconocimiento de las teclas en \gls{Linux}        & Se prueba cada tecla del teclado en \gls{Linux}        & OK         \\ \hline
Envío de Teclas  & Verifica el envío de teclas desde el teclado en \gls{Linux}    & Se verifica la comunicación USB en \gls{Linux}         & OK         \\ \hline
\gls{LED}S             & Verifica el funcionamiento de los \gls{LED} en \gls{Linux}           & Se prueba el encendido y apagado de los \gls{LED}        & OK         \\ \hline
\end{tabular}
\caption{Pruebas en \gls{Linux}}
\label{Table:PruebaSistemaLinux}
\end{table}
    
\chapter{Mejoras}

\section{Posibles mejoras}
\subsection{Software}
Dentro de las posibles mejoras a nivel de software se podría considerar la implementación de nuevas funcionalidades o la optimización del \glsnocase{Firmware} del teclado. Por ejemplo, se podría agregar soporte para macros programables, configuración de teclas multimedia adicionales. Además, se podría explorar la posibilidad de desarrollar software complementario o driver que permita una personalización más avanzada del teclado, como la asignación de funciones específicas a cada tecla o la creación de perfiles de usuario personalizados desde una interfaz gráfica. Todo esto con el objetivo de mejorar la experiencia de usuario y la versatilidad del teclado.

También se podría considerar la implementación de un sistema de actualización de \glsnocase{Firmware} over-the-air (OTA) que permita la actualización del \glsnocase{Firmware} del teclado de forma inalámbrica, sin necesidad de conectarlo a un computador. Esto facilitaría la corrección de errores, la adición de nuevas funcionalidades y la mejora de la seguridad del teclado.

Otra posible mejora sería la implementación de un sistema de detección de fallos y errores en el teclado, que permita identificar y notificar al usuario sobre posibles problemas en el funcionamiento del teclado, como teclas atascadas, errores de conexión, entre otros. Esto permitiría una mejor experiencia de usuario y una mayor confiabilidad del teclado.

Una mejora para la pantalla sería la implementación de un sistema de brillo automático que ajuste el brillo de la pantalla de acuerdo a las condiciones de iluminación del entorno, lo que permitiría una mejor visibilidad de la información mostrada en la pantalla y una reducción del consumo de energía. Además de poder mostrar información adicional como notificaciones de mensajes, correos electrónicos, entre otros. También se podría añadir un modo de personalización de zonas de la pantalla, para que el usuario pueda elegir qué información desea mostrar en cada zona de la pantalla.

También se podría crear un software o driver que controle la iluminación del teclado, permitiendo al usuario personalizar la iluminación de cada tecla de forma individual, crear efectos de iluminación personalizados y sincronizar la iluminación con otros dispositivos compatibles. Esto permitiría una mayor personalización del teclado y una experiencia de usuario más inmersiva.

\subsection{Hardware}
Se podría explorar la posibilidad de incorporar nuevas características físicas, como una iluminación LED más avanzada con opciones de personalización adicionales o la integración de una pantalla táctil para facilitar el control de funciones adicionales. También se podría considerar la implementación de un sistema de carga inalámbrica para la batería del teclado, lo que permitiría una mayor comodidad y versatilidad en el uso del teclado.

Además, se podría explorar la posibilidad de integrar un sistema de reconocimiento de huellas dactilares en el teclado, que permita una mayor seguridad en el acceso al dispositivo y la autenticación de usuarios. Esto permitiría poder usar el teclado para desbloquear el computador o acceder a aplicaciones y servicios de forma segura.

Una mejora que he intentado implementar en el teclado es que la opción de que este use \glsnocase{Switches} ópticos, pero no he podido encontrar los componentes necesarios para poder hacer esta implementación. Actualmente, estoy buscando los componentes necesarios para poder hacer esta tarea. Los \glsnocase{Switches} ópticos son más duraderos que los \glsnocase{Switches} mecánicos y no sufren de problemas de doble pulsación. Además, los \glsnocase{Switches} ópticos son más rápidos que los \glsnocase{Switches} mecánicos, ya que no tienen partes mecánicas que se muevan y, por tanto, no tienen un tiempo de respuesta tan alto como los \glsnocase{Switches} mecánicos.

\subsection{Materiales}
En cuanto a los materiales utilizados en la fabricación del teclado, se podría considerar la utilización de materiales más resistentes y duraderos, como el aluminio o el acero inoxidable, que permitan una mayor durabilidad y resistencia del teclado ante el uso diario. También se podría explorar la posibilidad de utilizar materiales reciclados o biodegradables en la fabricación del teclado, con el objetivo de reducir el impacto ambiental de su producción y promover la sostenibilidad.

\section{Consideraciones Personales}
Durante todas las versiones que he estado desarrollando del teclado, he ido añadiendo funcionalidades que me gustaría tener en un teclado, como la pantalla \gls{OLED}, la iluminación RGB, la batería recargable, entre otras. Sin embargo, hay algunas funcionalidades que me gustaría añadir en futuras versiones del teclado, como la posibilidad de personalizar la iluminación de cada tecla de forma individual, la implementación de un sistema de carga inalámbrica y la adición de un sistema de actualización de \glsnocase{Firmware} over-the-air (OTA).

Realmente en cuanto a materiales siempre he intentado conseguir el máximo con lo mínimo. He buscado los materiales más bonitos y resistentes que he podido encontrar a un precio aceptable y que no sean muy complicados de trabajar. He probado con diferentes tipos de madera, diferentes tipos de plásticos, diferentes tipos de pinturas, etc. Siempre he tenido en mente que este teclado sea un producto que pueda durar de por vida.

Aunque hay todavía cuestiones por solventar, como las piezas que no fabrico yo. Como los interruptores, que normalmente son los primeros en fallar. Durante el desarrollo del teclado se me han ocurrido una opción de desarrollar unos \glsnocase{Switches} propios, pero eso ya es un proyecto para el futuro. Estos \glsnocase{Switches} serían de tipo óptico y en vez de usar un muelle o resorte para devolver la tecla a su posición original, usarían una columna de imanes de neodimio en una configuración de Halbach para que no afectaran a los demás \glsnocase{Switches}. Estos nos tendrían nada mecánico que sufriese un desgaste y, por tanto, teniendo en cuenta que el material del \glsnocase{Switches} sería de un material resistente, estos \glsnocase{Switches} podrían durar toda la vida.

Para conseguir un teclado que pueda pasar de generación en generación, se tendría que tener en cuenta que el teclado sea actualizable, que se puedan cambiar las piezas que se desgasten y que se puedan actualizar las funcionalidades del teclado. Para ello tendríamos que tener en cuenta que el teclado sea modular, que se puedan cambiar las piezas de forma sencilla y que se puedan añadir nuevas funcionalidades de forma sencilla. Y al fin y al cabo que el teclado sea bonito y agradable de usar. Ese es el objetivo que me propuse al principio de este proyecto y que espero poder conseguir en futuras versiones del teclado.
\chapter{Conclusiones}

Aunque este proyecto empezase hace unos meses. Las primeras versiones de este empezaron hace años. Desde que empecé a interesarme por la programación y la electrónica, siempre he querido hacer un teclado por las razones que expuse en mi motivación. Por eso, aunque no haya sido un proyecto fácil, ha sido un proyecto muy gratificante y que me ha enseñado mucho. Y más aún cuando con cada versión me doy cuenta de que puedo hacer un teclado mejor. Y que dentro de un tiempo, pueda mirar este proyecto, tomar lo que he aprendido y las posibles mejoras que propuse y volver a empezar de nuevo el viaje.

Durante el proceso de diseño y desarrollo, se han enfrentado varios desafíos y se han tomado decisiones importantes para garantizar la calidad y el rendimiento del teclado. Se han realizado pruebas exhaustivas para verificar la funcionalidad y la compatibilidad del teclado en diferentes entornos, lo que ha permitido identificar áreas de mejora y optimización. Uno de los aspectos más destacados de este proyecto ha sido la oportunidad de aplicar conocimientos teóricos aprendidos en un entorno práctico.

En cuanto a los resultados obtenidos, el teclado diseñado ha demostrado ser funcional, robusto y altamente adaptable a las necesidades del usuario. Aunque todavía hay mucho margen para mejorar, el teclado ha demostrado ser una alternativa viable a los teclados convencionales y ofrece una serie de ventajas y características únicas que lo hacen atractivo para una amplia gama de usuarios.

En conclusión, este proyecto ha sido una experiencia enriquecedora que ha permitido aplicar conocimientos teóricos en un contexto práctico y desarrollar habilidades técnicas y profesionales. El teclado diseñado representa un avance significativo en mi camino hacía alcanzar la excelencia en estos campos y me ha motivado a seguir explorando nuevas oportunidades y desafíos en el futuro. El proyecto de los teclados mecánicos personalizados ha sido un largo camino durante estos años, pero cada vez que avanzo no puedo esperar a volver a tener ganas de volver a empezar de nuevo para mejorar lo que ya he hecho.

\section{Trabajo futuro}
Con este proyecto terminado ya estoy pensando en futuras versiones del teclado. Aunque este teclado ya tiene muchas funcionalidades que me gustaría tener en un teclado, hay algunas funcionalidades que me gustaría añadir en futuras versiones del teclado. Algunas de las mejoras que me gustaría añadir en futuras versiones del teclado para que este verdaderamente sea una obra de ingeniería y diseño son; los \gls{Switches} ópticos y magnéticos, la implementación de un sistema de carga inalámbrica, la adición de un sistema de actualización de firmware over-the-air (OTA), desarrollar un driver que permita personalizar el teclado de forma gráfica desde un programa en el ordenador y la implementación de módulos complejos como diccionarios de palabras, calculadoras o control de dispositivos IoT.
%
%%\nocite{*}
%\bibliographystyle{miunsrturl}
%
\appendix
%\input{apendices/manual_usuario/manual_usuario}
%\input{apendices/paper/paper}
%\newglossaryentry{DIY}{
    name={DIY},
    description={Hace referencia a "Do It Yourself" (Hazlo tú mismo). Se trata de la práctica de crear, construir o reparar cosas por uno mismo, en lugar de comprarlas prefabricadas. Es una filosofía que fomenta la creatividad, la autonomía y la satisfacción personal a través de la realización de proyectos artesanales}
}

\newglossaryentry{Keycaps}{
    name={Keycaps},
    description={Se refiere a las tapas individuales de las teclas de un teclado. Estas tapas, a menudo personalizables, pueden tener diferentes diseños, colores o materiales para proporcionar una experiencia de escritura única y estética}
}

\newglossaryentry{Dongle}{
    name={Dongle},
    description={Un dongle es un dispositivo de hardware que se conecta a otro para proporcionar funcionalidad adicional. Comúnmente, se utiliza para referirse a pequeños dispositivos que permiten la conexión inalámbrica o la adaptación de interfaces, como un dongle USB para conectividad Bluetooth}
}

\newglossaryentry{USB}{
    name={USB},
    description={Siglas de ``Universal Serial Bus`` (Bus Universal en Serie). Es un estándar de conexión que permite la transferencia de datos y la conexión de dispositivos electrónicos, como impresoras, cámaras y dispositivos de almacenamiento, a través de un cable estándar}
}

\newglossaryentry{PS2}{
    name={PS2},
    description={Se refiere al conector y protocolo de conexión utilizado comúnmente en teclados y ratones. Aunque ha sido ampliamente reemplazado por conexiones USB, el término PS/2 todavía se utiliza para referirse a dispositivos más antiguos o a sistemas compatibles con este estándar}
}

\newglossaryentry{Bluetooth}{
    name={Bluetooth},
    description={Una tecnología de comunicación inalámbrica de corto alcance que permite la transmisión de datos entre dispositivos electrónicos. El Bluetooth se utiliza comúnmente para la conexión de dispositivos como auriculares, altavoces, teclados y ratones sin necesidad de cables}
}

\newglossaryentry{Vintage}{
    name={Vintage},
    description={Se refiere a objetos, productos o estilos que tienen una cierta edad y que son considerados clásicos o representativos de una época pasada. En el contexto de la tecnología, se utiliza para describir dispositivos antiguos que tienen un atractivo nostálgico o colección}
}

\newglossaryentry{QWERTY}{
    name={QWERTY},
    description={El qwerty es un nombre que se le da a una disposición de las teclas del teclado de las máquinas de escribir que fue patentado en 1878 por Christopher Sholes. Fue el inventor de la máquina de escribir y el precursor del teclado moderno que conocemos hoy en día}
}

\newglossaryentry{Hot-Plugging}{
    name={Hot-plugging},
    description={La capacidad de conectar o desconectar un dispositivo mientras el sistema está en funcionamiento sin necesidad de reiniciar}
}

\newglossaryentry{ISO}{
    name={ISO},
    description={Organización Internacional de Normalización, una entidad que establece estándares para asegurar la calidad y la eficiencia de productos y servicios}
}

\newglossaryentry{Windows}{
    name={Windows},
    description={Un sistema operativo desarrollado por Microsoft que es ampliamente utilizado en computadoras personales}
}

\newglossaryentry{Linux}{
    name={Linux},
    description={Un sistema operativo de código abierto basado en el kernel Linux y utilizado en una variedad de dispositivos, desde servidores hasta dispositivos embebidos}
}

\newglossaryentry{Firmware}{
    name={Firmware},
    description={Software de bajo nivel almacenado en la memoria de hardware que proporciona control básico para los componentes del dispositivo}
}

\newglossaryentry{Ion de litio}{
    name={Ion de Litio},
    description={Un tipo de tecnología de batería recargable comúnmente utilizada en dispositivos electrónicos debido a su alta densidad de energía}
}

\newglossaryentry{Plug-and-Play}{
    name={Plug-and-Play},
    description={La capacidad de un sistema para reconocer automáticamente e instalar dispositivos sin intervención del usuario}
}

\newglossaryentry{Controladores}{
    name={Controladores},
    description={Software que permite la comunicación entre el sistema operativo y el hardware, permitiendo que los dispositivos funcionen correctamente}
}

\newglossaryentry{TTL}{
    name={TTL},
    description={(Transistor-Transistor Logic) también se utiliza para referirse a un programador USB, que permite cargar nuevo firmware en dispositivos. Este programador puede ser empleado en otros proyectos de "bare bones Arduino" o como una interfaz serie USB a TTL de propósito general. Proporciona una conexión y comunicación serial para la programación y configuración de dispositivos electrónicos}
}


\newglossaryentry{LED}{
    name={LED},
    description={Diodo Emisor de Luz, un dispositivo semiconductor que emite luz cuando una corriente eléctrica pasa a través de él}
}

\newglossaryentry{HID}{
    name={HID},
    description={Interfaz Humano-Computadora, un protocolo que permite la comunicación entre dispositivos de entrada, como teclados y ratones, y la computadora}
}

\newglossaryentry{PCB}{
    name={PCB},
    description={Placa de Circuito Impreso, un componente que proporciona conexiones eléctricas entre varios componentes en dispositivos electrónicos}
}

\newglossaryentry{Fresado}{
    name={Fresado},
    description={Un proceso de fabricación que utiliza una herramienta de corte rotativa para dar forma a materiales como metal o plástico}
}

\newglossaryentry{Latencia}{
    name={Latencia},
    description={El tiempo que tarda un sistema en responder a una solicitud después de recibirla, a menudo asociado con retrasos en la transmisión de datos}
}

\newglossaryentry{Switches}{
    name={Switches},
    description={Los switches son los mecanismos debajo de cada tecla de un teclado que detectan cuándo se presiona una tecla y envían la señal al dispositivo electrónico}
}

\newglossaryentry{Plate}{
    name={Plate},
    description={En el contexto de los teclados personalizados o mecánicos, un "plate" se refiere a una pieza de material (como metal, plástico o acrílico) que se coloca debajo de las teclas y sobre la placa base del teclado. El plate proporciona rigidez estructural al teclado y determina la disposición física de las teclas. Además de su función estructural, el diseño del plate también puede afectar la sensación táctil y la respuesta de las teclas al ser presionadas, lo que lo convierte en un componente importante para los entusiastas de los teclados personalizados}
}

\newglossaryentry{Online}{
    name={Online},
    description={En el contexto de la tecnología y la informática, "Online" se refiere a la condición de estar conectado a Internet o a una red informática. Cuando un dispositivo está "online", puede comunicarse con otros dispositivos o acceder a recursos en la red, como sitios web, servicios en la nube, aplicaciones en línea, etc}
}

\newglossaryentry{CNC}{
    name={CNC},
    description={Son las siglas en inglés de "Control Numérico por Computadora" (Computer Numerical Control). Se refiere a un sistema automatizado de control de máquinas herramienta, como fresadoras, tornos y cortadoras láser, mediante el uso de un programa computarizado. Los sistemas CNC son capaces de ejecutar operaciones de mecanizado de alta precisión basadas en instrucciones digitales, lo que los hace indispensables en la fabricación moderna}
}

\newglossaryentry{Deep Sleep}{
    name={Deep Sleep},
    description={En el contexto de los microcontroladores y sistemas embebidos, "Deep Sleep" (sueño profundo) es un modo de bajo consumo de energía diseñado para minimizar el consumo de energía cuando el dispositivo no está activamente realizando tareas. Durante el sueño profundo, el microcontrolador reduce su consumo de energía al mínimo al apagar la mayoría de sus funciones y circuitos, permitiendo que el dispositivo permanezca en un estado de reposo prolongado hasta que se activa por una interrupción externa, como una señal de temporizador o una entrada de sensor. Esta funcionalidad es fundamental en aplicaciones de bajo consumo de energía, como dispositivos portátiles, sensores remotos y sistemas alimentados por batería, donde se busca maximizar la vida útil de la batería o minimizar la dependencia de fuentes de energía externas}
}

\newglossaryentry{WiFi}{
    name={WiFi},
    description={Es una tecnología de comunicación inalámbrica que permite la conexión a Internet y la red de computadoras sin la necesidad de cables físicos. Utiliza ondas de radio de alta frecuencia para transmitir datos entre dispositivos, como computadoras, teléfonos inteligentes, tabletas y dispositivos de red, dentro de un área determinada llamada "zona de cobertura WiFi"}
}

\newglossaryentry{IoT}{
    name={IoT},
    description={Son las siglas en inglés de "Internet of Things" (Internet de las cosas). Se refiere a la red de dispositivos físicos que están integrados con sensores, software y otros componentes tecnológicos que les permiten conectarse y intercambiar datos a través de Internet. Estos dispositivos pueden incluir desde electrodomésticos y dispositivos portátiles hasta vehículos y equipos industriales. La tecnología IoT permite la recopilación, monitorización y control remoto de datos en tiempo real, lo que ofrece una amplia gama de aplicaciones en diversos sectores, como el hogar inteligente, la salud, la agricultura, la industria, la logística y la ciudad inteligente, entre otros}
}

\newglossaryentry{Arduino}{
    name={Arduino},
    description={Es una plataforma de hardware y software de código abierto diseñada para facilitar el desarrollo de proyectos electrónicos interactivos. Consiste en una placa de circuito impreso con un microcontrolador y un entorno de desarrollo integrado (IDE) que simplifica la programación y la interacción con los componentes electrónicos. Arduino es ampliamente utilizado por aficionados, estudiantes y profesionales para crear una amplia variedad de dispositivos y sistemas, desde simples proyectos de bricolaje hasta complejas aplicaciones de automatización y robótica}
}

\newglossaryentry{PlatformIO}{
    name={PlatformIO},
    description={Es un entorno de desarrollo integrado (IDE) de código abierto para el desarrollo de software embebido y aplicaciones IoT. Proporciona herramientas y funcionalidades para escribir, compilar, depurar y cargar código en una variedad de plataformas de hardware, incluyendo Arduino, ESP8266, ESP32, Raspberry Pi y muchas otras. PlatformIO ofrece una interfaz unificada y fácil de usar que facilita el desarrollo y la gestión de proyectos de hardware y software en múltiples plataformas, lo que lo convierte en una opción popular entre los desarrolladores de sistemas embebidos y de IoT}
}

\newglossaryentry{Wearable}{
    name={Wearable},
    description={Dispositivo electrónico vestible que se lleva encima o se incorpora en la ropa y que tiene capacidades de computación y conectividad. Los wearables suelen estar diseñados para monitorizar datos relacionados con la salud, el fitness, la ubicación, entre otros, y pueden incluir dispositivos como smartwatches, brazaletes de fitness y gafas inteligentes}
}

\newglossaryentry{Polling}{
    name={Polling},
    description={El polling rate de un teclado se refiere a la frecuencia con la que el teclado envía información al dispositivo al que está conectado. Es medida en hercios (Hz) y representa cuántas veces por segundo el teclado actualiza su estado y envía los datos correspondientes al dispositivo. Un polling rate más alto significa una respuesta más rápida del teclado, lo que puede resultar en una experiencia de usuario más suave y receptiva, especialmente en aplicaciones que requieren una entrada rápida y precisa, como los juegos}
}

\newglossaryentry{LCD}{
    name={LCD},
    description={Son las siglas en inglés de "Liquid Crystal Display" (Pantalla de Cristal Líquido). Se trata de una tecnología de visualización que utiliza cristales líquidos entre dos láminas de material polarizado para producir imágenes. Los LCD son comúnmente utilizados en dispositivos como televisores, monitores de computadora, relojes digitales y paneles de instrumentos de vehículos}
}

\newglossaryentry{OLED}{
    name={OLED},
    description={Son las siglas en inglés de "Organic Light-Emitting Diode" (Diodo Orgánico de Emisión de Luz). Se trata de una tecnología de visualización que utiliza diodos orgánicos para emitir luz y producir imágenes. A diferencia de los LCD, los OLED no requieren retroiluminación, lo que les permite ofrecer colores más vibrantes, negros más profundos y un mejor contraste. Los OLED son utilizados en dispositivos como teléfonos inteligentes, televisores de alta gama y pantallas de dispositivos portátiles}
}

\newglossaryentry{TFT}{
    name={TFT},
    description={Son las siglas en inglés de "Thin-Film Transistor" (Transistor de Película Fina). Se refiere a una tecnología de pantalla que utiliza transistores de película delgada para controlar cada píxel de la pantalla de manera individual. Los paneles TFT son comúnmente utilizados en pantallas de cristal líquido (LCD) para mejorar la calidad de imagen, aumentar la velocidad de respuesta y permitir una mayor variedad de colores. Los TFT son ampliamente utilizados en dispositivos como monitores de computadora, televisores y pantallas de dispositivos móviles}
}

\newglossaryentry{Multiplexor}{
    name={Multiplexor},
    description={Un multiplexor es un dispositivo electrónico que permite seleccionar una de varias señales de entrada y conectarla a una única salida. Es comúnmente utilizado en electrónica digital para reducir el número de líneas de control necesarias para seleccionar entre múltiples fuentes de datos}
}

\newglossaryentry{Termoretráctil}{
    name={Termoretráctil},
    description={El termoretráctil es un tipo de material plástico que se contrae cuando se calienta, generalmente mediante el uso de una pistola de calor o una fuente de calor similar. Es ampliamente utilizado en la industria electrónica para proteger y aislar conexiones eléctricas y componentes, proporcionando una capa adicional de resistencia al agua, aislamiento y soporte mecánico},
    text={Termoretráctil}
}

\newglossaryentry{API}{
    name={API},
    description={Una API (Interfaz de Programación de Aplicaciones) es un conjunto de definiciones y protocolos que permiten a los distintos componentes de software comunicarse entre sí. Proporciona una interfaz estandarizada para la interacción entre aplicaciones, permitiendo a los desarrolladores acceder a funciones específicas o datos de un software sin necesidad de conocer los detalles internos de su implementación. Las APIs son ampliamente utilizadas en el desarrollo de software para integrar servicios y funcionalidades de diferentes aplicaciones de manera eficiente y coherente}
}

\newglossaryentry{SMD}{
    name={SMD},
    description={Surface Mounted Device, que en inglés significa dispositivo de montaje superficial y se refiere tanto a una forma de encapsulado de componentes electrónicos, como a los equipos construidos a partir de estos componentes}
}

\newglossaryentry{Footprint}{
    name={Footprint},
    description={Un footprint en el contexto de diseño de PCB se refiere al señalamiento ó ubicación de los pads y otros elementos de conexión en la superficie de la placa de circuito impreso (PCB) para un componente electrónico específico}
}

\newglossaryentry{Ghosting}{
    name={Ghosting},
    description={El ghosting se produce cuando el teclado envía solo un comando al PC tras presionar varias teclas a la vez. Por ejemplo, si pulsamos las teclas S + D + F, manda la tecla “S”, por ser la primera pulsada. Otros teclados no mandan ningún comando cuando pulsamos varias a la vez}
}

\newglossaryentry{EMI}{
    name={EMI},
    description={Interferencia Electromagnética, que se refiere a la interferencia causada por campos electromagnéticos que pueden afectar el funcionamiento de dispositivos electrónicos}
}

\newglossaryentry{PID}{
    name={PID},
    description={Product ID, es un identificador único asignado a un dispositivo USB para identificarlo de manera única entre otros dispositivos conectados al mismo sistema}
}

\newglossaryentry{VID}{
    name={VID},
    description={Vendor ID, que se refiere al identificador de proveedor de un dispositivo USB}
}

\newglossaryentry{BackfeedCurrent}{
    name={Back Feed Current},
    description={Corriente de retroalimentación, que se refiere a la corriente que fluye en sentido contrario a través de un dispositivo o circuito, lo que puede causar daños o interferencias en el sistema},
    text={Back feed Current}
}

\newglossaryentry{Pads}{
    name={Pads},
    description={Son las áreas metálicas en una placa de circuito impreso (PCB) donde se sueldan los componentes electrónicos}
}

\newglossaryentry{EEPROM}{
    name={EEPROM},
    description={Son las siglas en inglés de "Electrically Erasable Programmable Read-Only Memory" (Memoria de sólo lectura programable y borrable eléctricamente). Se trata de un tipo de memoria no volátil que puede ser programada, leída y borrada eléctricamente, lo que la hace ideal para almacenar configuraciones, datos y firmware en dispositivos electrónicos. La EEPROM es ampliamente utilizada en microcontroladores y sistemas embebidos para almacenar información que debe persistir incluso cuando el dispositivo se apaga o reinicia}
}

\newglossaryentry{String}{
    name={String},
    description={En programación, una cadena (string) es una secuencia de caracteres, como letras, números y símbolos, que se utilizan para representar texto o datos en un lenguaje de programación. Las cadenas son un tipo de dato fundamental en la mayoría de los lenguajes de programación y se utilizan para almacenar y manipular información de texto, como nombres, mensajes, direcciones, etc}
}

\newglossaryentry{Int}{
    name={Int},
    description={En programación, "int" es un tipo de dato que se utiliza para representar números enteros, es decir, números sin parte decimal. Dependiendo del lenguaje de programación, el rango de valores que puede almacenar un entero puede variar, pero generalmente se utiliza para representar números enteros positivos y negativos}
}

\newglossaryentry{DEBUG}{
    name={DEBUG},
    description={En programación, "DEBUG" se refiere al proceso de identificar y corregir errores o problemas en el código de un programa. El debugging es una parte fundamental del desarrollo de software y se realiza utilizando herramientas y técnicas especializadas para rastrear, analizar y solucionar problemas en el código}
}
\printglossary
\printnoidxglossaries
\addcontentsline{toc}{chapter}{Glosario}
\listoffigures
\addcontentsline{toc}{chapter}{Indice de figuras}
%\input{Bibliografia/Bibliografia}
\bibliographystyle{plain}
\bibliography{Bibliografia/Bibliografia}
\addcontentsline{toc}{chapter}{Bibliografía}

\chapter*{}
\thispagestyle{empty}

\end{document}
