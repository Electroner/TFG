\newglossaryentry{DIY}{
    name={DIY},
    description={Hace referencia a "Do It Yourself" (Hazlo tú mismo). Se trata de la práctica de crear, construir o reparar cosas por uno mismo, en lugar de comprarlas prefabricadas. Es una filosofía que fomenta la creatividad, la autonomía y la satisfacción personal a través de la realización de proyectos artesanales}
}

\newglossaryentry{Keycaps}{
    name={Keycaps},
    description={Se refiere a las tapas individuales de las teclas de un teclado. Estas tapas, a menudo personalizables, pueden tener diferentes diseños, colores o materiales para proporcionar una experiencia de escritura única y estética}
}

\newglossaryentry{Dongle}{
    name={Dongle},
    description={Un dongle es un dispositivo de hardware que se conecta a otro para proporcionar funcionalidad adicional. Comúnmente, se utiliza para referirse a pequeños dispositivos que permiten la conexión inalámbrica o la adaptación de interfaces, como un dongle USB para conectividad Bluetooth}
}

\newglossaryentry{USB}{
    name={USB},
    description={Siglas de ``Universal Serial Bus`` (Bus Universal en Serie). Es un estándar de conexión que permite la transferencia de datos y la conexión de dispositivos electrónicos, como impresoras, cámaras y dispositivos de almacenamiento, a través de un cable estándar}
}

\newglossaryentry{PS2}{
    name={PS2},
    description={Se refiere al conector y protocolo de conexión utilizado comúnmente en teclados y ratones. Aunque ha sido ampliamente reemplazado por conexiones USB, el término PS/2 todavía se utiliza para referirse a dispositivos más antiguos o a sistemas compatibles con este estándar}
}

\newglossaryentry{Bluetooth}{
    name={Bluetooth},
    description={Una tecnología de comunicación inalámbrica de corto alcance que permite la transmisión de datos entre dispositivos electrónicos. El Bluetooth se utiliza comúnmente para la conexión de dispositivos como auriculares, altavoces, teclados y ratones sin necesidad de cables}
}

\newglossaryentry{Vintage}{
    name={Vintage},
    description={Se refiere a objetos, productos o estilos que tienen una cierta edad y que son considerados clásicos o representativos de una época pasada. En el contexto de la tecnología, se utiliza para describir dispositivos antiguos que tienen un atractivo nostálgico o colección}
}

\newglossaryentry{QWERTY}{
    name={QWERTY},
    description={El qwerty es un nombre que se le da a una disposición de las teclas del teclado de las máquinas de escribir que fue patentado en 1878 por Christopher Sholes. Fue el inventor de la máquina de escribir y el precursor del teclado moderno que conocemos hoy en día}
}

\newglossaryentry{Hot-Plugging}{
    name={Hot-plugging},
    description={La capacidad de conectar o desconectar un dispositivo mientras el sistema está en funcionamiento sin necesidad de reiniciar.}
}

\newglossaryentry{ISO}{
    name={ISO},
    description={Organización Internacional de Normalización, una entidad que establece estándares para asegurar la calidad y la eficiencia de productos y servicios.}
}

\newglossaryentry{Windows}{
    name={Windows},
    description={Un sistema operativo desarrollado por Microsoft que es ampliamente utilizado en computadoras personales.}
}

\newglossaryentry{Linux}{
    name={Linux},
    description={Un sistema operativo de código abierto basado en el kernel Linux y utilizado en una variedad de dispositivos, desde servidores hasta dispositivos embebidos.}
}

\newglossaryentry{Firmware}{
    name={Firmware},
    description={Software de bajo nivel almacenado en la memoria de hardware que proporciona control básico para los componentes del dispositivo.}
}

\newglossaryentry{Ion de litio}{
    name={Ion de Litio},
    description={Un tipo de tecnología de batería recargable comúnmente utilizada en dispositivos electrónicos debido a su alta densidad de energía.}
}

\newglossaryentry{Plug-and-Play}{
    name={Plug-and-Play},
    description={La capacidad de un sistema para reconocer automáticamente e instalar dispositivos sin intervención del usuario.}
}

\newglossaryentry{Controladores}{
    name={Controladores},
    description={Software que permite la comunicación entre el sistema operativo y el hardware, permitiendo que los dispositivos funcionen correctamente.}
}

\newglossaryentry{TTL}{
    name={TTL},
    description={TTL (Transistor-Transistor Logic) también se utiliza para referirse a un programador USB, que permite cargar nuevo firmware en dispositivos. Este programador puede ser empleado en otros proyectos de "bare bones Arduino" o como una interfaz serie USB a TTL de propósito general. Proporciona una conexión y comunicación serial para la programación y configuración de dispositivos electrónicos.}
}


\newglossaryentry{LED}{
    name={LED},
    description={Diodo Emisor de Luz, un dispositivo semiconductor que emite luz cuando una corriente eléctrica pasa a través de él.}
}

\newglossaryentry{HID}{
    name={HID},
    description={Interfaz Humano-Computadora, un protocolo que permite la comunicación entre dispositivos de entrada, como teclados y ratones, y la computadora.}
}

\newglossaryentry{PCB}{
    name={PCB},
    description={Placa de Circuito Impreso, un componente que proporciona conexiones eléctricas entre varios componentes en dispositivos electrónicos.}
}

\newglossaryentry{Fresado}{
    name={Fresado},
    description={Un proceso de fabricación que utiliza una herramienta de corte rotativa para dar forma a materiales como metal o plástico.}
}

\newglossaryentry{Latencia}{
    name={Latencia},
    description={El tiempo que tarda un sistema en responder a una solicitud después de recibirla, a menudo asociado con retrasos en la transmisión de datos.}
}

\newglossaryentry{Switches}{
    name={Switches},
    description={Los switches son los mecanismos debajo de cada tecla de un teclado que detectan cuándo se presiona una tecla y envían la señal al dispositivo electrónico.}
}

\newglossaryentry{Plate}{
    name={Plate},
    description={En el contexto de los teclados personalizados o mecánicos, un "plate" se refiere a una pieza de material (como metal, plástico o acrílico) que se coloca debajo de las teclas y sobre la placa base del teclado. El plate proporciona rigidez estructural al teclado y determina la disposición física de las teclas. Además de su función estructural, el diseño del plate también puede afectar la sensación táctil y la respuesta de las teclas al ser presionadas, lo que lo convierte en un componente importante para los entusiastas de los teclados personalizados}
}

\newglossaryentry{Online}{
    name={Online},
    description={En el contexto de la tecnología y la informática, "Online" se refiere a la condición de estar conectado a Internet o a una red informática. Cuando un dispositivo está "online", puede comunicarse con otros dispositivos o acceder a recursos en la red, como sitios web, servicios en la nube, aplicaciones en línea, etc}
}

\newglossaryentry{CNC}{
    name={CNC},
    description={CNC son las siglas en inglés de "Control Numérico por Computadora" (Computer Numerical Control). Se refiere a un sistema automatizado de control de máquinas herramienta, como fresadoras, tornos y cortadoras láser, mediante el uso de un programa computarizado. Los sistemas CNC son capaces de ejecutar operaciones de mecanizado de alta precisión basadas en instrucciones digitales, lo que los hace indispensables en la fabricación moderna}
}

\newglossaryentry{Deep Sleep}{
    name={Deep Sleep},
    description={En el contexto de los microcontroladores y sistemas embebidos, "Deep Sleep" (sueño profundo) es un modo de bajo consumo de energía diseñado para minimizar el consumo de energía cuando el dispositivo no está activamente realizando tareas. Durante el sueño profundo, el microcontrolador reduce su consumo de energía al mínimo al apagar la mayoría de sus funciones y circuitos, permitiendo que el dispositivo permanezca en un estado de reposo prolongado hasta que se activa por una interrupción externa, como una señal de temporizador o una entrada de sensor. Esta funcionalidad es fundamental en aplicaciones de bajo consumo de energía, como dispositivos portátiles, sensores remotos y sistemas alimentados por batería, donde se busca maximizar la vida útil de la batería o minimizar la dependencia de fuentes de energía externas}
}

\newglossaryentry{WiFi}{
    name={WiFi},
    description={WiFi es una tecnología de comunicación inalámbrica que permite la conexión a Internet y la red de computadoras sin la necesidad de cables físicos. Utiliza ondas de radio de alta frecuencia para transmitir datos entre dispositivos, como computadoras, teléfonos inteligentes, tabletas y dispositivos de red, dentro de un área determinada llamada "zona de cobertura WiFi"}
}

\newglossaryentry{IoT}{
    name={IoT},
    description={IoT son las siglas en inglés de "Internet of Things" (Internet de las cosas). Se refiere a la red de dispositivos físicos que están integrados con sensores, software y otros componentes tecnológicos que les permiten conectarse y intercambiar datos a través de Internet. Estos dispositivos pueden incluir desde electrodomésticos y dispositivos portátiles hasta vehículos y equipos industriales. La tecnología IoT permite la recopilación, monitorización y control remoto de datos en tiempo real, lo que ofrece una amplia gama de aplicaciones en diversos sectores, como el hogar inteligente, la salud, la agricultura, la industria, la logística y la ciudad inteligente, entre otros}
}

\newglossaryentry{Arduino}{
    name={Arduino},
    description={Arduino es una plataforma de hardware y software de código abierto diseñada para facilitar el desarrollo de proyectos electrónicos interactivos. Consiste en una placa de circuito impreso con un microcontrolador y un entorno de desarrollo integrado (IDE) que simplifica la programación y la interacción con los componentes electrónicos. Arduino es ampliamente utilizado por aficionados, estudiantes y profesionales para crear una amplia variedad de dispositivos y sistemas, desde simples proyectos de bricolaje hasta complejas aplicaciones de automatización y robótica}
}

\newglossaryentry{PlatformIO}{
    name={PlatformIO},
    description={PlatformIO es un entorno de desarrollo integrado (IDE) de código abierto para el desarrollo de software embebido y aplicaciones IoT. Proporciona herramientas y funcionalidades para escribir, compilar, depurar y cargar código en una variedad de plataformas de hardware, incluyendo Arduino, ESP8266, ESP32, Raspberry Pi y muchas otras. PlatformIO ofrece una interfaz unificada y fácil de usar que facilita el desarrollo y la gestión de proyectos de hardware y software en múltiples plataformas, lo que lo convierte en una opción popular entre los desarrolladores de sistemas embebidos y de IoT}
}

\newglossaryentry{Wearable}{
    name={Wearable},
    description={Dispositivo electrónico vestible que se lleva encima o se incorpora en la ropa y que tiene capacidades de computación y conectividad. Los wearables suelen estar diseñados para monitorizar datos relacionados con la salud, el fitness, la ubicación, entre otros, y pueden incluir dispositivos como smartwatches, brazaletes de fitness y gafas inteligentes}
}

\newglossaryentry{Polling}{
    name={Polling},
    description={El polling rate de un teclado se refiere a la frecuencia con la que el teclado envía información al dispositivo al que está conectado. Es medida en hercios (Hz) y representa cuántas veces por segundo el teclado actualiza su estado y envía los datos correspondientes al dispositivo. Un polling rate más alto significa una respuesta más rápida del teclado, lo que puede resultar en una experiencia de usuario más suave y receptiva, especialmente en aplicaciones que requieren una entrada rápida y precisa, como los juegos.}
}

\newglossaryentry{LCD}{
    name={LCD},
    description={LCD son las siglas en inglés de "Liquid Crystal Display" (Pantalla de Cristal Líquido). Se trata de una tecnología de visualización que utiliza cristales líquidos entre dos láminas de material polarizado para producir imágenes. Los LCD son comúnmente utilizados en dispositivos como televisores, monitores de computadora, relojes digitales y paneles de instrumentos de vehículos}
}

\newglossaryentry{OLED}{
    name={OLED},
    description={OLED son las siglas en inglés de "Organic Light-Emitting Diode" (Diodo Orgánico de Emisión de Luz). Se trata de una tecnología de visualización que utiliza diodos orgánicos para emitir luz y producir imágenes. A diferencia de los LCD, los OLED no requieren retroiluminación, lo que les permite ofrecer colores más vibrantes, negros más profundos y un mejor contraste. Los OLED son utilizados en dispositivos como teléfonos inteligentes, televisores de alta gama y pantallas de dispositivos portátiles}
}

\newglossaryentry{TFT}{
    name={TFT},
    description={TFT son las siglas en inglés de "Thin-Film Transistor" (Transistor de Película Fina). Se refiere a una tecnología de pantalla que utiliza transistores de película delgada para controlar cada píxel de la pantalla de manera individual. Los paneles TFT son comúnmente utilizados en pantallas de cristal líquido (LCD) para mejorar la calidad de imagen, aumentar la velocidad de respuesta y permitir una mayor variedad de colores. Los TFT son ampliamente utilizados en dispositivos como monitores de computadora, televisores y pantallas de dispositivos móviles}
}

\newglossaryentry{Multiplexor}{
    name={Multiplexor},
    description={Un multiplexor es un dispositivo electrónico que permite seleccionar una de varias señales de entrada y conectarla a una única salida. Es comúnmente utilizado en electrónica digital para reducir el número de líneas de control necesarias para seleccionar entre múltiples fuentes de datos.}
}

\newglossaryentry{Termoretractil}{
    name={Termoretractil},
    description={El termorretráctil es un tipo de material plástico que se contrae cuando se calienta, generalmente mediante el uso de una pistola de calor o una fuente de calor similar. Es ampliamente utilizado en la industria electrónica para proteger y aislar conexiones eléctricas y componentes, proporcionando una capa adicional de resistencia al agua, aislamiento y soporte mecánico.}
}

\newglossaryentry{API}{
    name={API},
    description={Una API (Interfaz de Programación de Aplicaciones) es un conjunto de definiciones y protocolos que permiten a los distintos componentes de software comunicarse entre sí. Proporciona una interfaz estandarizada para la interacción entre aplicaciones, permitiendo a los desarrolladores acceder a funciones específicas o datos de un software sin necesidad de conocer los detalles internos de su implementación. Las APIs son ampliamente utilizadas en el desarrollo de software para integrar servicios y funcionalidades de diferentes aplicaciones de manera eficiente y coherente.}
}

\newglossaryentry{SMD}{
    name={SMD},
    description={Surface Mounted Device, que en inglés significa dispositivo de montaje superficial y se refiere tanto a una forma de encapsulado de componentes electrónicos, como a los equipos construidos a partir de estos componentes.}
}

\newglossaryentry{Footprint}{
    name={Footprint},
    description={Un footprint en el contexto de diseño de PCB se refiere al señalamiento ó ubicación de los pads y otros elementos de conexión en la superficie de la placa de circuito impreso (PCB) para un componente electrónico específico.}
}

\newglossaryentry{Ghosting}{
    name={Ghosting},
    description={El ghosting se produce cuando el teclado envía solo un comando al PC tras presionar varias teclas a la vez. Por ejemplo, si pulsamos las teclas S + D + F, manda la tecla “S”, por ser la primera pulsada. Otros teclados no mandan ningún comando cuando pulsamos varias a la vez.}
}

\newglossaryentry{EMI}{
    name={EMI},
    description={Interferencia electromagnética, que se refiere a la interferencia causada por campos electromagnéticos que pueden afectar el funcionamiento de dispositivos electrónicos.}
}