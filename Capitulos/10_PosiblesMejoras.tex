\chapter{Mejoras}

\section{Posibles mejoras}
\subsection{Software}
Dentro de las posibles mejoras a nivel de software, se podría considerar la implementación de nuevas funcionalidades o la optimización del firmware del teclado. Por ejemplo, se podría agregar soporte para macros programables, configuración de teclas multimedia adicionales. Además, se podría explorar la posibilidad de desarrollar software complementario o driver que permita una personalización más avanzada del teclado, como la asignación de funciones específicas a cada tecla o la creación de perfiles de usuario personalizados desde una interfaz gráfica. Todo esto con el objetivo de mejorar la experiencia de usuario y la versatilidad del teclado.

También se podría considerar la implementación de un sistema de actualización de firmware over-the-air (OTA) que permita la actualización del firmware del teclado de forma inalámbrica, sin necesidad de conectarlo a un computador. Esto facilitaría la corrección de errores, la adición de nuevas funcionalidades y la mejora de la seguridad del teclado.

Otra posible mejora sería la implementación de un sistema de detección de fallos y errores en el teclado, que permita identificar y notificar al usuario sobre posibles problemas en el funcionamiento del teclado, como teclas atascadas, errores de conexión, entre otros. Esto permitiría una mejor experiencia de usuario y una mayor confiabilidad del teclado.

Una mejora para la pantalla sería la implementación de un sistema de brillo automático que ajuste el brillo de la pantalla de acuerdo a las condiciones de iluminación del entorno, lo que permitiría una mejor visibilidad de la información mostrada en la pantalla y una reducción del consumo de energía. Además de poder mostrar información adicional como notificaciones de mensajes, correos electrónicos, entre otros. También se podría añadir un modo de personalización de zonas de la pantalla, para que el usuario pueda elegir qué información desea mostrar en cada zona de la pantalla.

También se podría crear un software o driver que controle la iluminación del teclado, permitiendo al usuario personalizar la iluminación de cada tecla de forma individual, crear efectos de iluminación personalizados y sincronizar la iluminación con otros dispositivos compatibles. Esto permitiría una mayor personalización del teclado y una experiencia de usuario más inmersiva.

\subsection{Hardware}
Se podría explorar la posibilidad de incorporar nuevas características físicas, como una iluminación LED más avanzada con opciones de personalización adicionales o la integración de una pantalla táctil para facilitar el control de funciones adicionales. También se podría considerar la implementación de un sistema de carga inalámbrica para la batería del teclado, lo que permitiría una mayor comodidad y versatilidad en el uso del teclado.

Además, se podría explorar la posibilidad de integrar un sistema de reconocimiento de huellas dactilares en el teclado, que permita una mayor seguridad en el acceso al dispositivo y la autenticación de usuarios. Esto permitiría poder usar el teclado para desbloquear el computador o acceder a aplicaciones y servicios de forma segura.

Una mejora que he intentado implementar en el teclado es que la opción de que este use \glsnocase{Switches} ópticos, pero no he podido encontrar los componentes necesarios para poder hacer esta implementación. Actualmente, estoy buscando los componentes necesarios para poder hacer esta implementación. Los \glsnocase{Switches} ópticos son más duraderos que los \glsnocase{Switches} mecánicos y no sufren de problemas de doble pulsación. Además, los \glsnocase{Switches} ópticos son más rápidos que los \glsnocase{Switches} mecánicos, ya que no tienen partes mecánicas que se muevan y, por tanto, no tienen un tiempo de respuesta tan alto como los \glsnocase{Switches} mecánicos.

\subsection{Materiales}
En cuanto a los materiales utilizados en la fabricación del teclado, se podría considerar la utilización de materiales más resistentes y duraderos, como el aluminio o el acero inoxidable, que permitan una mayor durabilidad y resistencia del teclado ante el uso diario. También se podría explorar la posibilidad de utilizar materiales reciclados o biodegradables en la fabricación del teclado, con el objetivo de reducir el impacto ambiental de su producción y promover la sostenibilidad.

\section{Consideraciones Personales}
Durante todas las versiones que he ido desarrollando del teclado, he ido añadiendo funcionalidades que me gustaría tener en un teclado, como la pantalla OLED, la iluminación RGB, la batería recargable, entre otras. Sin embargo, hay algunas funcionalidades que me gustaría añadir en futuras versiones del teclado, como la posibilidad de personalizar la iluminación de cada tecla de forma individual, la implementación de un sistema de carga inalámbrica y la adición de un sistema de actualización de firmware over-the-air (OTA).

Realmente en cuanto a materiales siempre he intentado conseguir el máximo con lo mínimo. He buscado los materiales más bonitos y resistentes que he podido encontrar a un precio aceptable y que no sean muy complicados de trabajar. He probado con diferentes tipos de madera, diferentes tipos de plásticos, diferentes tipos de pinturas, etc. Siempre he tenido en mente que este teclado sea un producto que pueda durar de por vida.

Aunque hay todavía cuestiones que hay que solventar, como las piezas que no fabrico yo. Como los interruptores, que normalmente son los primeros en fallar. Durante el desarrollo del teclado se me han ocurrido una opción de desarrollar unos \glsnocase{Switches} propios, pero eso ya es un proyecto para el futuro. Estos \glsnocase{Switches} serían de tipo óptico y en vez de usar un muelle o resorte para devolver la tecla a su posición original, usarían una columna de imanes de neodimio en una configuración de Halbach para que no afectaran a los demás \glsnocase{Switches}. Estos nos tendrían nada mecánico que sufriese un desgaste y, por tanto, teniendo en cuenta que el material del \glsnocase{Switches} fuera de un material resistente, estos \glsnocase{Switches} podrían durar toda la vida.

Para conseguir un teclado que pueda pasar de generación en generación, se tendría que tener en cuenta que el teclado sea actualizable, que se puedan cambiar las piezas que se desgasten y que se puedan actualizar las funcionalidades del teclado. Por eso, se tendría que tener en cuenta que el teclado sea modular, que se puedan cambiar las piezas de forma sencilla y que se puedan añadir nuevas funcionalidades de forma sencilla. Y al fin y al cabo que el teclado sea bonito y agradable de usar. Ese es el objetivo que me propuse al principio de este proyecto y que espero poder conseguir en futuras versiones del teclado.