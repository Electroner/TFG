\chapter{Conclusiones}

Aunque este proyecto empezase hace unos meses. Las primeras versiones de este empezaron hace años. Desde que empecé a interesarme por la programación y la electrónica, siempre he querido hacer un teclado por las razones que expuse en mi motivación. Por eso, aunque no haya sido un proyecto fácil, ha sido un proyecto muy gratificante y que me ha enseñado mucho. Y más aún cuando con cada versión me doy cuenta de que puedo hacer un teclado mejor. Y que dentro de un tiempo, pueda mirar este proyecto, tomar lo que he aprendido y las posibles mejoras que propuse y volver a empezar de nuevo el viaje.

Durante el proceso de diseño y desarrollo, se han enfrentado varios desafíos y se han tomado decisiones importantes para garantizar la calidad y el rendimiento del teclado. Se han realizado pruebas exhaustivas para verificar la funcionalidad y la compatibilidad del teclado en diferentes entornos, lo que ha permitido identificar áreas de mejora y optimización. Uno de los aspectos más destacados de este proyecto ha sido la oportunidad de aplicar conocimientos teóricos aprendidos en un entorno práctico.

En cuanto a los resultados obtenidos, el teclado diseñado ha demostrado ser funcional, robusto y altamente adaptable a las necesidades del usuario. Aunque todavía hay mucho margen para mejorar, el teclado ha demostrado ser una alternativa viable a los teclados convencionales y ofrece una serie de ventajas y características únicas que lo hacen atractivo para una amplia gama de usuarios.

En conclusión, este proyecto ha sido una experiencia enriquecedora que ha permitido aplicar conocimientos teóricos en un contexto práctico y desarrollar habilidades técnicas y profesionales. El teclado diseñado representa un avance significativo en mi camino hacía alcanzar la excelencia en estos campos y me ha motivado a seguir explorando nuevas oportunidades y desafíos en el futuro. El proyecto de los teclados mecánicos personalizados ha sido un largo camino durante estos años, pero cada vez que avanzo no puedo esperar a volver a tener ganas de volver a empezar de nuevo para mejorar lo que ya he hecho.

\section{Trabajo futuro}
Con este proyecto terminado ya estoy pensando en futuras versiones del teclado. Aunque este teclado ya tiene muchas funcionalidades que me gustaría tener en un teclado, hay algunas funcionalidades que me gustaría añadir en futuras versiones del teclado. Algunas de las mejoras que me gustaría añadir en futuras versiones del teclado para que este verdaderamente sea una obra de ingeniería y diseño son; los \gls{Switches} ópticos y magnéticos, la implementación de un sistema de carga inalámbrica, la adición de un sistema de actualización de firmware over-the-air (OTA), desarrollar un driver que permita personalizar el teclado de forma gráfica desde un programa en el ordenador y la implementación de módulos complejos como diccionarios de palabras, calculadoras o control de dispositivos IoT.