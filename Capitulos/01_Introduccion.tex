\chapter{Introducción}
\section{Motivación}

Cada vez es más común el uso de ordenadores y sistemas informáticos en todos los ámbitos de la vida, por lo que el uso de dispositivos para poder interactuar con estos también es igual de común. Desde hace muchísimo tiempo, el dispositivo más usado para interactuar con los ordenadores y máquinas a lo largo del mundo es y ha sido el teclado. Existen teclados de todo tipo, formas, distribuciones y tecnologías. Todos ellos con el mismo objetivo, introducir caracteres en el sistema.

Personalmente, desde que puedo recordar siempre he usado el ordenador de casa y he crecido con el teclado a mi lado. He pasado por muchos teclados a lo largo de mi vida y todos ellos han acabado más o menos de la misma forma, estropeados o guardados porque me cansaba del teclado y de como era la sensación al usarlo. Reparar estos teclados era igual de costoso que el propio problema que tenían.

Intenté buscar siempre dispositivos que satisficieran mis necesidades; robustos, fáciles de arreglar, duraderos, inalámbricos y elegantes. En el mercado existen varios de estos, pero era mi último requisito el que hacía que mi búsqueda no arrojase ningún resultado, que fuera en español. Al fin y al cabo, escribo en español y estoy acostumbrado a una distribución española, por lo que dado las exigencias que tengo como usuario y que no hay mercado para lo que busco, decidí embarcarme en este proyecto para llenar el hueco que hay en el mercado y satisfacer mis demandas.

Mi desafío no solo radica en encontrar el teclado perfecto para mí, sino también en transformar la concepción convencional de diseño y fabricación de teclados. Para abordar este problema, mi enfoque va más allá de simplemente crear otro dispositivo; mi objetivo es redefinir lo que se considera estándar mediante la implementación de nuevas técnicas y conceptos innovadores.

En lugar de adherirme a las técnicas de fabricación y ensamblaje convencionales, que a menudo resultan en dispositivos difíciles de reparar y poco robustos, planeo adoptar un enfoque disruptivo. La fabricación de la placa de circuito impreso (\gls{PCB}) será un elemento central de esta estrategia. Al diseñar una \gls{PCB} que priorice la facilidad de reparación y la durabilidad, se espera superar las limitaciones de los teclados tradicionales.

La estética del teclado también experimentará un cambio significativo en mi proyecto. En lugar de centrarme únicamente en el aspecto visual, mi enfoque se orientará hacia la funcionalidad y la simplicidad elegante. Esto implica la eliminación de partes estéticas innecesarias para priorizar la funcionalidad y la durabilidad. La forma seguirá a la función, y mi objetivo es crear un teclado que no solo sea una herramienta eficiente, sino también un testimonio de la belleza en la simplicidad.

La modularidad será una característica clave de mi diseño. La capacidad de desmontar y reemplazar fácilmente componentes permitirá a los usuarios personalizar su experiencia según sus necesidades específicas. Desde interruptores hasta placas de circuito, cada elemento será diseñado para ser modular, facilitando tanto la personalización como el mantenimiento.

En términos de conectividad, mi teclado ofrecerá versatilidad. No solo será inalámbrico, aprovechando la comodidad de la tecnología sin cables, sino que también contará con una opción por cable para situaciones donde la estabilidad de la conexión sea prioritaria. Esta dualidad busca ofrecer a los usuarios la flexibilidad necesaria para adaptarse a diferentes entornos y preferencias.

En resumen, mi proyecto busca romper las convenciones establecidas en el diseño y la fabricación de teclados. A través de la implementación de varias tecnologías, un enfoque renovado en la estética, así como la incorporación de características como la modularidad y el diseño simple. Aspiro a crear un teclado que no solo satisfaga mis necesidades y pueda llegar a llenar ese vacío en el mercado existente y crear, de una manera u otra, un producto de buena calidad.
\pagebreak

\section{Estado actual de las alternativas}
%Sección
Actualmente, existen muchos teclados, formas, distribuciones, materiales y precios. \cite{cdw-keyboards} Dado que el nuestro, aunque pueda ser fabricado en grandes cantidades, va a ser fabricado como prototipo una única vez. Vamos a mirar las opciones que se conocen como \gls{DIY} o personalizados, que son las correspondientes a este tipo de mercado donde se fabrican bajo un precio menos ajustado y en menos cantidades.

En el ámbito de los teclados \gls{DIY} o personalizados, se encuentran diversas opciones que permiten a los usuarios crear su propio dispositivo según sus preferencias y necesidades. Estas alternativas suelen destacar por ofrecer un mayor nivel de personalización en términos de diseño, disposición de teclas, interruptores y retro-iluminación, en comparación con los teclados convencionales o fabricados en masa. \cite{Diy-Keyboard-POPSC}

Una de las opciones más populares para este tipo de teclados son las placas base personalizables. Estas placas permiten a los usuarios seleccionar y soldar sus propios interruptores y estabilizadores, lo que brinda una libertad total en la elección de la disposición de las teclas. Además, suelen admitir la programación de macros y asignación de funciones a través de software.

También existe la posibilidad de elegir interruptores mecánicos específicos, siendo una característica clave. Existen diversos tipos de interruptores, como los Cherry MX, Gateron, Kailh, entre otros, cada uno con características únicas en términos de tacto, recorrido y sonoridad. Además, acompañando a estos interruptores, siempre hay lo que se denomina como \gls{Keycaps}, que son los componentes que cubren los interruptores y que al fin y al cabo es lo que el usuario acaba presionando a la hora de escribir en el teclado.

Los teclados personalizados, aunque siempre ofrecen más opciones en cuanto a preferencias o más opciones en su diseño, también suele conllevar un mayor gasto y por eso es necesario tener en cuenta qué se quiere y como pueden cambiar los precios de estos.

%\newpage
\subsection{Conectividad}
En cuanto a conectividad, no hay tanta variedad, ya que esto se refiere a la forma de conectarse a un computador, y por el momento el grueso del mercado y de casi todos los teclados que se usan de forma comercial usan 4 tipos de conexión. 
\newpage
\begin{itemize}
    \item Por Cable \cite{Keyboards-connection-types-wired}
    \begin{itemize}
        \item \gls{USB} \\
            Esta forma de conectividad representa el grueso predominante en la actualidad, ya que prácticamente todos los dispositivos recurren a esta interfaz para establecer conexión con la computadora. Se trata de un estándar en constante evolución, actualizado mediante diversas versiones a lo largo del tiempo.  (\gls{USB} 1.0, \gls{USB} 2.0, \gls{USB} 3.0, \gls{USB} 3.1, \gls{USB} 3.2 ...)
        \item \gls{PS2} \\
            La conexión \gls{PS2} fue ampliamente utilizada en la década de 1990 y principios de la década de 2000 como un estándar para conectar periféricos a computadoras, especialmente dispositivos de entrada como teclados y ratones. Aunque ha sido superada en popularidad por el \gls{USB}, el \gls{PS2} todavía se encuentra en algunos dispositivos más antiguos. Esta interfaz se caracteriza por su conector redondo con pines y ha experimentado varias revisiones a lo largo del tiempo, como el \gls{PS2} estándar y el \gls{PS2} Mini-DIN de 6 pines. A medida que la tecnología ha avanzado, el \gls{PS2} ha quedado en gran medida relegado en favor de interfaces más modernas, pero su legado persiste en sistemas heredados y dispositivos retro. 
    \end{itemize}
    \item Por radiofrecuencia \cite{Keyboards-connection-types-wireless}
    \begin{itemize}
        \item \gls{Bluetooth} \\
            La conectividad por \glsnocase{Bluetooth} ha ganado amplia aceptación en el ámbito inalámbrico, facilitando la comunicación entre dispositivos a corta distancia. Este estándar ha demostrado ser especialmente útil para la conexión de periféricos como auriculares, teclados y ratones de forma inalámbrica. Este también ha ido sufriendo actualizaciones que lo han mejorado en todos sus aspectos. Actualmente casi todos los teclados que son inalámbricos disponen de \glsnocase{Bluetooth}.
    
        \item \gls{Dongle} receptor \\
            El \gls{Dongle} receptor, también conocido como adaptador, desempeña un papel crucial al habilitar la conectividad por radiofrecuencia en dispositivos que no cuentan nativamente con la capacidad \glsnocase{Bluetooth}. Al conectar este pequeño dispositivo, se amplía la gama de dispositivos compatibles y se permite la comunicación inalámbrica. Este se trata de una antena encapsulada en un conector que añadido al dispositivo y conectado a un \gls{USB} hace de receptor, normalmente usa un protocolo que no es \glsnocase{Bluetooth} y es más específico a la aplicación, por lo que hace que su \glsnocase{Latencia} y alcance mejoren considerablemente.
    \end{itemize}
\end{itemize}

\subsection{Formatos}

A lo largo de la evolución de los teclados, las configuraciones y disposiciones de las teclas han experimentado cambios significativos para adaptarse a las necesidades cambiantes de los usuarios y a los avances tecnológicos.

Inicialmente, los teclados adoptaron el diseño \gls{QWERTY}, popularizado por las máquinas de escribir y posteriormente estandarizado por IBM. Este diseño se mantiene como el más común en la actualidad.

A lo largo de los años, surgieron alternativas, como el diseño Dvorak, que redistribuye las letras según su frecuencia de uso para aumentar la eficiencia de la escritura.

Para abordar preocupaciones ergonómicas, los teclados divididos surgieron con el objetivo de reducir la tensión al separar el teclado en dos secciones, ya sea físicamente o mediante un diseño ergonómico que coloca las secciones en ángulos más naturales para las manos.

En términos de tamaño y portabilidad, los teclados completos (100\%) ofrecen la disposición estándar, mientras que los teclados Tenkeyless (TKL) eliminan el teclado numérico para reducir el tamaño. Los teclados 75\% reducen aún más el tamaño al ajustar la disposición sin sacrificar funciones esenciales. Luego en el mundo de \gls{DIY} existen variaciones extremas de esto, pudiendo encontrar teclados de 50\%, 45\% y hasta un 35\% que solo disponen de las teclas alfabéticas.

La elección de un formato de teclado se basa en las preferencias y necesidades individuales del usuario, considerando aspectos como la ergonomía, la portabilidad y el uso previsto, ya sea para trabajo, juegos u otros propósitos específicos. \cite{Keyboards-types}

\subsection{Precios}
Cuando se trata de precios, la diversidad en el mercado de teclados refleja una amplia gama de opciones que se adaptan a diferentes presupuestos y necesidades. Desde opciones más asequibles hasta modelos de alta gama, los precios de los teclados varían según diversos factores.

Los teclados estándar con configuraciones tradicionales y cableado suelen ser más asequibles, brindando una opción económica para aquellos con presupuestos más ajustados. Estos teclados son ideales para usuarios que no necesitan características avanzadas o diseños especializados.

En el extremo superior del espectro, los teclados de gama alta ofrecen características avanzadas como retro-iluminación personalizable, interruptores mecánicos de alta calidad, y construcciones premium. Estos modelos suelen apuntar a entusiastas de los juegos, profesionales creativos o usuarios que buscan una experiencia de escritura excepcional.

Además, la introducción de teclados inalámbricos, ergonómicos o compactos también afecta los precios. Los teclados ergonómicos diseñados para reducir la fatiga y mejorar la comodidad pueden tener un coste ligeramente superior, mientras que los modelos inalámbricos ofrecen la ventaja de la movilidad a un precio adicional.

En resumen, la amplia variedad de teclados disponibles en el mercado garantiza que haya opciones para todos los presupuestos. La clave está en identificar las características prioritarias y el uso previsto para encontrar el equilibrio perfecto entre funcionalidad y precio. \cite{Keyboards-Prices}

\subsection{Otros teclados personalizados}
Siempre puede haber otra razón para buscar un teclado personalizado, entre ellas podemos encontrar motivos como diseños retro y \glsnocase{Vintage}. Teclados Temáticos o de Edición Limitada. 
También se hace si se quiere  conseguir como objetivo principal lograr una funcionalidad extra que un teclado normal no va a proveer, o de forma extraordinaria una forma de presumir de las capacidades de fabricación y diseño del autor.