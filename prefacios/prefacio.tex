\chapter*{}
%\thispagestyle{empty}
%\cleardoublepage

%\thispagestyle{empty}

\input{portada/portada_2}



\cleardoublepage
\thispagestyle{empty}

\begin{center}
{\large\bfseries Diseño y Desarrollo de un Dispositivo de Interfaz Humana: Teclado ISO 105 Español}\\
\end{center}
\begin{center}
Carlos López Martínez\\
\end{center}

%\vspace{0.7cm}
\noindent{\textbf{Palabras clave}: ISO, PlatformIO, Microcontrolador, ESP32-S3, Bluetooth, Wireless, USB, Windows, Linux}\\

\vspace{0.7cm}
\noindent{\textbf{Resumen}}\\

Creación de un dispositivo de interfaz humana siguiendo el estándar ISO 105 en español. Enfocado en la creación de un dispositivo de interfaz humana, este proyecto se propone diseñar y desarrollar un teclado conforme al estándar ISO 105 en español, ya que no existe gran variedad de estos en el mercado. Para alcanzar este objetivo, se empleará la plataforma "PlatformIO" junto con un microcontrolador ESP32-S3, aprovechando sus capacidades de conectividad y procesamiento. El dispositivo permitirá una interacción con SO Windows y Linux por USB (Cableado) y Bluetooth (Wireless)

Para todo el desarrollo del dispositivo se tendrá en cuenta el coste del mismo, ya que deberá ser abordable, al menos como producto de alta gama, así como se tendrá en cuenta su reparabilidad, estética y en todo momento el rendimiento y funcionamiento.
El dispositivo tiene que funcionar de forma correcta tanto de forma inalámbrica como cableada independientemente, pudiéndose bajar los costes si solo se desea una versión cableada.

El dispositivo dispondrá de una pantalla LCD a color para mostrar información y configuración del mismo, así como código ampliable y su asignación a teclas especiales.
\cleardoublepage


\thispagestyle{empty}


\begin{center}
{\large\bfseries Design and Development of a Human Interface Device: ISO 105 Spanish Keyboard}\\
\end{center}
\begin{center}
Carlos López Martínez\\
\end{center}

%\vspace{0.7cm}
\noindent{\textbf{Palabras clave}: ISO, PlatformIO, microcontroller, ESP32-S3, Bluetooth, Wireless, USB, Windows, Linux}\\

\vspace{0.7cm}
\noindent{\textbf{Abstract}}\\

Creation of a human interface device following the ISO 105 standard in Spanish. Focused on the creation of a human interface device, this project aims to design and develop a keyboard in accordance with the ISO 105 standard in Spanish, as there is not a wide variety of these in the market. To achieve this goal, the "PlatformIO" platform will be used along with an ESP32-S3 microcontroller, taking advantage of its connectivity and processing capabilities. The device will allow interaction with Windows and Linux OS via USB (Wired) and Bluetooth (Wireless).

Throughout the development of the device, its cost will be taken into account, as it must be affordable, at least as a high-end product. Its repairability, aesthetics, and performance and operation will also be considered at all times. The device has to function correctly both wirelessly and wired independently. Costs can be reduced if only a wired version is desired.

The device will have a color LCD screen to display information and configuration of the same. As well as expandable code and its assignment to special keys.

\chapter*{}
\thispagestyle{empty}

\noindent\rule[-1ex]{\textwidth}{2pt}\\[4.5ex]

Yo, \textbf{Carlos López Martínez}, alumno de la titulación INGENIERÍA INFORMÁTICA de la \textbf{Escuela Técnica Superior
de Ingenierías Informática y de Telecomunicación de la Universidad de Granada}, con DNI 20888530E, autorizo la
ubicación de la siguiente copia de mi Trabajo Fin de Grado en la biblioteca del centro para que pueda ser
consultada por las personas que lo deseen.

\vspace{3cm}

\begin{flushleft}
       \hspace*{1.25cm}\includegraphics[width=0.2\linewidth]{prefacios/Firma.jpg}
\end{flushleft}
\noindent Fdo: Carlos López Martínez

\vspace{2cm}

\begin{flushright}
\today
\end{flushright}


\chapter*{}
\thispagestyle{empty}

\noindent\rule[-1ex]{\textwidth}{2pt}\\[4.5ex]

D. \textbf{Jesús González Peñalver}, Catedrático del departamento de Arquitectura y Tecnología de Computadores de la Universidad de Granada.

\vspace{0.5cm}

\textbf{Informa:}

\vspace{0.5cm}

Que el presente trabajo, titulado \textit{\textbf{Diseño y Desarrollo de un Dispositivo de Interfaz Humana}},
ha sido realizado bajo su supervisión por \textbf{Carlos López Martínez}, y autoriza la defensa de dicho trabajo ante el tribunal
que corresponda.

\vspace{0.5cm}

Y para que conste, expiden y firman el presente informe en Granada a \today

\vspace{1cm}

\textbf{El director:}

\vspace{5cm}

\noindent \textbf{Jesús González Peñalver}

\chapter*{Agradecimientos}
\thispagestyle{empty}

       \vspace{1cm}

A mi familia, por todo el apoyo que me han dado a lo largo de todos
los años. A mis amigos por las ideas y momentos de inspiración. Y a
todos los profesores que me han ayudado y han logrado hacer del Grado de
Ingeniería Informática una gran experiencia.